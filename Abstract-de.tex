\begin{otherlanguage}{ngerman}
\chapter{Zusammenfassung}

In moderne Programmiersprachen, Abfragen auf Speicher-basierte Kollektion sind oft
teurer als erforderlich.
Datenbankenabfragen können einfach optimiert werden, aber Kollectionabfragen
können oft schwer, in Datenbankenabfragen umgesetzt werden, denn moderne
Programmiersprachen sind expressiver als Datenbanken, weil sie beide
verschachtelte Daten und Erste-Klasse-Funktionen unterstützen.

Kollektionabfragen können per Hand optimiert und inkrementalisieren werden,
aber das verringert Modularität, und ist oft zu fehleranfällig um
ausführbar zu sein oder um Instandhaltung von entstandene Programm zu ermöglichen.
In dieser Doktorarbeit optimieren und inkrementalisieren wir
Kollektionabfragen um Programmierer von solchen Last zu befreien.
Erzeugte Programme werden in derselbe Kernsprache ausgedruckt, damit sie weiter
standardmäßig optimiert werden können.

Um Optimierung von Kollektionabfragen zu ermöglichen, die im Programme
vorkommen, entwickeln wir eine gestaged Variante der Scala-Kollektion-API, die
Abfragen als Abstrakten Syntaxbäume konkretisiert. Auf diese Schnittstelle
anpassen wir domänenspezifische Optimierungen vom Programmiersprachen und
Datenbanken; unter anderem umschreiben wir Abfragen um
von Programmierer ausgewählte Indexe zu benutzen. Dank Indexen können wir zeigen
bedeutend Beschleunigung in unserer experimentelle Auswertung, mit einem
Durchschnitt von 12x und einem Maximum von 12800x.

Um Programme mit Funktionen von höhere Ordnung durch Programmtransformation zu
inkrementalisieren, erweitern wir \emph{Finite-Differenzen-Methode}
\citep{Paige82FDC,Blakeley:1986:EUM,Gupta99MMV} und entwickeln wir den ersten
Ansatz zur Inkrementalisierung durch Programmtransformation für Programme mit
Funktionen höherer Ordnung. Eingangsprogramme werden zu \emph{Ableitungen}
transformiert, d.h. Programme die \emph{Eingangsdifferenzen} in
\emph{Ausgangdifferenzen} umwandeln.
Wir beweisen dass unserer Ansatz zur Inkrementalisierung korrekt ist für
einfach-getypt und ungetypt λ-Kalkül, und besprechen Erweiterungen zu System~F.

Ableitungen müssen oft Ergebnisse von Eingangsprogramme wiederverwenden. Um
solche Wiederverwendung zu ermöglichen, wir erweitern Arbeit von \citet{Liu95}
zu Programme mit Funktionen höherer Ordnung. Wir entwickeln eine
Programmtransformation von solche Programme in \emph{Cache-Transfer-Stil}.

Für effiziente Inkrementalisierung muss man passende Grundroutinen auswählen und
manuell inkrementalisieren. Wir inkrementalisieren ein bedeutend Auswahl von
Grundroutinen für Kollektionen; dann durchführen wir Fallstudien, die starke
Beschleunigung in Laufzeit und asymptotische Komplexität zeigen.
\end{otherlanguage}

% Local Variables:
% ispell-local-dictionary: german
% End:
