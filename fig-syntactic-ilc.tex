% Emacs, this is -*- latex -*-!
%% ODER: format ==         = "\mathrel{==}"
%% ODER: format /=         = "\neq "
%
%
\makeatletter
\@ifundefined{lhs2tex.lhs2tex.sty.read}%
  {\@namedef{lhs2tex.lhs2tex.sty.read}{}%
   \newcommand\SkipToFmtEnd{}%
   \newcommand\EndFmtInput{}%
   \long\def\SkipToFmtEnd#1\EndFmtInput{}%
  }\SkipToFmtEnd

\newcommand\ReadOnlyOnce[1]{\@ifundefined{#1}{\@namedef{#1}{}}\SkipToFmtEnd}
\usepackage{amstext}
\usepackage{amssymb}
\usepackage{stmaryrd}
\DeclareFontFamily{OT1}{cmtex}{}
\DeclareFontShape{OT1}{cmtex}{m}{n}
  {<5><6><7><8>cmtex8
   <9>cmtex9
   <10><10.95><12><14.4><17.28><20.74><24.88>cmtex10}{}
\DeclareFontShape{OT1}{cmtex}{m}{it}
  {<-> ssub * cmtt/m/it}{}
\newcommand{\texfamily}{\fontfamily{cmtex}\selectfont}
\DeclareFontShape{OT1}{cmtt}{bx}{n}
  {<5><6><7><8>cmtt8
   <9>cmbtt9
   <10><10.95><12><14.4><17.28><20.74><24.88>cmbtt10}{}
\DeclareFontShape{OT1}{cmtex}{bx}{n}
  {<-> ssub * cmtt/bx/n}{}
\newcommand{\tex}[1]{\text{\texfamily#1}}	% NEU

\newcommand{\Sp}{\hskip.33334em\relax}


\newcommand{\Conid}[1]{\mathit{#1}}
\newcommand{\Varid}[1]{\mathit{#1}}
\newcommand{\anonymous}{\kern0.06em \vbox{\hrule\@width.5em}}
\newcommand{\plus}{\mathbin{+\!\!\!+}}
\newcommand{\bind}{\mathbin{>\!\!\!>\mkern-6.7mu=}}
\newcommand{\rbind}{\mathbin{=\mkern-6.7mu<\!\!\!<}}% suggested by Neil Mitchell
\newcommand{\sequ}{\mathbin{>\!\!\!>}}
\renewcommand{\leq}{\leqslant}
\renewcommand{\geq}{\geqslant}
\usepackage{polytable}

%mathindent has to be defined
\@ifundefined{mathindent}%
  {\newdimen\mathindent\mathindent\leftmargini}%
  {}%

\def\resethooks{%
  \global\let\SaveRestoreHook\empty
  \global\let\ColumnHook\empty}
\newcommand*{\savecolumns}[1][default]%
  {\g@addto@macro\SaveRestoreHook{\savecolumns[#1]}}
\newcommand*{\restorecolumns}[1][default]%
  {\g@addto@macro\SaveRestoreHook{\restorecolumns[#1]}}
\newcommand*{\aligncolumn}[2]%
  {\g@addto@macro\ColumnHook{\column{#1}{#2}}}

\resethooks

\newcommand{\onelinecommentchars}{\quad-{}- }
\newcommand{\commentbeginchars}{\enskip\{-}
\newcommand{\commentendchars}{-\}\enskip}

\newcommand{\visiblecomments}{%
  \let\onelinecomment=\onelinecommentchars
  \let\commentbegin=\commentbeginchars
  \let\commentend=\commentendchars}

\newcommand{\invisiblecomments}{%
  \let\onelinecomment=\empty
  \let\commentbegin=\empty
  \let\commentend=\empty}

\visiblecomments

\newlength{\blanklineskip}
\setlength{\blanklineskip}{0.66084ex}

\newcommand{\hsindent}[1]{\quad}% default is fixed indentation
\let\hspre\empty
\let\hspost\empty
\newcommand{\NB}{\textbf{NB}}
\newcommand{\Todo}[1]{$\langle$\textbf{To do:}~#1$\rangle$}

\EndFmtInput
\makeatother
%
%
%
%
%
%
% This package provides two environments suitable to take the place
% of hscode, called "plainhscode" and "arrayhscode". 
%
% The plain environment surrounds each code block by vertical space,
% and it uses \abovedisplayskip and \belowdisplayskip to get spacing
% similar to formulas. Note that if these dimensions are changed,
% the spacing around displayed math formulas changes as well.
% All code is indented using \leftskip.
%
% Changed 19.08.2004 to reflect changes in colorcode. Should work with
% CodeGroup.sty.
%
\ReadOnlyOnce{polycode.fmt}%
\makeatletter

\newcommand{\hsnewpar}[1]%
  {{\parskip=0pt\parindent=0pt\par\vskip #1\noindent}}

% can be used, for instance, to redefine the code size, by setting the
% command to \small or something alike
\newcommand{\hscodestyle}{}

% The command \sethscode can be used to switch the code formatting
% behaviour by mapping the hscode environment in the subst directive
% to a new LaTeX environment.

\newcommand{\sethscode}[1]%
  {\expandafter\let\expandafter\hscode\csname #1\endcsname
   \expandafter\let\expandafter\endhscode\csname end#1\endcsname}

% "compatibility" mode restores the non-polycode.fmt layout.

\newenvironment{compathscode}%
  {\par\noindent
   \advance\leftskip\mathindent
   \hscodestyle
   \let\\=\@normalcr
   \let\hspre\(\let\hspost\)%
   \pboxed}%
  {\endpboxed\)%
   \par\noindent
   \ignorespacesafterend}

\newcommand{\compaths}{\sethscode{compathscode}}

% "plain" mode is the proposed default.
% It should now work with \centering.
% This required some changes. The old version
% is still available for reference as oldplainhscode.

\newenvironment{plainhscode}%
  {\hsnewpar\abovedisplayskip
   \advance\leftskip\mathindent
   \hscodestyle
   \let\hspre\(\let\hspost\)%
   \pboxed}%
  {\endpboxed%
   \hsnewpar\belowdisplayskip
   \ignorespacesafterend}

\newenvironment{oldplainhscode}%
  {\hsnewpar\abovedisplayskip
   \advance\leftskip\mathindent
   \hscodestyle
   \let\\=\@normalcr
   \(\pboxed}%
  {\endpboxed\)%
   \hsnewpar\belowdisplayskip
   \ignorespacesafterend}

% Here, we make plainhscode the default environment.

\newcommand{\plainhs}{\sethscode{plainhscode}}
\newcommand{\oldplainhs}{\sethscode{oldplainhscode}}
\plainhs

% The arrayhscode is like plain, but makes use of polytable's
% parray environment which disallows page breaks in code blocks.

\newenvironment{arrayhscode}%
  {\hsnewpar\abovedisplayskip
   \advance\leftskip\mathindent
   \hscodestyle
   \let\\=\@normalcr
   \(\parray}%
  {\endparray\)%
   \hsnewpar\belowdisplayskip
   \ignorespacesafterend}

\newcommand{\arrayhs}{\sethscode{arrayhscode}}

% The mathhscode environment also makes use of polytable's parray 
% environment. It is supposed to be used only inside math mode 
% (I used it to typeset the type rules in my thesis).

\newenvironment{mathhscode}%
  {\parray}{\endparray}

\newcommand{\mathhs}{\sethscode{mathhscode}}

% texths is similar to mathhs, but works in text mode.

\newenvironment{texthscode}%
  {\(\parray}{\endparray\)}

\newcommand{\texths}{\sethscode{texthscode}}

% The framed environment places code in a framed box.

\def\codeframewidth{\arrayrulewidth}
\RequirePackage{calc}

\newenvironment{framedhscode}%
  {\parskip=\abovedisplayskip\par\noindent
   \hscodestyle
   \arrayrulewidth=\codeframewidth
   \tabular{@{}|p{\linewidth-2\arraycolsep-2\arrayrulewidth-2pt}|@{}}%
   \hline\framedhslinecorrect\\{-1.5ex}%
   \let\endoflinesave=\\
   \let\\=\@normalcr
   \(\pboxed}%
  {\endpboxed\)%
   \framedhslinecorrect\endoflinesave{.5ex}\hline
   \endtabular
   \parskip=\belowdisplayskip\par\noindent
   \ignorespacesafterend}

\newcommand{\framedhslinecorrect}[2]%
  {#1[#2]}

\newcommand{\framedhs}{\sethscode{framedhscode}}

% The inlinehscode environment is an experimental environment
% that can be used to typeset displayed code inline.

\newenvironment{inlinehscode}%
  {\(\def\column##1##2{}%
   \let\>\undefined\let\<\undefined\let\\\undefined
   \newcommand\>[1][]{}\newcommand\<[1][]{}\newcommand\\[1][]{}%
   \def\fromto##1##2##3{##3}%
   \def\nextline{}}{\) }%

\newcommand{\inlinehs}{\sethscode{inlinehscode}}

% The joincode environment is a separate environment that
% can be used to surround and thereby connect multiple code
% blocks.

\newenvironment{joincode}%
  {\let\orighscode=\hscode
   \let\origendhscode=\endhscode
   \def\endhscode{\def\hscode{\endgroup\def\@currenvir{hscode}\\}\begingroup}
   %\let\SaveRestoreHook=\empty
   %\let\ColumnHook=\empty
   %\let\resethooks=\empty
   \orighscode\def\hscode{\endgroup\def\@currenvir{hscode}}}%
  {\origendhscode
   \global\let\hscode=\orighscode
   \global\let\endhscode=\origendhscode}%

\makeatother
\EndFmtInput
%
%
%
% First, let's redefine the forall, and the dot.
%
%
% This is made in such a way that after a forall, the next
% dot will be printed as a period, otherwise the formatting
% of `comp_` is used. By redefining `comp_`, as suitable
% composition operator can be chosen. Similarly, period_
% is used for the period.
%
\ReadOnlyOnce{forall.fmt}%
\makeatletter

% The HaskellResetHook is a list to which things can
% be added that reset the Haskell state to the beginning.
% This is to recover from states where the hacked intelligence
% is not sufficient.

\let\HaskellResetHook\empty
\newcommand*{\AtHaskellReset}[1]{%
  \g@addto@macro\HaskellResetHook{#1}}
\newcommand*{\HaskellReset}{\HaskellResetHook}

\global\let\hsforallread\empty

\newcommand\hsforall{\global\let\hsdot=\hsperiodonce}
\newcommand*\hsperiodonce[2]{#2\global\let\hsdot=\hscompose}
\newcommand*\hscompose[2]{#1}

\AtHaskellReset{\global\let\hsdot=\hscompose}

% In the beginning, we should reset Haskell once.
\HaskellReset

\makeatother
\EndFmtInput


% https://github.com/conal/talk-2015-essence-and-origins-of-frp/blob/master/mine.fmt
% Complexity notation:






% If an argument to a formatting directive starts with let, lhs2TeX likes to
% helpfully prepend a space to the let, even though that's seldom desirable.
% Write lett to prevent that.













































% Hook into forall.fmt:
% Add proper spacing after forall-generated dots.











% We shouldn't use /=, that means not equal (even if it can be overriden)!







% XXX



%  format `stoup` = "\blackdiamond"






% Cancel the effect of \; (that is \thickspace)



% Use as in |vapply vf va (downto n) v|.
% (downto n) is parsed as an application argument, so we must undo the produced
% spacing.

% indexed big-step eval
% without environments
% big-step eval
% change big-step eval








% \, is 3mu, \! is -3mu, so this is almost \!\!.


\def\deriveDefCore{%
\begin{align*}
  \ensuremath{\Derive{\lambda (\Varid{x}\typcolon\sigma)\to \Varid{t}}} &= \ensuremath{\lambda (\Varid{x}\typcolon\sigma)\;(\Varid{dx}\typcolon\Delta \sigma)\to \Derive{\Varid{t}}} \\
  \ensuremath{\Derive{\Varid{s}\;\Varid{t}}} &= \ensuremath{\Derive{\Varid{s}}\;\Varid{t}\;\Derive{\Varid{t}}} \\
  \ensuremath{\Derive{\Varid{x}}} &= \ensuremath{\Varid{dx}} \\
  \ensuremath{\Derive{\Varid{c}}} &= \ensuremath{\DeriveConst{\Varid{c}}}
\end{align*}
}


% Drop unsightly numbers from function names. The ones at the end could be
% formatted as subscripts, but not the ones in the middle.


\begin{figure}[h]
  \small
\begin{subfigure}[c]{0.5\textwidth}
\begin{hscode}\SaveRestoreHook
\column{B}{@{}>{\hspre}l<{\hspost}@{}}%
\column{3}{@{}>{\hspre}l<{\hspost}@{}}%
\column{9}{@{}>{\hspre}c<{\hspost}@{}}%
\column{9E}{@{}l@{}}%
\column{14}{@{}>{\hspre}l<{\hspost}@{}}%
\column{E}{@{}>{\hspre}l<{\hspost}@{}}%
\>[3]{}\Varid{p}{}\<[9]%
\>[9]{}\mathbin{::=}{}\<[9E]%
\>[14]{}\mathbf{succ}\mid \mathbf{add}\mid \ldots{}\<[E]%
\\
\>[3]{}\Varid{m},\Varid{n}{}\<[9]%
\>[9]{}\mathbin{::=}{}\<[9E]%
\>[14]{}\ldots\mbox{\onelinecomment  numbers}{}\<[E]%
\\
\>[3]{}\Varid{c}{}\<[9]%
\>[9]{}\mathbin{::=}{}\<[9E]%
\>[14]{}\Varid{n}\mid \ldots{}\<[E]%
\\
\>[3]{}\Varid{w}{}\<[9]%
\>[9]{}\mathbin{::=}{}\<[9E]%
\>[14]{}\Varid{x}\mid \lambda \Varid{x}\to \Varid{t}\mid \langle\Varid{w},\Varid{w}\rangle\mid \Varid{c}{}\<[E]%
\\
\>[3]{}\Varid{s},\Varid{t}{}\<[9]%
\>[9]{}\mathbin{::=}{}\<[9E]%
\>[14]{}\Varid{w}\mid \Varid{w}\;\Varid{w}\mid \Varid{p}\;\Varid{w}\mid \mathbf{let}\;\Varid{x}\mathrel{=}\Varid{t}\;\mathbf{in}\;\Varid{t}{}\<[E]%
\\
\>[3]{}\Varid{v}{}\<[9]%
\>[9]{}\mathbin{::=}{}\<[9E]%
\>[14]{}\rho\;[\mskip1.5mu \lambda \Varid{x}\to \Varid{t}\mskip1.5mu]\mid \langle\Varid{v},\Varid{v}\rangle\mid \Varid{c}{}\<[E]%
\\
\>[3]{}\rho{}\<[9]%
\>[9]{}\mathbin{::=}{}\<[9E]%
\>[14]{}\Varid{x}_{1}\mathbin{:=}\Varid{v}_{1},\ldots,\Varid{x}_n\mathbin{:=}v_n{}\<[E]%
\ColumnHook
\end{hscode}\resethooks
\caption{Base terms (\ilcUntau).}
\label{sfig:anf-syntax}
\end{subfigure}
%
\hfill
%
\begin{subfigure}[c]{0.49\textwidth}
\begin{hscode}\SaveRestoreHook
\column{B}{@{}>{\hspre}l<{\hspost}@{}}%
\column{3}{@{}>{\hspre}l<{\hspost}@{}}%
\column{11}{@{}>{\hspre}c<{\hspost}@{}}%
\column{11E}{@{}l@{}}%
\column{16}{@{}>{\hspre}l<{\hspost}@{}}%
\column{E}{@{}>{\hspre}l<{\hspost}@{}}%
\>[3]{}\Varid{dc}{}\<[11]%
\>[11]{}\mathbin{::=}{}\<[11E]%
\>[16]{}\mathrm{0}{}\<[E]%
\\
\>[3]{}\Varid{dw}{}\<[11]%
\>[11]{}\mathbin{::=}{}\<[11E]%
\>[16]{}\Varid{dx}\mid \lambda \Varid{x}\;\Varid{dx}\to \Varid{dt}\mid \langle\Varid{dw},\Varid{dw}\rangle\mid \Varid{dc}{}\<[E]%
\\
\>[3]{}\Varid{ds},\Varid{dt}{}\<[11]%
\>[11]{}\mathbin{::=}{}\<[11E]%
\>[16]{}\Varid{dw}\mid \Varid{dw}\;\Varid{w}\;\Varid{dw}\mid \NilC{\Varid{p}}\;\Varid{w}\;\Varid{dw}\mid {}\<[E]%
\\
\>[16]{}\mathbf{let}\;\Varid{x}\mathrel{=}\Varid{t};\Varid{dx}\mathrel{=}\Varid{dt}\;\mathbf{in}\;\Varid{dt}{}\<[E]%
\\
\>[3]{}\Varid{dv}{}\<[11]%
\>[11]{}\mathbin{::=}{}\<[11E]%
\>[16]{}\rho\mathrel{\filleddiamond}\D\rho\;[\mskip1.5mu \lambda \Varid{x}\;\Varid{dx}\to \Varid{dt}\mskip1.5mu]\mid \langle\Varid{dv},\Varid{dv}\rangle\mid \Varid{dc}{}\<[E]%
\\
\>[3]{}\D\rho{}\<[11]%
\>[11]{}\mathbin{::=}{}\<[11E]%
\>[16]{}\Varid{dx}_{1}\mathbin{:=}\Varid{dv}_{1},\ldots,\Varid{dx}_n\mathbin{:=}\Varid{dv}_n{}\<[E]%
\ColumnHook
\end{hscode}\resethooks
\caption{Change terms (\dilcUntau).}
\label{sfig:anf-change-syntax}
\end{subfigure}

\begin{subfigure}[t]{0.5\textwidth}
\begin{align*}
  \ensuremath{\DeriveConst{\Varid{n}}} &= \ensuremath{\mathrm{0}}\\
  \\
  \ensuremath{\Derive{\Varid{x}}} &= \ensuremath{\Varid{dx}} \\
  \ensuremath{\Derive{\lambda \Varid{x}\to \Varid{t}}} &= \ensuremath{\lambda \Varid{x}\;\Varid{dx}\to \Derive{\Varid{t}}} \\
  \ensuremath{\Derive{\langle\Varid{w}_a,\Varid{w}_b\rangle}} &= \ensuremath{\langle\Derive{\Varid{w}_a},\Derive{\Varid{w}_b}\rangle}\\
  \ensuremath{\Derive{\Varid{c}}} &= \ensuremath{\DeriveConst{\Varid{c}}}\\
  \\
  \ensuremath{\Derive{\Varid{w}_f\;\Varid{w}_a}} &= \ensuremath{\Derive{\Varid{w}_f}\;\Varid{w}_a\;\Derive{\Varid{w}_a}} \\
  \ensuremath{\Derive{\Varid{p}\;\Varid{w}}} &= \ensuremath{\NilC{\Varid{p}}\;\Varid{w}\;\Derive{\Varid{w}}} \\
  \ensuremath{\Derive{\mathbf{let}\;\Varid{x}\mathrel{=}\Varid{s}\;\mathbf{in}\;\Varid{t}}} &= \ensuremath{\mathbf{let}\;\Varid{x}\mathrel{=}\Varid{s};\Varid{dx}\mathrel{=}\Derive{\Varid{s}}\;\mathbf{in}\;\Derive{\Varid{t}}}
\end{align*}
\caption{Differentiation.}
%\caption{Differentiation, from base terms to change terms}
\label{sfig:anf-derive}
\end{subfigure}
%
\hfill
%
\begin{subfigure}[t]{0.49\textwidth}
\begin{hscode}\SaveRestoreHook
\column{B}{@{}>{\hspre}l<{\hspost}@{}}%
\column{3}{@{}>{\hspre}l<{\hspost}@{}}%
\column{27}{@{}>{\hspre}l<{\hspost}@{}}%
\column{E}{@{}>{\hspre}l<{\hspost}@{}}%
\>[3]{}\tau\mathbin{::=}\mathbb{N}\mid \tau_a\times\tau_b\mid \sigma\to \tau{}\<[E]%
\\[\blanklineskip]%
\>[3]{}\Delta \mathbb{N}{}\<[27]%
\>[27]{}\mathrel{=}\mathbb{N}{}\<[E]%
\\
\>[3]{}\Delta (\tau_a\times\tau_b){}\<[27]%
\>[27]{}\mathrel{=}\Delta \tau_a\times\Delta \tau_b{}\<[E]%
\\
\>[3]{}\Delta (\sigma\to \tau){}\<[27]%
\>[27]{}\mathrel{=}\sigma\to \Delta \sigma\to \Delta \tau{}\<[E]%
\\[\blanklineskip]%
\>[3]{}\Delta \EmptyContext{}\<[27]%
\>[27]{}\mathrel{=}\EmptyContext{}\<[E]%
\\
\>[3]{}\Delta (\Gamma,\Varid{x}\typcolon\tau){}\<[27]%
\>[27]{}\mathrel{=}\Delta \Gamma,\Varid{dx}\typcolon\Delta \tau{}\<[E]%
\ColumnHook
\end{hscode}\resethooks
\caption{Types and contexts.}
\label{sfig:anf-types}
\end{subfigure}

\caption{ANF $\lambda$-calculus: \ilcUntau{} and \dilcUntau{}.}
\label{fig:anf-lambda-calculus}
\end{figure}

\begin{figure}
  \footnotesize
\begin{subfigure}[t]{\textwidth}
\begin{typing}
\Axiom[T-Lit]{\ensuremath{\vdash_{\CONST}\Varid{n}\typcolon\mathbb{N}}}

\Axiom[T-Succ]{\ensuremath{\vdash_{\mathcal{P}}\mathbf{succ}\typcolon\mathbb{N}\to \mathbb{N}}}

\Axiom[T-Add]{\ensuremath{\vdash_{\mathcal{P}}\mathbf{add}\typcolon(\mathbb{N}\times\mathbb{N})\to \mathbb{N}}}
\end{typing}
\begin{typing}
\noindent
\Rule[T-Var]
  {\ensuremath{\Varid{x}\typcolon\tau\in \Gamma}}
  {\ensuremath{\Gamma\vdash\Varid{x}\typcolon\tau}}

\Rule[T-Const]
 {\ensuremath{\vdash_{\CONST}\Varid{c}\typcolon\tau}}
 {\ensuremath{\Gamma\vdash\Varid{c}\typcolon\tau}}

\raisebox{0.5\baselineskip}{\fbox{\ensuremath{\Gamma\vdash\Varid{t}\typcolon\tau}}}

\Rule[T-App]
  {\ensuremath{\Gamma\vdash\Varid{w}_f\typcolon\sigma\to \tau}\\
  \ensuremath{\Gamma\vdash\Varid{w}_a\typcolon\sigma}}
  {\ensuremath{\Gamma\vdash\Varid{w}_f\;\Varid{w}_a\typcolon\tau}}

\Rule[T-Lam]
  {\ensuremath{\Gamma,\Varid{x}\typcolon\sigma\vdash\Varid{t}\typcolon\tau}}
  {\ensuremath{\Gamma\vdash\lambda \Varid{x}\to \Varid{t}\typcolon\sigma\to \tau}}

\Rule[T-Let]
  {\ensuremath{\Gamma\vdash\Varid{s}\typcolon\sigma}\\
  \ensuremath{\Gamma,\Varid{x}\typcolon\sigma\vdash\Varid{t}\typcolon\tau}}
  {\ensuremath{\Gamma\vdash\mathbf{let}\;\Varid{x}\mathrel{=}\Varid{s}\;\mathbf{in}\;\Varid{t}\typcolon\tau}}

\Rule[T-Pair]
  {\ensuremath{\Gamma\vdash\Varid{w}_a\typcolon\tau_a}\\
  \ensuremath{\Gamma\vdash\Varid{w}_b\typcolon\tau_b}}
  {\ensuremath{\Gamma\vdash\langle\Varid{w}_a,\Varid{w}_b\rangle\typcolon\tau_a\times\tau_b}}

\Rule[T-Prim]
  {\ensuremath{\vdash_{\mathcal{P}}\Varid{p}\typcolon\sigma\to \tau}\\
   \ensuremath{\Gamma\vdash\Varid{w}\typcolon\sigma}}
 {\ensuremath{\Gamma\vdash\Varid{p}\;\Varid{w}\typcolon\tau}}
\end{typing}
\caption{\ilcTau{} base typing.}
\label{sfig:anf-base-typing}
\end{subfigure}

\begin{subfigure}[t]{\textwidth}
\begin{typing}
\Rule[T-DVar]
  {\ensuremath{\Varid{x}\typcolon\tau\in \Gamma}}
  {\ensuremath{\Gamma\vdash_{\Delta}\Varid{dx}\typcolon\tau}}

\Rule[T-DConst]
 {\ensuremath{\vdash_{\CONST}\Varid{c}\typcolon\Delta \tau}}
 {\ensuremath{\Gamma\vdash_{\Delta}\Varid{c}\typcolon\tau}}

\raisebox{0.5\baselineskip}{\fbox{\ensuremath{\Gamma\vdash_{\Delta}\Varid{dt}\typcolon\tau}}}

\Rule[T-DApp]
  {\ensuremath{\Gamma\vdash_{\Delta}\Varid{dw}_f\typcolon\sigma\to \tau}\\
    \ensuremath{\Gamma\vdash\Varid{w}_a\typcolon\sigma}\\
    \ensuremath{\Gamma\vdash_{\Delta}\Varid{dw}_a\typcolon\sigma}}
  {\ensuremath{\Gamma\vdash_{\Delta}\Varid{dw}_f\;\Varid{w}_a\;\Varid{dw}_a\typcolon\tau}}

\Rule[T-DLet]{
  \ensuremath{\Gamma\vdash\Varid{s}\typcolon\sigma}\\
  \ensuremath{\Gamma\vdash_{\Delta}\Varid{ds}\typcolon\sigma}\\
  \ensuremath{\Gamma,\Varid{x}\typcolon\sigma\vdash_{\Delta}\Varid{dt}\typcolon\tau}}
  {\ensuremath{\Gamma\vdash_{\Delta}\mathbf{let}\;\Varid{x}\mathrel{=}\Varid{s};\Varid{dx}\mathrel{=}\Varid{ds}\;\mathbf{in}\;\Varid{dt}\typcolon\tau}}

\Rule[T-DLam]
  {\ensuremath{\Gamma,\Varid{x}\typcolon\sigma\vdash_{\Delta}\Varid{dt}\typcolon\tau}}
  {\ensuremath{\Gamma\vdash_{\Delta}\lambda \Varid{x}\;\Varid{dx}\to \Varid{dt}\typcolon\sigma\to \tau}}

\Rule[T-DPair]
  {\ensuremath{\Gamma\vdash_{\Delta}\Varid{dw}_a\typcolon\tau_a}\\
  \ensuremath{\Gamma\vdash_{\Delta}\Varid{dw}_b\typcolon\tau_b}}
  {\ensuremath{\Gamma\vdash_{\Delta}\langle\Varid{dw}_a,\Varid{dw}_b\rangle\typcolon\tau_a\times\tau_b}}

\Rule[T-DPrim]
  {\ensuremath{\vdash_{\mathcal{P}}\Varid{p}\typcolon\sigma\to \tau}\\
    \ensuremath{\Gamma\vdash\Varid{w}\typcolon\sigma}\\
    \ensuremath{\Gamma\vdash_{\Delta}\Varid{dw}\typcolon\sigma}}
 {\ensuremath{\Gamma\vdash_{\Delta}\NilC{\Varid{p}}\;\Varid{w}\;\Varid{dw}\typcolon\tau}}
\end{typing}
\caption{\dilcTau{} change typing. Judgement \ensuremath{\Gamma\vdash_{\Delta}\Varid{dt}\typcolon\tau} means that variables from
  both \ensuremath{\Gamma} and \ensuremath{\Delta \Gamma} are in scope in \ensuremath{\Varid{dt}}, and the final type is in fact
  \ensuremath{\Delta \tau}.}
\label{sfig:anf-change-typing}
\end{subfigure}

\caption{ANF $\lambda$-calculus, \ilcTau{} and \dilcTau{} type system.}
\label{fig:anf-lambda-calculus-typing}
\end{figure}


\begin{figure}
  \footnotesize
\begin{subfigure}[t]{\textwidth}
\begin{typing}
  \Axiom[E-Prim]{\ensuremath{\rho\vdash\Varid{p}\;\Varid{w}\Downarrow_{\mathrm{1}}\mathcal{P}\mean{\Varid{p}}(\mathcal{V}\mean{\Varid{w}}\rho)}}

  \Axiom[E-Val]{\ensuremath{\rho\vdash\Varid{w}\Downarrow_{\mathrm{0}}\mathcal{V}\mean{\Varid{w}}\rho}}

  \raisebox{0.5\baselineskip}{\fbox{\ensuremath{\rho\vdash\Varid{t}\Downarrow_{\Varid{n}}\Varid{v}}}}

  \Rule[E-Let]{\ensuremath{\rho\vdash\Varid{s}\Downarrow_{\Varid{n}_s}\Varid{v}_s}\\\ensuremath{(\rho,\Varid{x}\mathbin{:=}\Varid{v}_s)\vdash\Varid{t}\Downarrow_{\Varid{n}_t}\Varid{v}_t}}{\ensuremath{\rho\vdash\mathbf{let}\;\Varid{x}\mathrel{=}\Varid{s}\;\mathbf{in}\;\Varid{t}\Downarrow_{\mathrm{1}\mathbin{+}\Varid{n}_s\mathbin{+}\Varid{n}_t}\Varid{v}_t}}

  \Rule[E-App]{
    \ensuremath{\rho\vdash\Varid{w}_f\Downarrow_{\mathrm{0}}\Varid{v}_f}\\
    \ensuremath{\rho\vdash\Varid{w}_a\Downarrow_{\mathrm{0}}\Varid{v}_a}\\
    \ensuremath{\mathsf{app}\;\Varid{v}_f\;\Varid{v}_a\;\negthickspace\Downarrow_{\Varid{n}}\negthickspace\;\Varid{v}}
  }{\ensuremath{\rho\vdash\Varid{w}_f\;\Varid{w}_a\Downarrow_{\mathrm{1}\mathbin{+}\Varid{n}}\Varid{v}}}
\end{typing}
\begin{hscode}\SaveRestoreHook
\column{B}{@{}>{\hspre}l<{\hspost}@{}}%
\column{3}{@{}>{\hspre}l<{\hspost}@{}}%
\column{30}{@{}>{\hspre}l<{\hspost}@{}}%
\column{E}{@{}>{\hspre}l<{\hspost}@{}}%
\>[3]{}\mathsf{app}\;(\rho\myquote\;[\mskip1.5mu \lambda \Varid{x}\to \Varid{t}\mskip1.5mu])\;\Varid{v}_a{}\<[30]%
\>[30]{}\mathrel{=}(\rho\myquote,\Varid{x}\mathbin{:=}\Varid{v}_a)\vdash\Varid{t}{}\<[E]%
\\[\blanklineskip]%
\>[3]{}\mathcal{V}\mean{\Varid{x}}\rho{}\<[30]%
\>[30]{}\mathrel{=}\rho\;(\Varid{x}){}\<[E]%
\\
\>[3]{}\mathcal{V}\mean{\lambda \Varid{x}\to \Varid{t}}\rho{}\<[30]%
\>[30]{}\mathrel{=}\rho\;[\mskip1.5mu \lambda \Varid{x}\to \Varid{t}\mskip1.5mu]{}\<[E]%
\\
\>[3]{}\mathcal{V}\mean{\langle\Varid{w}_a,\Varid{w}_b\rangle}\rho{}\<[30]%
\>[30]{}\mathrel{=}\langle\mathcal{V}\mean{\Varid{w}_a}\rho,\mathcal{V}\mean{\Varid{w}_b}\rho\rangle{}\<[E]%
\\
\>[3]{}\mathcal{V}\mean{\Varid{c}}\rho{}\<[30]%
\>[30]{}\mathrel{=}\Varid{c}{}\<[E]%
\\[\blanklineskip]%
\>[3]{}\mathcal{P}\mean{\mathbf{succ}}\Varid{n}{}\<[30]%
\>[30]{}\mathrel{=}\mathrm{1}\mathbin{+}\Varid{n}{}\<[E]%
\\
\>[3]{}\mathcal{P}\mean{\mathbf{add}}(\Varid{m},\Varid{n}){}\<[30]%
\>[30]{}\mathrel{=}\Varid{m}\mathbin{+}\Varid{n}{}\<[E]%
\ColumnHook
\end{hscode}\resethooks
\caption{Base semantics.
  Judgement \ensuremath{\rho\vdash\Varid{t}\;\negthickspace\Downarrow_{\Varid{n}}\negthickspace\;\Varid{v}} says that \ensuremath{\rho\vdash\Varid{t}}, a pair of
  environment \ensuremath{\rho} and term \ensuremath{\Varid{t}}, evaluates to value \ensuremath{\Varid{v}} in \ensuremath{\Varid{n}} steps.
  Notation \ensuremath{\mathsf{app}\;\Varid{v}_f\;\Varid{v}_a\;\negthickspace\Downarrow_{\Varid{n}}\negthickspace\;\Varid{v}} (used in rule \textsc{E-App}) is short
  for \ensuremath{\mathsf{app}\;\Varid{v}_f\;\Varid{v}_a\mathrel{=}\rho\vdash\Varid{t}} and \ensuremath{\rho\vdash\Varid{t}\;\negthickspace\Downarrow_{\Varid{n}}\negthickspace\;\Varid{v}}, that is,
  says that the pair \ensuremath{\rho\vdash\Varid{t}} given by \ensuremath{\mathsf{app}\;\Varid{v}_f\;\Varid{v}_a} evaluates to \ensuremath{\Varid{v}} in \ensuremath{\Varid{n}} steps.}
\label{sfig:anf-base-semantics}
\end{subfigure}

\begin{subfigure}[t]{\textwidth}
\begin{typing}
  \Axiom[E-DVal]{\ensuremath{\rho\mathrel{\filleddiamond}\D\rho\vdash_{\Delta}\Varid{dw}\Downarrow\mathcal{V}_{\Delta}\mean{\Varid{dw}}\rho\;\D\rho}}

\raisebox{0.5\baselineskip}{\fbox{\ensuremath{\rho\mathrel{\filleddiamond}\D\rho\vdash_{\Delta}\Varid{dt}\Downarrow\Varid{dv}}}}

  \Axiom[E-DPrim]{\ensuremath{\rho\mathrel{\filleddiamond}\D\rho\vdash_{\Delta}\NilC{\Varid{p}}\;\Varid{w}\;\Varid{dw}\Downarrow\mathcal{P}_{\Delta}\mean{\Varid{p}}(\mathcal{V}\mean{\Varid{w}}\rho)\;(\mathcal{V}_{\Delta}\mean{\Varid{dw}}\rho\;\D\rho)}}

  \Rule[E-DApp]{%
    \ensuremath{\rho\mathrel{\filleddiamond}\D\rho\vdash_{\Delta}\Varid{dw}_f\Downarrow\Varid{dv}_f}\\
    \ensuremath{\rho\vdash\Varid{w}_a\Downarrow\Varid{v}_a}\\
    \ensuremath{\rho\mathrel{\filleddiamond}\D\rho\vdash_{\Delta}\Varid{dw}_a\Downarrow\Varid{dv}_a}\\
    \ensuremath{\mathsf{dapp}\;\Varid{dv}_f\;\Varid{v}_a\;\Varid{dv}_a\;\negthickspace\Downarrow\negthickspace\;\Varid{dv}}}
  {\ensuremath{\rho\mathrel{\filleddiamond}\D\rho\vdash_{\Delta}\Varid{dw}_f\;\Varid{w}_a\;\Varid{dw}_a\Downarrow\Varid{dv}}}

  \Rule[E-DLet]{
    \ensuremath{\rho\vdash\Varid{s}\Downarrow\Varid{v}_s}\\
    \ensuremath{\rho\mathrel{\filleddiamond}\D\rho\vdash_{\Delta}\Varid{ds}\Downarrow\Varid{dvs}}\\
    \ensuremath{(\rho,\Varid{x}\mathbin{:=}\Varid{v}_s)\mathrel{\filleddiamond}(\D\rho;\Varid{dx}\mathbin{:=}\Varid{dvs})\vdash_{\Delta}\Varid{dt}\Downarrow\Varid{dvt}}}
  {\ensuremath{\rho\mathrel{\filleddiamond}\D\rho\vdash_{\Delta}\mathbf{let}\;\Varid{x}\mathrel{=}\Varid{s};\Varid{dx}\mathrel{=}\Varid{ds}\;\mathbf{in}\;\Varid{dt}\Downarrow\Varid{dvt}}}
\end{typing}
\begin{hscode}\SaveRestoreHook
\column{B}{@{}>{\hspre}l<{\hspost}@{}}%
\column{3}{@{}>{\hspre}l<{\hspost}@{}}%
\column{54}{@{}>{\hspre}l<{\hspost}@{}}%
\column{E}{@{}>{\hspre}l<{\hspost}@{}}%
\>[3]{}\mathsf{dapp}\;(\rho\myquote\mathrel{\filleddiamond}\D\rho\myquote\;[\mskip1.5mu \lambda \Varid{x}\;\Varid{dx}\to \Varid{dt}\mskip1.5mu])\;\Varid{v}_a\;\Varid{dv}_a{}\<[54]%
\>[54]{}\mathrel{=}(\rho\myquote,\Varid{x}\mathbin{:=}\Varid{v}_a)\mathrel{\filleddiamond}(\D\rho\myquote,\Varid{dx}\mathbin{:=}\Varid{dv}_a)\vdash_{\Delta}\Varid{dt}{}\<[E]%
\\[\blanklineskip]%
\>[3]{}\mathcal{V}_{\Delta}\mean{\Varid{dx}}\rho\;\D\rho{}\<[54]%
\>[54]{}\mathrel{=}\D\rho\;(\Varid{dx}){}\<[E]%
\\
\>[3]{}\mathcal{V}_{\Delta}\mean{\lambda \Varid{x}\;\Varid{dx}\to \Varid{dt}}\rho\;\D\rho{}\<[54]%
\>[54]{}\mathrel{=}\rho\mathrel{\filleddiamond}\D\rho\;[\mskip1.5mu \lambda \Varid{x}\;\Varid{dx}\to \Varid{dt}\mskip1.5mu]{}\<[E]%
\\
\>[3]{}\mathcal{V}_{\Delta}\mean{\langle\Varid{dw}_a,\Varid{dw}_b\rangle}\rho\;\D\rho{}\<[54]%
\>[54]{}\mathrel{=}\langle\mathcal{V}_{\Delta}\mean{\Varid{dw}_a}\rho\;\D\rho,\mathcal{V}_{\Delta}\mean{\Varid{dw}_b}\rho\;\D\rho\rangle{}\<[E]%
\\
\>[3]{}\mathcal{V}_{\Delta}\mean{\Varid{c}}\rho\;\D\rho{}\<[54]%
\>[54]{}\mathrel{=}\Varid{c}{}\<[E]%
\\[\blanklineskip]%
\>[3]{}\mathcal{P}_{\Delta}\mean{\mathbf{succ}}\Varid{n}\;\Varid{dn}{}\<[54]%
\>[54]{}\mathrel{=}\Varid{dn}{}\<[E]%
\\
\>[3]{}\mathcal{P}_{\Delta}\mean{\mathbf{add}}\langle\Varid{m},\Varid{n}\rangle\;\langle\Varid{dm},\Varid{dn}\rangle{}\<[54]%
\>[54]{}\mathrel{=}\Varid{dm}\mathbin{+}\Varid{dn}{}\<[E]%
\ColumnHook
\end{hscode}\resethooks
\caption{Change semantics.
  Judgement \ensuremath{\rho\mathrel{\filleddiamond}\D\rho\vdash_{\Delta}\Varid{t}\;\negthickspace\Downarrow\negthickspace\;\Varid{dv}} says that \ensuremath{\rho\mathrel{\filleddiamond}\D\rho\vdash_{\Delta}\Varid{dt}}, a triple of environment
  \ensuremath{\rho}, change environment \ensuremath{\D\rho} and change term \ensuremath{\Varid{t}}, evaluates to change value \ensuremath{\Varid{dv}}.
  Notation \ensuremath{\mathsf{dapp}\;\Varid{dv}_f\;\Varid{v}_a\;\Varid{dv}_a\;\negthickspace\Downarrow\negthickspace\;\Varid{dv}} (used in rule \textsc{E-DApp}) is short
  for \ensuremath{\mathsf{dapp}\;\Varid{dv}_f\;\Varid{v}_a\;\Varid{dv}_a\mathrel{=}\rho\mathrel{\filleddiamond}\D\rho\vdash_{\Delta}\Varid{dt}} and
  \ensuremath{\rho\mathrel{\filleddiamond}\D\rho\vdash_{\Delta}\Varid{t}\;\negthickspace\Downarrow\negthickspace\;\Varid{dv}}, that is, says that the triple \ensuremath{\rho\mathrel{\filleddiamond}\D\rho\vdash_{\Delta}\Varid{dt}}
  given by \ensuremath{\mathsf{dapp}\;\Varid{dv}_f\;\Varid{v}_a\;\Varid{dv}_a} evalutes to \ensuremath{\Varid{dv}}.}
\label{sfig:anf-change-semantics}
\end{subfigure}

\caption{ANF $\lambda$-calculus (\ilcUntau{} and \dilcUntau{}), CBV semantics.}
\label{fig:anf-lambda-calculus-semantics}
\end{figure}
