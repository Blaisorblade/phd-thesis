\section{Addendum: change equality}
\label{sec:change-eq}

While preparing the final version of the paper, we introduced a
small technical mistake: in \cref{thm:deriv-nil}, we assert that
two changes are equal, even though in
\cref{ssec:change-structures} we promised that ``Throughout
our theory, we only discuss equality of base values, not of
changes.''

However, by that time, we had formalized in Agda the correct
equivalence relation to use on change, that we call
delta-observational equivalence (d.o.e.). Intuitively, two
changes are d.o.e. if using them as changes gives the same
observations, as long as we only observe base values and not
changes themselves; below we show the theorems that we have
proved in Agda and that substantiate this intuition, at least to
a great extent.

In this section, we
state the corrected version of \cref{thm:deriv-nil}, and explain
what we know about this change equivalence.

\begin{definition}[Delta-observational equivalence]
  Given a change structure $\ChangeStruct{V}$, a value $v \in V$,
  and two changes $\D v_1, \D v_2 \in \Change{v}$, if and only if
  $\Update{v}{\D v_1} = \Update{v}{\D v_2}$ we say that $\D v_1$
  is delta-observationally equivalent (d.o.e.) to $\D v_2$, and
  we write $\D v_1 \Doe \D v_2$.
\end{definition}

Using this relation, we can state a correct version of
\cref{thm:deriv-nil}:

\begin{lemma}[Behavior of derivatives on $\NIL$]
  \label{thm:deriv-nil-2}
  Given change structures $\ChangeStruct{A}$ and
  $\ChangeStruct{B}$, a function $f \in A \to B$, an element $a$
  of $A$, and the derivative $f'$ of $f$, we have
  $\App{\App{f'}{a}}{\Nil{a}} \Doe \Nil{\App* f a}$.
\end{lemma}

Moreover, we have a number of lemmas about delta-observational
equivalence.

\pg{Revise statements for clarity.}

\begin{lemma}[D.o.e. is an equivalence]
  For any $x \in X$ with a change structure on $X$, d.o.e. is an
  equivalence relation (reflexive, symmetric, transitive) among
  elements of $\Change{x}$.
\end{lemma}

\begin{lemma}[Identities using d.o.e.]
  Using d.o.e. we can state additional algebraic equivalences,
  that complement \cref{def:update-diff}.

\begin{align*}
\Update x \D{x} = x &\Leftrightarrow \D{x} \Doe \Nil{x}\\
\Diff x x &\Doe \Nil {x}\\
\Diff {\Update*{x}{\D{x}}} x &\Doe \D{x}
\end{align*}
\end{lemma}

\begin{lemma}[Function change application preserves d.o.e.]
  For each $x \in X$, with a change structure on $X$, changes
  $\D{x}_1, \D{x}_2 \in \Change{x}$, for each $f \in X \to Y$,
with a change structure on $Y$, and changes
$\D{f}_1, \D{f}_2 \in \Change{f}$, then $\D{f}_1 \Doe \D{f}_2$ and
$\D{x}_1 \Doe \D{x}_2$ imply
$\D{f}_1~x~\D{x}_1 \Doe \D{f}_2~x~\D{x}_2$.
\end{lemma}

That is, in all simple contexts that we can construct, two d.o.e.
changes will behave indistinguishably. In fact, for programs that
only use changes as changes (without looking at their
implementation details), we conjecture that d.o.e. changes are
observationally equivalent. However, making this conjecture
precise and proving it are efforts left for future work.

\begin{lemma}[Extensionality for d.o.e.]
  If two function changes $\D{f}_1, \D{f}_2$ behave equally when
  applied to arbitrary type-correct inputs $x$,
  $\D{x}_1, \D{x}_2 \in \Change{x}$, then $\D{f}_1 \Doe \D{f}_2$.
\end{lemma}

\begin{lemma}[Derivatives are unique up to d.o.e.]
  If two function changes $\D{f}_1, \D{f}_2$ are derivatives for
  $f$, then they're d.o.e. to each other and to $f$'s nil change:
  $\D{f}_1 \Doe \Nil{f} \Doe \D{f}_2$.
\end{lemma}
We only have such an uniqueness theorem up to d.o.e., not to
standard equality, since multiple d.o.e. changes can be
different.
% XXX continue: Transcribe lemmas from Base.Change.Equivalence
