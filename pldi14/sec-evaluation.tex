% Emacs, this is -*- latex -*-!
\section{Benchmarks}
% Disabled since we shouldn't refer to this appendix, it's just
% part of the extended version of the paper.
%\label{sec:benchmarks}

Benchmarking our case study shows
that \ILC\ can offer order-of-magnitude speedups for a
realistic higher-order program.

\begin{oldSec} % ID=Adaptation
We performed additional microbenchmarks on simpler examples
(summing the values of a map\pg{Cai: why not the elements of a
  bag?}, mapping the successor function on bag elements, bag
merge);\pg{the examples are really small, is it OK? I guess so.}
results on those examples did not yield additional insight, hence
we omit them from this report.\pg{Add the links for details?}

\paragraph{Adaptation}
We first adapted our example program to our object language,
as parameterized by the plugin we presented in
Sec.~\ref{sec:plugins}. Where primitives were insufficient, we
added new ones as appropriate~\pg{Ensure we don't contradict this
  claim elsewhere.}.

First, we adapted the generic MapReduce framework as a library of
polymorphic object-level functions. As usual, we use the
meta-language both to define the library and to encode
polymorphism.

This produces a more restrictive variant of MapReduce. \pg{Explain.}

Future work, supporting non-self-incrementalizable primitives through caching, would allow lifting these limitations.

The result is shown in \cref{fig:case-study-pseudocode}.
\end{oldSec}
