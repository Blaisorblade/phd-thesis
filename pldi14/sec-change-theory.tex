% Emacs, this is -*- latex -*-!
%\section{Changes as First-Class Values}
\section{A theory of changes}
\label{sec:1st-order-changes}

This section introduces a formal concept of changes; this
concept was already used informally in \cref{eq:correctness} and is central
to our approach. We first define change structures formally, then construct 
change structures for functions between change structures,
and conclude with a theorem that relates function changes to derivatives. 

\subsection{Change structures}\label{ssec:change-structures}
Consider a set of values, for instance the set of natural numbers
$\mathbb{N}$. A change $\D v$ for $v \in \mathbb{N}$ should
describe the difference between $v$ and another natural $\New{v}
\in \mathbb{N}$. We do not define changes directly, but we
specify operations which must be defined on them. They are:
\begin{itemize}
\item We can \emph{update} a base value $v$ with a
  change $\D v$ to obtain an updated or \emph{new} value
  $\New{v}$. We write $\New{v} = \Apply{\D v}{v}$.
\item We can compute a change between two arbitrary
  values $\Old{v}$ and $\New{v}$ of the set we are considering.
  We write $\D v = \Diff{\New{v}}{\Old{v}}$.
\end{itemize}

For naturals, it is usual to describe changes using standard
subtraction and addition. That is, for naturals we can define
$\Apply{\D v}{v} = v + \D v$ and $\Diff{\New{v}}{\Old{v}} =
\New{v} - \Old{v}$. To ensure that $\APPLY$ and $\DIFF$ are
always defined, we need to define the set of changes carefully.
$\mathbb{N}$ is too small, because subtraction does not always
produce a natural; the set of integers $\mathbb{Z}$ is instead
too big, since adding a natural and an integer does not always
produce a natural. In fact, we cannot use the same set of all
changes for all naturals. Hence we must adjust the requirements:
for each base value $v$ we introduce a set $\Change{v}$ of
changes for $v$, and require $\Diff{\New{v}}{\Old{v}}$ to produce
values in $\Change{\Old{v}}$, and $\Apply{\D v}{v}$ to be defined
for $\D v$ in $\Change{v}$. For natural $v$, we set $\Change{v} =
\left\{\D v \mid v + \D v \geq 0 \right\}$; $\DIFF$ and $\APPLY$ are
then always defined.

\begin{oldSec}

\ldots, we could use \emph{functional
changes}, that is by defining changes to be functions from the
old value to the new value:
\begin{align*}
\Change{\Gt} & = \Gt \r \Gt, \\
\Apply{\D v}{\Old{v}} & = \App{\D v}{\Old{v}},\\
\Diff{\New{v}}{\Old{v}} & = \Lam{x}{\New{v}}.
\end{align*}
However,
this definition does not allow derivatives to analyze changes to
be more efficient than recomputation. To understand why, let us
consider the following example.

Let $\Old{v} = \{1, 2, \ldots, n\}$ be a bag (or multiset) of
integers, let $f$ be a function from bags to integers summing the
elements of its argument, and let $\Old{s} = \App{f}{\Old{v}}$.

Later during program execution, assume we add $n + 1$ to
$\Old{v}$ and need to update $\Old{s}$. Hence,
 $\New{v} = \{1, 2, \ldots, n, n + 1\}$, $\D v$ represent the change of $v$,
and we need to compute the result of $\New{s} = \App{f}{\New{v}}$.
%
Thanks to \cref{eq:correctness}, we can guarantee that
$\New{s} = \Apply{\App{\App{\Derive{f}}{\Old{v}}}{\D
    v}}{\Old{v}}$.

Now, if $\Derive{f}$ would know that $\D v$ only added $n + 1$ to
the bag, it could produce in $O(1)$ a change $\D s$ such that
$\Apply{\D s}{s} = n + 1 + s$. But if $\D v$ is simply a function
such that $\App{\D v}{\Old{v}} = \New{v}$, we have no way of
inspecting its intension, since in $\lambda$-calculus functions
are opaque. Instead, the difference between two bags can be
described as another bag, and $\APPLY$ for bags can be defined as
bag merge.%
\footnote{Negative multiplicities are required to represent
  removals, as we discuss in Sec.~\ref{sec:plugins}.} Similarly,
we can describe the difference between two numbers $x$ and $y$ as
their arithmetical difference $x-y$. In this case, the change
application operator $\APPLY$ would be the normal addition
operator $+$. With these definitions, thanks to the structure of
$+$, $\App{\App{\Derive{f}}{\Old{v}}}{\D v}$ can produce its
result without even using $\Old{v}$, in time $O(|\D v|)$ (we
explain later how to compute $\Derive{f}$ automatically).

For now, we simply note that we cannot fix $\Change{\Gt} = \Gt \r
\Gt$. We need a more flexible encoding of changes, which allows
inspecting their structure; moreover, this structure needs to
allow writing efficient derivatives, in particular efficient
derivatives for the primitives acting on $\Gt$.

Hence, to make our general framework
independent of such domain- and application-specific
considerations, we simply require language plugins to define not
only base types and primitives for them, but also $\Change{\tau}$
whenever $\tau$ is a base type, and operators $\APPLY_\tau$ and
$\DIFF_\tau$.
Using $\APPLY$, we can recover a function $\Gt\r\Gt$
from any $\D x$ of type $\Change{\Gt}$; it is $\Lam*x{\Apply{\D
x}{x}}$.
\end{oldSec}

\pg{We never say why we use ``structure''. On second thought,
  this might be OK since we have little space.}
The following definition sums up the discussion so far:

\pg{Consider less heavyweight phrasing, such as: ``To each $v \in V$
  we associate a set of changes $\Change{v}$. But do this consistently.}
\begin{definition}[Change structures]
  \label{def:change-struct}
  A tuple $\ChangeStruct{V} = (V, \CHANGE,
  \UPDATE,
  \DIFF)$ is a \emph{change structure} (for $V$) if:

  \begin{subdefinition}
  \item $V$ is a set, called the \emph{base set}.
  \item Given $v \in V$, $\Change{v}$ is a set, called the \emph{change set}.
  \item Given $v \in V$ and $\D v \in \Change{v}$, $\Apply{\D v}{v} \in V$.
    \label{def:update}
  \item Given $u, v \in V$, $\Diff{u}{v} \in \Change{v}$.
    \label{def:diff}
  \item Given $u, v \in V$, $\Apply{\Diff*{u}{v}}{v}$ equals $u$.
    \qed
    \label{def:update-diff}
  \end{subdefinition}
\end{definition}

One might expect a further assumption that
$\Diff{\Apply*{\D v}{v}}{v} = \D v$. While it does hold
for the change structure of $\mathbb{N}$, it is not needed in general.
This means that multiple changes can represent the difference between
the same two base values. Throughout our theory, we only discuss equality of
base values, not of changes.

\paragraph{Notation}
We overload operators $\CHANGE$, $\DIFF$ and $\UPDATE$ to refer
to the corresponding operations of different change structures;
we will subscript these symbols when needed to prevent ambiguity.
For any $\ChangeStruct{S}$, we write $S$ for its first component,
as above. We make $\UPDATE$ left-associative, that is,
$\Update{\Update{v}{dv_1}}{dv_2}$ means $\Update{\Update*{v}{dv_1}}{dv_2}$.
We assign precedence to function application over
$\UPDATE$ and $\DIFF$, that is, $\Update{\App{f}{a}}{\App{\App{g}{a}}{\D a}}$ means
$\Update{\App*{f}{a}}{\App*{\App{g}{a}}{\D a}}$.

\begin{examples}
We demonstrate a change structure on \emph{bags with signed
multiplicities}~\citep{Koch10IQE}.
These are
unordered collections where each element can appear an integer
number of times. 
\begin{enumerate}[(a)]
\item
Let $S$ be any set.
The base set $V=\Bag S$ is the set of bags of elements of $S$ with signed
multiplicities. The bag $\Set{1,1,\bar2}$ contains two positive
occurrences of $1$ and a negative occurrence of $2$.

\item For each bag $v\in V$, set the change set $\Change v = V$.
Every bag can be a change to any other bag. The bag
$\Set{1,1,\bar5}$ represents two insertions of $1$ and one
deletion of $5$.

\item The update operator is bag merge: $\UPDATE=\MERGE$. The
merge of two bags is the element-wise sum of multiplicities:
\[
\Merge{\Set{\bar1,2}}{\Set{1,1,\bar5}}=\Set{1,2,\bar5}.
\]

\item Let $\NEGATE$ be the negation of multiplicities:
\[
\Negate{\Set{1,1,\bar5}}=\Set{\bar1,\bar1,5}.
\]
To compute the
difference of two bags, compute the merge with a negated bag:
\[
\Diff{u}{v}=\Merge{u}{\Negate*{v}}.
\]

\item Given the above definition of $\UPDATE$ and $\DIFF$, it is
not hard to show that $\Apply{\Diff*{u}{v}}{v}$ for all bags
$u,v\in V$.
\end{enumerate}
The change structure we just described is written succinctly
\begin{alignat*}3
\ChangeStruct{\Bag S} = (
&\Bag S,
&&\Lam*{v} {\Bag S},
\\
&\MERGE,
&&\Lam*{x\; y}{\Merge{x}{\Negate*{y}}}).
\end{alignat*}

This change structure is an instance of a general construction:
we can build a change structure from an arbitrary \emph{abelian group}.
An abelian group is a tuple $(G, \boxplus,
\boxminus, e)$, where $\boxplus$ is a commutative
and associative binary operation, $e$ is its identity
element, and $\boxminus$ produces inverses of elements $g$
of $G$, such that $(\boxminus g) \boxplus g = g \boxplus
(\boxminus g) = e$. For instance, integers,
unlike naturals, form the abelian group $(\mathbb{Z}, +, -, 0)$
(where $-$ represents the unary minus). Each abelian group
$(G, \boxplus, \boxminus, e)$ induces a change structure,
namely $\left(G, \Lam{g}{G}, \boxplus, \Lam{g\; h}{g
    \boxplus (\boxminus h)}\right)$, where the change set
for any $g \in G$ is the whole $G$. Change structures
are more general, though, as the example with natural numbers illustrates.
%
If $\Empty$ represents the empty bag, then $(\Bag{S}, \MERGE,
\NEGATE, \Empty)$ is an abelian group, which induces the
change structure we have just seen.

The abelian group on integers induces also a change structure on
integers, namely $\ChangeStruct{\mathbb{Z}} = (\mathbb{Z},
\Lam*{v} {\mathbb{Z}}, +, -)$.
\end{examples}

\paragraph{Nil changes and derivatives}
A particularly important change is the \emph{nil change} of a value:
\begin{definition}[Nil change]
  \label{def:nil-change}
  Given a change structure $\ChangeStruct{V}$ and a value $v \in V$, the change
  $\Diff{v}{v}$ is the nil change for $v$.
  \[
    \Nil{v} = \Diff{v}{v} \qed
  \]
\end{definition}
The nil change for a value does indeed not change it.
\begin{lemma}[Behavior of $\NIL$]
  \label{thm:update-nil}
  Given a change structure $\ChangeStruct{V}$ and a value $v \in V$,
  $\Apply{\Nil{v}}{v} = v$.
\end{lemma}

\begin{optionalproof}
Follows from \cref{def:update-diff,def:nil-change}.
\end{optionalproof}

\pg{Maybe should move this before nil changes?}
After defining change structures, we can restate the definition of derivatives from \cref{eq:correctness}.

\begin{definition}[Derivatives]
  \label{def:derivatives}
  Given change structures $\ChangeStruct{A}$ and $\ChangeStruct{B}$ and a function $f \in A \to
  B$ on the change sets of these change structures, we call a binary function $f'$ the \emph{derivative} of $f$ if
  for all values $a \in A$ and corresponding changes $\D a \in
  \Change[A]{a}$,
  \[\App{f}{\Apply*{\D a}{a}} = \Apply{\App{\App{f'}{a}}{\D a}}{\App{f}{a}}\text{.}\qed\]
\end{definition}

Applying a derivative to a value and its nil change gives a nil
change.%
\footnote{Post-print note: There's a small technical mistake in
  the following lemma. See \cref{sec:change-eq} for a corrected
  statement.}
%
\begin{lemma}[Behavior of derivatives on $\NIL$]
  \label{thm:deriv-nil}
  Given change structures $\ChangeStruct{A}$ and
  $\ChangeStruct{B}$, a function $f \in A \to B$, an element $a$
  of $A$, and the derivative $f'$ of $f$, we have
  $\App{\App{f'}{a}}{\Nil{a}} = \Nil{\App* f a}$.
\end{lemma}

\begin{examples}
Let $\Term{f}:\Fun{\Bag S}{\Bag S}$ be the constant function mapping
everything to the empty bag. Its derivative
$\Term{f'}:\Fun{\Bag S}{\Fun{\Bag S}{\Bag S}}$ has to ignore its two
arguments and produce the empty bag in all cases.

Let $\Term{id}:\Fun{\Bag S}{\Bag S}$ be the identity function between
bags. Its derivative $\Term{id'}$ is defined by
$\Term{id'}~v~\D v = \D v$.
\end{examples}

\input{pldi14/sec-function-change}
