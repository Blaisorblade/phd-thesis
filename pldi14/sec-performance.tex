% Emacs, this is -*- latex -*-!

\subsection{Self-maintainability}
\label{sec:performance-cons}
\label{ssec:self-maint}

\begin{oldSec}
On its own, deriving $\Gl$-abstractions, applications and
variables (\cref{fig:correctness:derive}) does not improve performance.
It only relates the computations embodied in the primitives in
an incremental and higher-order setting, and provides a method to
generalize the vast quantity of research on first-order incremental
computation in the field of databases.
Our first approach toward performance gain is based on the idea of \emph{self-maintainable uses of primitives};
other approaches are certainly feasible. We did not formalize the
optimizations or prove them correct.
\end{oldSec}

In databases, a self-maintainable view~\citep{Gupta99MMV} is a function that can
update its result from input changes alone, without looking at
the actual input. By analogy, we call a derivative
\emph{self-maintainable} if it uses no
base parameters, only their changes. Self-maintainable derivatives
describe efficient incremental computations: since they do not
use their base input, their running time does not have to depend on the input
size.

\begin{examples}
$\Derive{\MERGE} = \Lam {x\; \D x\; y \; \D y}{\Merge{\D x}{\D y}}$
is self-maintainable with the
change structure $\ChangeStruct{\Bag{S}}$ described in
\cref{ex:valid-bag-int}, because it does not use the base
inputs $x$ and $y$.
Other derivatives are self-maintainable only in certain contexts.
The derivative of element-wise function application
$\App*{\App\MAP f}{\mathit{xs}}$ ignores the original
value of the bag $\mathit{xs}$ if the changes to
%
$f$ are always nil, because the underlying primitive $\FOLDBAG$
is self-maintainable in this case (as discussed in next section).
%
We take advantage of this by implementing a specialized
derivative for $\FOLDBAG$.

Similarly to what we have seen in \cref{sec:derive-example-merge} that $\Derivative$
needlessly recomputes $\Merge\Xs\Ys$ without optimizations. However, the result is a
base input to $\FOLD'$.
%
In next section, we'll replace $\FOLD'$ by a self-maintainable
derivative (based again on $\FOLDBAG$) and will avoid this
recomputation.
\end{examples}

%% Rationale for previous statement. Too complicated; left out.
%deriving a closed term without primitives yields a term that is
%self-maintainable whenever its higher-order arguments receive
%self-maintainable changes.
%The default rule $\Derive{c} = \Diff c c$ does not yield
%self-maintainable derivatives, because $\DIFF$ and $\APPLY$ use
%the input~$x$ in a significant way (\cref{fig:diff-apply}).
%However,

To conservatively predict whether a derivative is going to be
self-maintainable (and thus efficient), one can inspect whether
the program restricts itself to (conditionally) self-maintainable
primitives, like $\MERGE$ (always) or $\MAP \APP f$ (only if $\D
f$ is nil, which is guaranteed when $f$ is a closed term).

%% From the author response.
%It is possible to give a conservative approximation of whether a program's derivative is self-maintainable:
%If the original program only uses primitives with fully
%self-maintainable derivatives, the derivative of the program will
%be self-maintainable, too. We can syntactically approximate full
%self-maintainability of a primitive by checking whether the code
%of its derivative uses the $x$ variable (in addition to $\D x$) \pg{bind x! What about multiple arguments?}. For
%instance, the derivative of $\MERGE$ uses only $\D x$ and $\D y$, never $x$
%or $y$, and is hence fully self-maintainable. The derivative of
%$\FOLDBAG$ uses both $f$ and $\D f$; it is only self-maintainable
%sometimes (when df is the nil change).

\begin{oldSec}
Sometimes we can safely replace the derivative of a primitive by
a self-maintainable term, in which case we call it a
\emph{self-maintainable use of a primitive.}
Terms have self-maintainable derivatives if they use
primitives in a self-maintainable manner.

\pg{As Klaus points out, this is not true until we fix the set of
  changes - discussed in \#265. So we should specify the set of
  changes, or use a different example. We should also motivate
  it.}
%
\pg{Agreed: Convert to bags and to our running example, using foldBag.}
To illustrate, suppose $\MAP$ is a primitive of sets of
integers,
and changes to sets consist of insertions and deletions only.
The term
\[
\Lam x {\MapT {\Lam*n {n + 1}} x}
\]
contains a self-maintainable use of $\MAP$, because any change to
the input set~$x$, say $\{\App\INSERT5, \App\DELETE7\}$, can be
converted to an output change, say $\{\App\INSERT6,
\App\DELETE8\}$, without looking at $x$ itself. The use of $\MAP$
in the following term is not self-maintainable:
\[
\Lam x {\MapT {\Lam*n {n + \Sum* x}} x}.
\]
Even one insertion to the input set~$x$ generates a sweeping
change over all elements of the output set that is impossible to
express in terms of insertions and deletions without knowledge of $x$.

\pg{Before this, show that some of our primitives are
  self-maintainable, and some are self-maintainable if some
  inputs are nil.}
Our prototype optimization framework proceeds in two steps.
\begin{enumerate}
\item A static analysis identifies when changes are guaranteed to be nil.
\item During the differentiation transformation, $\DERIVE$
selects an appropriate self-maintainable function whenever possible,
considering the results of the static analysis.
\end{enumerate}
Due to space constraint, we cannot go into details of those
steps.
%
\yc{link to code}
%
\end{oldSec}

To avoid recomputing base arguments for self-maintainable derivatives
(which never need them), we
currently employ lazy evaluation.  Since we could use standard techniques for dead-code
elimination~\citep{Appel97} instead, laziness is not central to our
approach.

A significant restriction is that not-self-maintainable derivatives can require expensive computations to supply their base
arguments, which can be expensive to compute. Since they are also
computed while running the base program, one could reuse the previously
computed value through memoization or extensions of static
caching (as discussed in \cref{ssec:staticmemo}). We leave implementing these optimizations for future work. As a consequence,
our current implementation delivers good results only if
most derivatives are self-maintainable.

\begin{oldSec} % ID=Appel97
\pg{Do not remove without universal agreement. I don't think the paper can avoid discussing this aspect. }
One important issue is left. In a call-by-value
implementation of lambda calculus, running the program
\[
\Derive{\App{s}{t}} = \App{\App{\Derive{s}}{t}}{\Derive{t}}
\]
computes $t$
again, even though it was computed in the base program, thus
leading to wasteful repeated computation.
However, we claim this is a simpler problem to solve.
Three possibilities arise:
\begin{enumerate}
\item $t$ is very cheap to compute (for instance, it is a
  literal), so the problem does not occur.
\item $t$ is passed to a function which does not use it, hence
  we can avoid computing it using separate optimization steps, by
  executing the derivative with a lazy semantics (as done in our
  benchmarks) or (we expect) by using known techniques for
  interprocedural dead-code elimination~\citep{Appel97}.
\item Otherwise, since the term $t$
  was already computed while running the base
  program, we can save its value for use in the derivative.
  Approaches to
  implement this include memoization~\pg{cite} and extensions of
  static caching (as discussed in Sec.~\ref{sec:rw}).
  We leave investigating the different
  approaches to future work.
\end{enumerate}
\end{oldSec}

\begin{oldSec} % ID=Gupta99MMV
We next analyze when $t$ is going to be unused.
Inspecting the definition of $\DERIVE$ shows that only derivatives of (functions
containing) primitives can use $t$. For instance,
$\Derive{\Lam{x}{x}} = \Lam{x}{\Lam{\D x}{\D x}}$ does not use its
base input $x$, because $\Derive{x} = \D x$.
\pg{Make this more tentative. Remove 'prove'.}
It's easy to prove
by induction that if derivatives of primitives appearing in $s$
do not use their base input, neither does $s$.\footnote{Function
  changes coming from outside the program are not covered by this
  proof and are possible counterexamples.}

We term primitives whose derivative does not use their base input
self-maintainable, by analogy with the database concept of
self-maintainable views~\citep{Gupta99MMV}. \ko{Give example of 
self-maintainable and non-self-maintainable primitive}. 
We then extend this to programs: a
closed program is self-maintainable if its result does not depend
on base inputs or base intermediate results.
%
For programs which do not depend on functions as parameters (and
whose derivatives hence do not have function changes as
parameters), it should be straightforward to prove that being
self-maintainable is equivalent to only using self-maintainable
primitives.

Hence, we can argue informally that if a program only uses
self-maintainable primitives, its derivative will not recompute
base intermediate results (except if they are computed for
unrelated reasons). In our case study (Sec.~\ref{sec:eval}),
such derivatives are already
extremely efficient.

\pg{This should be somehow better integrated in the rest of the section.}
\end{oldSec}
