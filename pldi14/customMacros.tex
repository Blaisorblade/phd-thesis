% Macros specific to this paper project

% Macros for terms, types and values
%
% Naming Convention:
%   \UPPERCASE -- single keyword
%   \Mixed     -- term constructor
%
% All term constructors have a starred form to add parentheses.
%
% Example:
%
%   \App{\Lam*{x}{\Var{x}}}{\Lit{42}}  gives  (λx.x) 42

\usepackage{mathtools}
\newcommand\validfromto[4]{#3 \;\rhd\; #2 \xhookrightarrow{#1} #4}
\newcommand\correctdt[2]{#1 \;\rhd \; #2}
\DeclareMathAlphabet{\mathcal}{OT1}{pzc}{m}{it}

\newcommand{\Keyword}[1]{\textbf{#1}}
\newcommand{\Term}[1]{\Varid{#1}}

\newenvironment{equational}{
\def\commentbegin{\quad\{\ }
\def\commentend{\}}
}{}

% Lists

\NewDocumentCommand{\List}{sm}{\Parens#1{\overline{#2}}}

% Helper for optional subscripts and superscripts

\NewDocumentCommand{\Index}{sm}{\IfValueTF{#2}{\IfBooleanTF{#1}{^}{_}{#2}}{}}

% \DIFF          for   ⊖
% \Diff{u}{v}    for   u ⊖ v

% Deprecated:
% \APPLY         for   ⊕
% \Apply{dv}{v}  for   v ⊕ dv
%
% New:
% \UPDATE        for   ⊕
% \Update{v}{dv} for   v ⊕ dv.
\NewDocumentCommand{\DIFF}{o}{\ominus\Index{#1}}
\NewDocumentCommand{\UPDATE}{o}{\oplus\Index{#1}}
%Deprecated alias.
\let\APPLY\UPDATE

\NewDocumentCommand{\Upd}{sm}{\Parens#1{#2 \UPDATE \D #2}}

\NewDocumentCommand{\Diff}{somm}{\Parens#1{#3 \DIFF[#2] #4}}

\NewDocumentCommand{\Update}{somm}{\Parens#1{#3 \UPDATE[#2] #4}}
% Deprecated variant. Beware of the copy and paste!
\NewDocumentCommand{\Apply}{somm}{\Parens#1{#4 \UPDATE[#2] #3}}

\NewDocumentCommand{\WideDiff}{somm}{\Parens#1{#3 \;\;\DIFF[#2]\;\; #4}}
\NewDocumentCommand{\WideApply}{somm}{\Parens#1{#4 \;\;\APPLY[#2]\;\; #3}}

\NewDocumentCommand{\NILC}{}{\mathbf{0}}
\NewDocumentCommand{\NilC}{om}{\NILC_{#2}}

\NewDocumentCommand{\ChangeStruct}{m}{\widehat{#1}}
\NewDocumentCommand{\chs}{m}{\ChangeStruct{#1}}

%http://tex.stackexchange.com/a/163831/1340
\usepackage{stackengine}
\NewDocumentCommand{\Doe}{}{\mathrel{\stackon[1pt]{$=$}{$\scriptscriptstyle\Delta$}}}
%{\ensuremath{\overset{\scriptscriptstyle\GD}{=}}}
%{\triangleq}

\NewDocumentCommand{\MAP}{}{\Term{map}}

\NewDocumentCommand{\Xs}{}{\Term{xs}}
\NewDocumentCommand{\Ys}{}{\Term{ys}}
\NewDocumentCommand{\Zs}{}{\Term{zs}}
\NewDocumentCommand{\Program}{}{\Term{grand\_total}}
\NewDocumentCommand{\ProgramBody}{}{\Lam{\Xs}{\Lam{\Ys}{\FOLD~(+)~0~(\Merge\Xs\Ys)}}}
\NewDocumentCommand{\Derivative}{}{\Term{dgrand\_total}}
\NewDocumentCommand{\Output}{}{\Term{output}}

\NewDocumentCommand{\DXs}{}{\Term{dxs}}
\NewDocumentCommand{\DYs}{}{\Term{dys}}
\NewDocumentCommand{\DZs}{}{\Term{dzs}}
\NewDocumentCommand{\DOutput}{}{\Term{doutput}}

% Semantic brackets
% \mean t       for   ⟦t⟧
% \mean[\Gr]t   for   ⟦t⟧ ρ
% \mean*[\Gr]t  for   (⟦t⟧ ρ)
\NewDocumentCommand{\mean}{soomo}{\Parens#1{%
\left\llbracket\,#4\,\right\rrbracket\IfNoValueTF{#5}{}{^{#5}}\IfNoValueTF{#2}{}{\;#2}\IfNoValueTF{#3}{}{\;#3}}}

% Environment update
% \update x v \Gr   for   ρ[x ↦ v]
% \update* x v \Gr  for   (ρ[x ↦ v])
\NewDocumentCommand{\update}{smmm}{\Parens#1{#4[#2\mapsto#3]}}

% \D x   for  dx
%
% Warning: \D x typesets dx as a single word, unlike dx.
% See discussion of mathit in, e.g.,
% http://tex.stackexchange.com/a/58108/1340.
\NewDocumentCommand{\D}{m}{\Term{d#1}}

% \Old f for  f_old
\NewDocumentCommand{\Old}{m}{#1_{\textrm{old}}}
\NewDocumentCommand{\New}{m}{#1_{\textrm{new}}}

\NewDocumentCommand{\APPFun}{}{\Keyword{app}}
\NewDocumentCommand{\IF}{}{\mathop{\Keyword{if}}}
\NewDocumentCommand{\THEN}{}{\mathbin{\Keyword{then}}}
\NewDocumentCommand{\ELSE}{}{\mathbin{\Keyword{else}}}
\NewDocumentCommand{\TRUE}{}{\Keyword{true}}
\NewDocumentCommand{\FALSE}{}{\Keyword{false}}
\NewDocumentCommand{\BAG}{}{\mathop{\Keyword{Bag}}}
\NewDocumentCommand{\MERGE}{}{\Term{merge}}
\NewDocumentCommand{\NAT}{}{\Keyword{Nat}}
\NewDocumentCommand{\INT}{}{\Keyword{Int}}
\NewDocumentCommand{\BOOL}{}{\Keyword{Bool}}
\NewDocumentCommand{\EVAL}{}{\mathcal{Eval}}
\NewDocumentCommand{\DOM}{}{D}
\NewDocumentCommand{\NEGATE}{}{\Term{negate}}
\NewDocumentCommand{\CHANGE}{o}{\Delta\Index{#1}}
\NewDocumentCommand{\BASE}{}{\mathop{\Keyword{Base}}}

\NewDocumentCommand{\Parens}{mm}{\IfBooleanTF#1{\left(#2\right)}{#2}}

\NewDocumentCommand{\Fun}{smm}{\Parens#1{#2 \to #3}}
\NewDocumentCommand{\Base}{sO{}}{\Parens#1{\App\BASE{#2}}}
\NewDocumentCommand{\Bag}{sO{}}{\Parens#1{\IfValueTF{#1}{\App\BAG{#2}}{\BAG}}}
\NewDocumentCommand{\Nat}{s}{\Parens#1{\NAT}}
\NewDocumentCommand{\Int}{s}{\Parens#1{\INT}}

\NewDocumentCommand{\Set}{m}{\left\lbrace\left\lbrace#1\right\rbrace\right\rbrace}

\NewDocumentCommand{\APP}{}{\;}
\NewDocumentCommand{\App}{smm}{\Parens#1{#2\;#3}}
\NewDocumentCommand{\Lam}{smm}{\Parens#1{\lambda#2 \rightarrow \ #3}}
\NewDocumentCommand{\Var}{sm}{\Parens#1{#2}}
\NewDocumentCommand{\Const}{sm}{\Parens#1{#2}}
\NewDocumentCommand{\Lit}{sm}{\Parens#1{#2}}
\NewDocumentCommand{\Add}{smm}{\Parens#1{#2 + #3}}
\NewDocumentCommand{\Minus}{sm}{\Parens#1{- #2}}
\NewDocumentCommand{\Bool}{s}{\Parens#1{\BOOL}}
\NewDocumentCommand{\TrueT}{s}{\Parens#1{\TRUE}}
\NewDocumentCommand{\FalseT}{s}{\Parens#1{\FALSE}}
\NewDocumentCommand{\If}{smmm}{\Parens#1{\IF #2 \THEN #3 \ELSE #4}}
\NewDocumentCommand{\Empty}{s}{\Parens#1{\emptyset}}
\NewDocumentCommand{\BagLit}{m}{\left\lbrace#1\right\rbrace}
\NewDocumentCommand{\SINGLETON}{}{\Term{singleton}}
\NewDocumentCommand{\SingletonT}{sm}{\Parens#1{\App\SINGLETON{#2}}}
\NewDocumentCommand{\INSERT}{}{\Term{insert}}
\NewDocumentCommand{\InsertT}{smm}{\Parens#1{\App{\App\INSERT{#2}}{#3}}}
\NewDocumentCommand{\Merge}{smm}{\Parens#1{\App{\App\MERGE {#2}}{#3}}}
\NewDocumentCommand{\MapT}{smm}{\Parens#1{\App{\CMap{#2}}{#3}}}
\NewDocumentCommand{\CMap}{sm}{\Parens#1{\App{\MAP}{#2}}}
\NewDocumentCommand{\FLATMAP}{}{\Term{flatmap}}
\NewDocumentCommand{\Flatmap}{smm}{\Parens#1{\App{\App\FLATMAP{#2}}{#3}}}
\NewDocumentCommand{\Sum}{sm}{\Parens#1{\App{\Term{sum}}{#2}}}
\NewDocumentCommand{\Negate}{sm}{\Parens#1{\App{\NEGATE}{#2}}}
\NewDocumentCommand{\DELETE}{}{\Term{delete}}
\NewDocumentCommand{\Delete}{smm}
  {\Parens#1{\App{\App{\App{\DELETE}{#2}}{\Term{from}}}{#3}}}

\NewDocumentCommand{\Change}{som}{\Parens#1{\CHANGE[#2]#3}}

\NewDocumentCommand{\Eval}{m}{\mean{#1}}
\NewDocumentCommand{\EvalWith}{smm}{\Parens#1{\mean[#3]{#2}}}
\NewDocumentCommand{\EvalInc}{m}{\mean{#1}[\GD]}
\NewDocumentCommand{\EvalIncWith}{smmm}{\Parens#1{\mean[#3][#4]{#2}[\GD]}}

% For special situations.
\NewDocumentCommand{\EvalIncSmash}{m}{\mean{#1}[\smash{\GD}\!\!\!]}
\NewDocumentCommand{\EvalIncSmashWith}{smmm}{\Parens#1{\mean[#3][#4]{#2}[\smash{\GD}\!\!\!]}}

\NewDocumentCommand{\CONST}{}{\mathcal{C}}

\NewDocumentCommand{\EvalConst}{sm}{\Parens#1{\mean{#2}[\CONST]}}
\NewDocumentCommand{\DERIVE}{}{\mathcal{D}}
\NewDocumentCommand{\Derive}{sm}{\Parens#1{\DERIVE(#2)}}
\NewDocumentCommand{\DeriveConst}{sm}{\Parens#1{\DERIVE^{\CONST}(#2)}}
\NewDocumentCommand{\Dom}{m}{\DOM_{#1}}

\NewDocumentCommand{\Append}{smm}{\Parens#1{#2 , #3}}
\NewDocumentCommand{\HasType}{smm}{\Parens#1{#2 : #3}}
\NewDocumentCommand{\HasTypeAligned}{mm}{#1 & : #2}
\NewDocumentCommand{\EmptyContext}{}{\varepsilon}
\NewDocumentCommand{\Extend}{sO{\Gamma}mm}{\Parens#1{\Append{#2}{\HasType{#3}{#4}}}}
\NewDocumentCommand{\Typing}{O{\Gamma}mm}{#1 \vdash \HasType{#2}{#3}}
\NewDocumentCommand{\ConstTyping}{mm}{\vdash^{\CONST} \HasType{#1}{#2}}

\NewDocumentCommand{\HasValue}{smm}{\Parens#1{#2 = #3}}
% I'm not sure I like the implementation of \EmptyEnv, please fill in a better one.
\NewDocumentCommand{\EmptyEnv}{}{\varnothing}
\NewDocumentCommand{\ExtendEnv}{sO{\rho}mm}{\Parens#1{\Append{#2}{\HasValue{#3}{#4}}}}

\NewDocumentCommand{\Rule}{O{}mm}{\inferrule{#2}{#3}\;\textsc{#1}}
\NewDocumentCommand{\Axiom}{O{}m}{\Rule[#1]{\hbox{}}{#2}}

\NewDocumentCommand{\IMPLEMENTS}{oo}{\sim\Index{#1}\Index*{#2}}
\NewDocumentCommand{\Implements}{soomm}{\Parens#1{#4\IMPLEMENTS[#2][#3] #5}}

% This is often inlined, so it can't be changed easily.
\NewDocumentCommand{\DeltaType}{m}{\GD #1}

\NewDocumentCommand{\DeltaContext}{m}{\GD #1}

% Proof-embedded domains and set of validity proofs
\NewDocumentCommand{\PD}{}{E}
\NewDocumentCommand{\VP}{}{V}
\NewDocumentCommand{\Pd}{m}{\PD_{\GD#1}}
\NewDocumentCommand{\Vp}{mm}{\VP[#1,#2]}
\NewDocumentCommand{\Vt}{mmm}{\VP_{#1}[#2,#3]}

\newenvironment{syntax}
{\[\begin{tabular}{>{$}r<{$}@{\;}>{$}c<{$}@{\;}>{$}l<{$}@{\qquad}l}}
{\end{tabular}\]}

\NewDocumentCommand{\FramedSignature}{m}
  {\fbox{\(#1\)}}

\NewDocumentCommand{\RightFramedSignature}{m}
  {\vbox{\hfill\FramedSignature{#1}}}

\NewDocumentEnvironment{typing}{o}
{\IfValueTF{#1}{\RightFramedSignature{#1}}{}
 \begin{mathpar}}
{\end{mathpar}}

\newenvironment{signature}
{\begin{tabular}{>{$}c<{$}@{$\hbox{}:\hbox{}$}>{$}l<{$}@{\qquad}l}}
{\end{tabular}}

% Proofs by case analysis
% Usage:
% \Case t = \Lam x s: Lorem ipsum dolor sit amet.
% \Case t = \App {t_0} {t_1}: Consectetur, adipisci velit.
\def\Case#1:{\smallbreak\noindent\textit{Case} $#1$:\par}

% Separation between groups of equations
\newdimen\eqsep
\eqsep=1em

% refering to items
\def\itref#1{(\ref{#1})}

% HISTOGRAM EXAMPLE
\NewDocumentCommand{\HISTOGRAM}{}{\Term{histogram}}
\NewDocumentCommand{\WORDCOUNT}{}{\Term{wordcount}}
\NewDocumentCommand{\HASHMAP}{}{\Keyword{Map}}
\NewDocumentCommand{\HashMap}{mm}{\App{\App\HASHMAP{#1}}{#2}}
\NewDocumentCommand{\DOCUMENT}{}{\Keyword{Document}}
\NewDocumentCommand{\DOCUMENTID}{}{\Keyword{ID}}
\NewDocumentCommand{\WORD}{}{\Keyword{Word}}
% bags defined somewhere above
% foldbag
\NewDocumentCommand{\FOLD}{}{\Term{fold}}
\NewDocumentCommand{\FOLDBAG}{}{\Term{foldBag}}
\NewDocumentCommand{\FoldBag}{smm}{\Parens#1{\App{\App\FOLDBAG {#2}}{#3}}}

\NewDocumentCommand{\ABELIAN}{}{\Keyword{AbelianGroup}}
\NewDocumentCommand{\Abelian}{sm}{\Parens#1{\App\ABELIAN{#2}}}
\NewDocumentCommand{\ABELIANCONS}{}{\Term{abelianGroup}}
% maps
\NewDocumentCommand{\LIFTGROUP}{}{\Term{liftGroup}}
\NewDocumentCommand{\MAKEMAP}{}{\Term{singletonMap}}
\NewDocumentCommand{\FOLDMAP}{}{\Term{foldMap}}
\NewDocumentCommand{\LOOKUP}{}{\Term{lookup}}
%
\NewDocumentCommand{\MAYBE}{}{\Keyword{Maybe}}
\NewDocumentCommand{\Maybe}{sm}{\Parens#1{\App\MAYBE{#2}}}

% a series of indentations for the derivative of the running example
\newcommand\DeriveProgramEnv{
\def\zero{\hspace{1.375em}}
\def\one{\hspace{2.75em}}
\def\two{\hspace{5em}}
\def\three{\hspace{8.5em}}}
