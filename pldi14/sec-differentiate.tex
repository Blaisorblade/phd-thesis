% Emacs, this is -*- latex -*-!



\section{Incrementalizing \TitleLambda{}-calculi}
\label{sec:differentiate}
\label{sec:correctness}

\pg{reformat for new layout, things overlap!}
% Emacs, this is -*- latex -*-!
\begin{figure*}
\begin{tabular}{>{$}r<{$}@{$\;::=\;$}>{$}c<{$}@{$\;$}>{$}l<{$}@{\quad}>{(}l<{)}}
\Gi      & \rlap{\ldots} &                       & base types\\
\Gs, \Gt & \Gi           & \mid \Fun{\Gt}{\Gt}   & types\\
\GG      & \EmptyContext & \mid \Extend{x}{\tau} & typing contexts\\
c        & \rlap{\ldots} &                       & constants\\
s, t     & c             & \mid \Lam{x}{t}
                           \mid \App{t}{t}
                           \mid x                & terms
\end{tabular}
\caption{Our base calculus: Syntax}
\label{fig:syntax}
\end{figure*}

\begin{figure*}
\begin{typing}
\noindent
\Rule[Const]
  {\ldots}
  {\Typing[]{c}{\tau}}

\Axiom[Lookup]
  {\Typing[\Append{\GG_1}{\Append{\HasType{x}{\tau}}{\GG_2}}]{\Var{x}}{\tau}}

\raisebox{0.5\baselineskip}{\fbox{$\Typing{t}{\tau}$}}

\Rule[Lam]
  {\Typing[\Extend{x}{\Gs}]{t}{\Gt}}
  {\Typing{\Lam{x}{t}}{\Fun{\Gs}{\Gt}}}

\Rule[App]
  {\Typing{s}{\Fun{\Gs}{\Gt}}\\
   \Typing{t}{\Gs}}
  {\Typing{\App{s}{t}}{\Gt}}
\end{typing}
\caption{Our base calculus: Typing}
\label{fig:typing}
\end{figure*}


\pg{Here we need an example like the one in Sec. 2.2, using the syntactic counterpart to closures, that is open terms.}
In this section, we show how to incrementalize an arbitrary
program in simply-typed $\Gl$-calculus. To this end, we define
the source-to-source transformation $\DERIVE$. Using the
denotational semantics $\Eval{-}$ we define later (in
\cref{sec:denotational-sem}), we can specify $\DERIVE$'s intended
behavior: to ensure \cref{eq:correctness},
$\Eval{\Derive{f}}$ must be the derivative of $\Eval{f}$
for any closed term $\HasType{f}{A \to B}$. We will overload the word
``derivative'' and say simply that $\Derive{f}$ is the derivative of
$f$.

It is easy to define derivatives of arbitrary functions as:
\[\App{\App{f'}{x}}{\D x} = \Diff{\App{f}{\Update*{x}{\D x}}}{\App f x}\text{.}\]
We could implement $\DERIVE$ following the same strategy.
However, the resulting incremental programs would be no faster
than recomputation. We cannot do better for arbitrary mathematical functions,
since they are infinite objects which we cannot fully inspect.
%
\pg{Revise - I reordered sentences to sort them logically with minimal rewriting.}%
Therefore, we resort to a source-to-source transformation
on simply-typed $\Gl$-calculus as defined in 
\cref{fig:syntax,fig:typing}. In this section, we focus on the
incrementalization of the features that are shared among all
instances of the plugin interface, that is, function types and the
associated syntactic forms, $\Gl$-abstraction, application and
variable references.

The sets of base types and primitive
constants, as well as the typing rules for primitive constants, are
on purpose left unspecified and only defined by plugins --- they are \emph{extensions points}.
Definitions provided by the plugin are replaced, in figures, by ellipses
(``$\ldots$'').
Defining different plugins allows to experiment with
sets of base types, associated primitives and incrementalization strategies.
We summarize requirements on plugins in \cref{ssec:plugin}:
Satisfying these requirements is sufficient to ensure
correct incrementalization.
We show an example plugin in our case study
(\cref{sec:plugins}).

\subsection{Change types and erased change structures}
\label{ssec:change-types}

% Emacs, this is -*- latex -*-!
\begin{figure}
\begin{signature}
\CHANGE  & \Fun{\ast}{\ast}
         & the type of changes\\[0.5ex]
\APPLY   & \Fun{\tau}{\Fun{\Change{\tau}}{\tau}}
         & update a value with a change\\
\DIFF    & \Fun{\tau}{\Fun{\tau}{\Change{\tau}}}
         & the change between two values\\
\end{signature}
\caption{Erased change structures on simple types.}
\label{fig:change-operations}
\end{figure}

% Emacs, this is -*- latex -*-!
\begin{figure}
\begin{align*}
\Change{\Fun* \Gs \Gt} &= \Fun{\Gs}{\Fun{\Change\Gs}{\Change\Gt}}\\
\DIFF_{\Fun{\Gs}{\Gt}} & = \Lam{g\ f\ x\ \D x}
  {\Diff{\App*g{\Apply*{\D x}{x}}}{\App*f x}}\\
\APPLY_{\Fun{\Gs}{\Gt}} & = \Lam{f\ \D f\ x}
  {\Apply{\App*{\App{\D f}{x}}{\Diff*xx}}{\App*f x}}
\end{align*}
\caption{The erased change structures for function types.}
\label{fig:diff-apply}
\end{figure}


We developed the theory of change structures in the previous
section to guide our implementation of $\DERIVE$. By
\cref{thm:nil-is-derivative}, $\DERIVE$ has only to find the nil
change to the program itself, because nil changes \emph{are}
derivatives. However, the theory of change structures is not
directly applicable to the simply-typed $\Gl$-calculus, 
because a precise implementation of
change structures requires dependent types. For instance,  
we cannot describe the set of
changes $\Change[\Gt]{v}$ precisely as a non-dependent type, because it depends on the value we plan
to update with these changes. 




To work around this limitation of
our object language, we use a form of \emph{erasure} of dependent types
to simple types. In \cref{fig:change-operations} and \cref{fig:correctness:change-types}, we
define change types $\Change{\Gt}$ as an approximate description
of change sets $\Change[\Gt]{v}$ (\cref{fig:correctness:changes}). 
In particular, all changes in $\Change[\Gt]{v}$ correspond to values of terms with type $\Change{\Gt}$,
but not necessarily the other way around. 
For instance, in the
change structure for natural numbers described in \cref{ssec:change-structures}, we would
have $\Change{\Nat} = \Int$, even though not every
integer is a change for every natural number.
For primitive types $\iota$, 
$\Change{\iota}$ and its associated $\UPDATE$ and $\DIFF$ operator
must be provided by the plugin developer.
For function types, erased change structures are given by \cref{fig:diff-apply}.
%
Erasing dependent types in all components of a change structure,
we obtain \emph{erased change structures}, which represent change
structures as simply-typed $\Gl$-terms
where $\UPDATE$ and $\DIFF$ are
families of $\Gl$-terms. 

Erased change structures are not change structures themselves.
However, we will show how change structures and erased changes
structures have ``almost the same'' behavior
(\cref{sec:differentiate-correct}). We will hence be able to
apply our theory of changes.

\subsection{Differentiation}
\label{ssec:differentiation}
% The following should explain the invariant for \DERIVE.

When $f$ is a closed term of function type,
$\Derive{f}$ should be its derivative, that is its nil change.
Since $\DERIVE$ recurses on open terms, we need a more general
specification.
%
% We don't need to mention parameters, and not mentioning them
% simplifies the discussion. We will mention them again in the case for abstraction.
We require that if $\Typing{t}{\tau}$, then $\Derive{t}$
represents the change in $t$ (of type $\Change{\Gt}$) in terms of
changes to the values of its free variables. As a special case,
when $t$ is a closed term, there is no free variable to change;
hence, the change to $t$ will be, as desired, the nil change of
$t$.

The following typing rule shows the static semantics of
$\DERIVE$:
\begin{typing}
\Rule[Derive]
  {\Typing{t}{\tau}}
  {\Typing[\Append{\GG}{\DeltaContext{\GG}}]{\Derive{t}}{\DeltaType{\tau}}}
\end{typing}

We see that $\Derive{t}$ has access both to the
free variables in $t$ (from $\GG$) and to their changes (from
$\DeltaContext{\GG}$, defined in
\cref{fig:correctness:change-contexts}).
For example, if a
well-typed term $t$ contains $x$ free, then $\GG$ contains an
assumption $\HasType{x}{\Gt}$ for some $\Gt$ and
$\DeltaContext{\GG}$ contains the corresponding assumption
$\HasType{\D x}{\DeltaType{\Gt}}$. Hence, $\Derive{t}$ can
access the change of $x$ by using $\D x$. For simplicity, 
we assume that the original program contains no variable names
that start with $\D{}$.
The definition of $\DERIVE$ will ensure that
the $\D x$ variables are bound if the original term is closed.

Let us analyzes each case of the definition of $\Derive{u}$
(\cref{fig:correctness:derive}):
\begin{itemize}
\item If $u = x$, $\Derive{x}$ must be the change of $x$, that is
$\D x$.
\item If $u = \Lam{x}{t}$, $\Derive{t}$ is the change of
  $u$ given the change in its free variables. The change of $u$
  is then the change of $t$ as a function of the \emph{base input}
  $x$ and its change
  $\D x$, with respect to changes in other open variables. Hence,
  we simply need to bind $\D x$ by defining $\Derive{\Lam{x}{t}}
  = \Lam{x}{\Lam{\D x}{\Derive{t}}}$.
\item If $u = \App{s}{t}$, $\Derive{s}$ is the change of $s$ as a function
  of its base input and change. Hence, we simply apply $\Derive{s}$ to 
  the actual base input $t$ and change $\Derive{t}$, giving
  $\Derive{\App{s}{t}} =
  \App{\App{\Derive{s}}{t}}{\Derive{t}}$.
\item If $t = c$: since $c$ is a closed term, its
  change is a nil change, hence (by \cref{thm:nil-is-derivative}) $c$'s derivative. We can
  obtain a correct derivative for constants by setting:
  \[
  \Derive{c} =
  \Diff{c}{c} = \NilC{c} = c'
  \]
  This definition is inefficient for functional constants; hence plugins must provide derivatives
  of the primitives they define.
\end{itemize}

This might seem deceptively simple. But $\lambda$-calculus only
implements binding of values, leaving ``real work'' to
primitives; likewise, differentiation for $\lambda$-calculus only
implement binding of changes, leaving ``real work'' to
derivatives of primitives.
However, our support for
$\Gl$-calculus allows to \emph{glue} the primitives together.

\begin{examples}
Let us apply the transformation on the program $\Program$ defined
in \cref{sec:intro}.
{\DeriveProgramEnv
\begin{align*}
&\Program = \ProgramBody\\
&\Derive\Program=\\
&\zero
\Lam{\Xs}{\Lam{\DXs}{}}\Lam{\Ys}{\Lam{\DYs}{}}\\
&\one
\FOLD'~(+)~(+')~0~0'\\
&\two
(\Merge\Xs\Ys)\\
&\two
(\MERGE'~\Xs~\DXs~\Ys~\DYs)
\end{align*}
}%
The names $\FOLD'$, $\MERGE'$, $+'$, $0'$ stand for the
derivatives of the corresponding primitives. The variables
$\DXs$ and $\DYs$ are systematically named after $\Xs$
and $\Ys$ to stand for their changes. As we shall see in
\cref{ssec:plugin},
\[
\MERGE'=\Lam{u}{\Lam{\D u}{\Lam{v}{\Lam{\D v}{\Merge{\D u}{\D v}}}}},
\]
so the derivative of $\Program$ is $\beta$-equivalent to
{\DeriveProgramEnv
\begin{align*}
&\zero
\Lam{\Xs}{\Lam{\DXs}{}}\Lam{\Ys}{\Lam{\DYs}{}}\\
&\one
\FOLD'~(+)~(+')~0~0'\\
&\two
(\Merge\Xs\Ys)~(\MERGE~\DXs~\DYs).
\end{align*}}%
%
This derivative is inefficient because it needlessly recomputes
$\Merge\Xs\Ys$. But we still need to inline the derivatives of
$\FOLD$ and other primitives to complete derivation. We'll
complete the derivation process and see how to avoid this waste in
\cref{ssec:self-maint}.
\end{examples}

We have now informally
derived the definition of $\DERIVE$ (\cref{fig:correctness:derive})
from its specification (\cref{eq:correctness}) and
its typing. But formally speaking,
$\DERIVE$ is instead a \emph{definition}. So in the rest of this section,
we'll have to
prove that $\DERIVE$ satisfies \cref{eq:correctness}.

% Emacs, this is -*- latex -*-!

\begin{figure*}
  \small

  \NewDocumentCommand{\Subcaption}{mm}
    {\subfigure[\label{#1}{#2}]{\rule{\linewidth}{0pt}}\vspace{0.8cm}}

  \NewDocumentCommand{\Align}{m}
    {{\begin{align*}#1\end{align*}}\vspace*{-0.8cm}}

  \centering
  \begin{tabular}{p{0.25\linewidth}p{0.40\linewidth}p{0.25\linewidth}}
    \hfill
    \FramedSignature{\Change{\Gt}}
    &
    \hfill
    \FramedSignature{\D v, \D f \in \Change[\Gt]{v}}
    &
    \hfill
    \FramedSignature{v, f \in \Eval{\Gt}}
    \\
    \Align{
      \Change{\iota}
        & = \ldots\\
      \Change{\Fun*{\Gs}{\Gt}}
        & = \Gs \to
            \Change{\Gs} \to
            \Change{\Gt}
    }
    &
    \Align{
      \Change[\iota]{v}
        & = \ldots \subseteq \Eval{\Change{\Gi}}\\
      \Change[\Fun*{\Gs}{\Gt}]{f}
        & = \left\{ \D f \in \HasType*{x}{\Eval{\Gs}} \to
            \Change[\Gs]{x} \to
            \Change[\Gt]{\App*{f}{x}} \mid \right.\\
        & \left.\qquad
          \App{\Apply*[A \to B]{\D f}{f}}{\Apply*[A]{\D a}{a}}
          = \Apply[B]{\App{\App{\D f}{a}}{\D a}}{\App{f}{a}}
        \right\}
    }
    &
    \Align{
      \Eval{\iota}
        & = \ldots\\
      \Eval{\Fun{\Gs}{\Gt}}
        & = \Eval{\Gs} \to \Eval{\Gt}
    }
    \\
    \Subcaption{fig:correctness:change-types}{
      Change types.
    }
    &
    \Subcaption{fig:correctness:changes}{
      Change values.
    }
    &
    \Subcaption{fig:correctness:values}{
      Standard values.
    }
    \\
    \hfill
    \FramedSignature{\Change{\GG}}
    &
    \hfill
    \FramedSignature{\D \Gr \in \Change[\GG]{\Gr}}
    &
    \hfill
    \FramedSignature{\Gr \in \Eval{\GG}}
    \\
    \Align{
      \Change{\EmptyContext}
        & = \EmptyContext \\
      \Change{\Extend*{x}{\Gt}}
        & = \Extend[\Change{\GG}]{\D x}{\Change{\Gt}}
    }
    &
    \Align{
      \Change[\EmptyContext]{\EmptyEnv}
        & = \left\{ \EmptyEnv \right\} \\
      \Change[\Extend*{x}{\Gt}]{\ExtendEnv*{x}{v}}
        & = \left\{ \ExtendEnv*[\D \Gr]{\D x}{\D v} \mid \D \Gr \in \Change[\GG]{\Gr} \land \D v \in \Change[\Gt]{v} \right\}
    }
    &
    \Align{
      \Eval{\EmptyContext}
        & = \left\{ \EmptyEnv \right\} \\
      \Eval{\Extend{x}{\Gt}}
        & = \left\{ \ExtendEnv*{x}{v} \mid \Gr \in \Eval{\GG} \land v \in \Eval{\Gt}\right\}
    }
    \\
    \Subcaption{fig:correctness:change-contexts}{
      Change contexts.
    }
    &
    \Subcaption{fig:correctness:change-environments}{
      Change environments.
    }
    &
    \Subcaption{fig:correctness:environments}{
      Standard environments.\pg{Fix overfull hbox!}
    }
    \\
    \hfill
    \FramedSignature{\Change{t}}
    &
    \hfill
    \FramedSignature{\EvalIncSmashWith{t}{\Gr}{\D \Gr}}
    &
    \hfill
    \FramedSignature{\EvalWith{t}{\Gr}}
    \\
    \Align{
      \Derive{\Const{c}}
        & = \ldots\\
      \Derive{\Lam{x}{t}}
        & = \lambda x\;\D x.\ \Derive{t}\\
      \Derive{\App{s}{t}}
        & = \App{\App{\Derive{s}}{t}}{\Derive{t}}\\
      \Derive{\Var{x}}
        & = \Var{\D x}
    }
    &
    \Align{
      \EvalIncSmashWith{c}{\rho}{\D \rho}
        & = \ldots\\
      \EvalIncSmashWith{\Lam{x}{t}}{\rho}{\D \rho}
        & = \lambda v\;\D v.\ 
                   \EvalIncSmashWith
                   {t}
                   {\ExtendEnv*{x}{v}}
                   {\ExtendEnv*[\D \rho]{\D x}{\D v}}\\
      \EvalIncSmashWith{\App{s}{t}}{\rho}{\D \rho}
        & = \App
                   {\App
                     {\EvalIncSmashWith*{s}{\rho}{\D \rho}}
                     {\EvalWith*{t}{\rho}}}
                   {\EvalIncSmashWith*{t}{\rho}{\D \rho}}\\
      \EvalIncSmashWith{x}{\Gr}{\D \Gr}
        & = \textit{lookup $\D x$ in $\D \Gr$}
    }
    &
    \Align{
      \EvalWith{c}{\rho}
        & =\ldots\\
      \EvalWith{\Lam{x}{t}}{\rho}
        & = \lambda v.\ \EvalWith{t}{\ExtendEnv*{x}{v}}\\
      \EvalWith{\App{s}{t}}{\rho}
        & = \App{\EvalWith*{s}{\rho}}{\EvalWith*{t}{\rho}}\\
      \EvalWith{x}{\Gr}
        & = \textit{lookup $x$ in $\Gr$}
    }
    \\
    \Subcaption{fig:correctness:derive}{
      Differentiation.
    }
    &
    \Subcaption{fig:correctness:change-evaluation}{
      Differential evaluation.
    }
    &
    \Subcaption{fig:correctness:evaluation}{
      Standard evaluation.
    }
  \end{tabular}

  \caption{Standard and differential behavior of the simply-typed
    $\lambda$-calculus.
    %
    The left column defines differentiation as a source-to-source
    transformation.
    %
    The right column defines the standard semantics of the
    simply-typed lambda calculus.
    %
    The middle column connects these artifacts via a differential
    semantics that maps $\Gl$-terms to the derivative of their
    standard semantics.}
  \label{fig:correctness}
\end{figure*}



\subsection{Architecture of the proof}

$\Derive{t}$ is defined using change types. As discussed in
\cref{ssec:change-types}, change types impose on their members
less restrictions than corresponding change structures -- they
contain ``junk'' (such as the change $-5$ for the natural number $3$). 
We cannot constrain the behavior of
$\Derive{t}$ on such junk; a direct correctness proof fails. To
avoid this problem, our proof defines a version of $\DERIVE$
which uses change structures instead.


To this end, we first present a standard denotational semantics
$\Eval{-}$ for simply-typed $\Gl$-calculus. Using our theory of
changes, we associate change structures to our domains. We define
a non-standard denotational semantics $\EvalInc{-}$, which is
analogous to $\DERIVE$ but operates on elements of change
structures, so that it needn't deal with junk. As a consequence,
we can prove that $\EvalInc{t}$ is the derivative of $\Eval{t}$:
this is our key result.

Finally, we define a correspondence between change sets and
domains associated with change types, and show that whenever
$\EvalInc{t}$ has a certain behavior on an input,
$\Eval{\Derive{t}}$ has the corresponding behavior on the
corresponding input. Our correctness property follows as a
corollary.

\subsection{Denotational semantics}
\label{sec:denotational-sem}

In order to prove that incrementalization preserves the meaning
of terms, we define a denotational semantics of the object
language. We first associate a domain with every type, given the
domains of base types provided by the plugin. Since our calculus
is strongly normalizing and all functions are total, we can
avoid using domain theory to model partiality: our domains are
simply sets. Likewise, we can use functions as the domain of function types.


\begin{definition}[Domains]
  The domain $\Eval{\Gt}$ of a type $\Gt$ is defined as in
  \cref{fig:correctness:values}.
\end{definition}

Given this domain
construction, we can now define an evaluation function for
terms. The plugin has to provide the evaluation function for
constants. In general, the evaluation function $\Eval{t}$ computes the value of a
well-typed term $t$ given the values of all free variables in
$t$. The values of the free variables are provided in an
environment.

\begin{definition}[Environments]
  An environment $\Gr$ assigns values to the names of free
  variables.

  \begin{syntax}
    \Gr ::= \EmptyContext \mid \ExtendEnv{x}{v}
  \end{syntax}

  We write $\Eval{\GG}$ for the set of environments that assign
  values to the names bound in $\GG$ (see
  \cref{fig:correctness:environments}).
\end{definition}

\begin{definition}[Evaluation]
  \label{def:evaluation}
  Given $\Typing{t}{\tau}$, the meaning of $t$ is defined by the
  function $\Eval{t}$ of type $\Fun{\Eval{\GG}}{\Eval{\tau}}$
  in \cref{fig:correctness:evaluation}.
\end{definition}

This is the standard semantics of the simply-typed
$\Gl$-calculus.
We can now specify what it means to incrementalize the
simply-typed $\Gl$ calculus with respect to this semantics.

\subsection{Change semantics}
The informal specification of differentiation is to map
changes in a program's input to changes in the program's
output. In order to formalize this specification in terms of
change structures and the denotational semantics of the object
language,
we now define a non-standard denotational semantics of the object
language that computes changes. The evaluation function
$\EvalInc{t}$ computes how the value of a well-typed term $t$
changes given both the values and the changes of all free
variables in $t$.
In the special case that none of the free variables change,
$\EvalInc{t}$ computes the nil change. By
\cref{thm:nil-is-derivative}, this is the derivative of
$\Eval{t}$ which maps changes to the input of $\Eval{t}$ to
changes of the output of $\Eval{t}$, as required.

First, we define a change structure on $\Eval{\Gt}$ for all
$\Gt$. The carrier $\Change[\Gt]$ of these change structures will
serve as non-standard domain for the change semantics. The plugin
provides a change structure $\ChangeStruct{C}_\Gi$ on base type
$\Gi$ such that $\forall v. \Change[\Gi]{v} \subseteq \Eval{\Change{\Gi}}$.


\begin{definition}[Changes]
  Given a type $\Gt$, we define a change structure
  $\ChangeStruct{C}_\Gt$ for $\Eval{\Gt}$ by induction on the
  structure of $\Gt$. If $\Gt$ is a base type $\Gi$, then
  the result $\ChangeStruct{C}_\Gi$ is supplied by the plugin.
  Otherwise we use the construction from \cref{thm:func-changestruct} and
  define
  \begin{align*}
%    \ChangeStruct{C}_{\Gi} & = \ChangeStruct{C}_\Gi\\
    \ChangeStruct{C}_{\Fun{\Gs}{\Gt}} & = \ChangeStruct{C}_{\Gs} \to \ChangeStruct{C}_{\Gt}.
  \qedAligned
  \end{align*}
\end{definition}

To talk about the derivative of $\Eval{t}$, we need a change
structure on its domain, the set of environments.
Since environments are (heterogeneous) lists of values, we
can lift operations on change structures to change structures on
environments acting pointwise.

\begin{definition}[Change environments]
  \label{def:change-environments}
  Given a context $\GG$, we define a change
  structure $\ChangeStruct{C}_\GG$ on the corresponding
  environments $\Eval{\GG}$ and change environments $\Change[\GG]{\Gr}$
  in \cref{fig:correctness:change-environments}.

  The operations $\APPLY[\Gr]$ and $\DIFF[\Gr]$ are defined as follows.
%
  \begin{align*}
    \Apply{\EmptyContext}{\EmptyContext} &= {\EmptyContext} \\
    \Apply{\ExtendEnv*[\D \Gr]{\D x}{\D v}}{\ExtendEnv*{x}{v}} &= \ExtendEnv[\Apply*{\D \Gr}{\Gr}]{x}{\Apply*{\D v}{v}} \\[\eqsep]
    \Diff{\EmptyContext}{\EmptyContext} &= {\EmptyContext} \\
    \Diff{\ExtendEnv*[\Gr_2]{x}{v_2}}{\ExtendEnv*[\Gr_1]{x}{v_1}} &= \ExtendEnv[\Diff*{\Gr_2}{\Gr_1}]{x}{\Diff*{v_2}{v_1}}
  \end{align*}
%
  The properties in \Cref{def:change-struct} follow directly from the same properties
  for the underlying change structures $\ChangeStruct{C}_\Gt$.
\end{definition}

At this point, we can define the change semantics of terms and
prove that $\EvalInc{t}$ it is the derivative of $\Eval{t}$. For
each constant $c$, the plugin provides $\EvalInc{c}$, the derivative
of $\Eval{c}$.

\begin{definition}[Change semantics]
  \label{def:change-evaluation}
  The function $\EvalInc{t}$ is defined in
  \cref{fig:correctness:change-evaluation}.
\end{definition}

\begin{lemma}
  \label{lem:change-semantics-correct}
  Given $\Typing{t}{\Gt}$, $\EvalInc{t}$ is the derivative of $\Eval{t}$.
\end{lemma}

\begin{optionalproof}
\pg{This optional proof is phrased for fully applied constants.}
  Given a derivation of $\Typing{t}{\Gt}$, an environment $\Gr
  \in \Eval{\GG}$, and a corresponding change environments $\D
  \Gr \in \Change[\GG]{\Gr}$, we prove
  \[
    \Apply{\EvalIncWith*{t}{\Gr}{\D \Gr}}
          {\EvalWith*{t}{\Gr}}
    =
    \EvalWith{t}{\Apply*{\D \Gr}{\Gr}}
  \]
  by induction on the structure of the derivation of
  $\Typing{t}{\Gt}$. There is one case for each of the typing
  rules in \cref{fig:typing}.

  \Case \Lam{x}{t_1}: In this case, $\Gt = \Fun{\Gs}{\Gt}$
  and we prove extensional equality of two functions. For an
  arbitrary value $v \in \Eval{\Gs}$, we have
  \begin{align*}
    &         \App
                {\Apply*
                  {\EvalIncWith{\Lam{x}{t_1}}{\Gr}{\D \Gr}}
                  {\EvalWith{\Lam{x}{t_1}}{\Gr}}}
                {v}\\
    & \quad = \Apply
                {\App
                  {\App
                    {\EvalIncWith*{\Lam{x}{t_1}}{\Gr}{\D \Gr}}
                    {v}}
                  {\NilC[\Gt _1]{v}}}
                {\App
                  {\EvalWith*{\Lam{x}{t_1}}{\Gr}}
                  {v}}\\
    & \quad = \Apply
                {\EvalIncWith
                  {t_1}
                  {\ExtendEnv*[\Gr]{x}{v}}
                  {\ExtendEnv*[\D \Gr]{\D x}{\NilC{v}}}}
                {\EvalWith
                  {t_1}
                  {\ExtendEnv*[\Gr]{x}{v}}}\\
    & \quad = \EvalWith
                {t_1}
                {\Apply*
                  {\ExtendEnv*[\D \Gr]{\D x}{\NilC{v}}}
                  {\ExtendEnv*[\Gr]{x}{v}}}\\
    & \quad = \EvalWith
                {t_1}
                {\ExtendEnv*
                  [\Apply{\D \Gr}{\Gr}]
                  {x}
                  {\Apply{\NilC{v}}{v}}}\\
    & \quad = \App
                {\EvalWith*
                  {\Lam{x}{t_1}}
                  {\Apply*{\D \Gr}{\Gr}}}
                {\Apply*{\NilC{v}}{v}}\\
    & \quad = \App
                {\EvalWith*
                  {\Lam{x}{t_1}}
                  {\Apply*{\D \Gr}{\Gr}}}
                {v}
  \end{align*}
  by
  \cref{def:function-changes:update,def:evaluation,def:change-evaluation},
  the induction hypothesis on $t_1$,
  \cref{def:change-environments,thm:update-nil}.

  \Case \App{s}{t_1}: We have
  \begin{align*}
    &         \Apply
                {\EvalIncWith{\App{s}{t_1}}{\Gr}{\D \Gr}}
                {\EvalWith{\App{s}{t_1}}{\Gr}}\\
    & \quad = \Apply
                {\App
                  {\App
                    {\EvalIncWith*{s}{\rho}{\D \rho}}
                    {\EvalWith*{t_1}{\rho}}}
                  {\EvalIncWith*{t_1}{\rho}{\D \rho}}}
                {\App
                  {\EvalWith*{s}{\Gr}}
                  {\EvalWith*{t_1}{\Gr}}}\\
    & \quad = \App
                {\Apply*
                  {\EvalIncWith{s}{\Gr}{\D \Gr}}
                  {\EvalWith{s}{\Gr}}}
                {\Apply*
                  {\EvalIncWith{t_1}{\Gr}{\D \Gr}}
                  {\EvalWith{t_1}{\Gr}}}\\
    & \quad = \App
                {\EvalWith*{s}{\Apply*{\D \Gr}{\Gr}}}
                {\EvalWith*{t_1}{\Apply*{\D \Gr}{\Gr}}}\\
    & \quad = \EvalWith
                {\App{s}{t_1}}
                {\Apply*{\D \Gr}{\Gr}}
  \end{align*}
  by
  \cref{def:evaluation,def:change-evaluation,thm:incrementalization}
  and the induction hypotheses on $s$ and $t_1$.

  \Case \Var{x}: Let $v$ be the entry for $\Var{x}$ in $\Gr$, and
  let $\D v$ be the entry for $\Var{\D x}$ in $\D \Gr$. We know that
  these entries exist from $\Typing{\Var{x}}{\Gt}$ with $\Gr \in
  \Eval{\GG}$ and $\D \Gr \in \Change[\GG]{\Gr}$. The entry for
  $\Var{x}$ in $\Apply{\D \Gr}{\Gr}$ is $\Apply{\D v}{v}$, so we
  have
  \begin{align*}
    &         \Apply
                {\EvalIncWith*{\Var{x}}{\Gr}{\D \Gr}}
                {\EvalWith*{\Var{x}}{\Gr}}\\
    & \quad = \Apply{\D v}{v}\\
    & \quad = \EvalWith{\Var{x}}{\Apply*{\Gr}{\D \Gr}}
  \end{align*}
  as required.

\pg{Drop arguments here to update the proof!}
  \Case \Const{c}{\List{t}}: We have
  \begin{align*}
    &         \Apply
                {\EvalIncWith*{\Const{c}{\List{t}}}{\Gr}{\D \Gr}}
                {\EvalWith*{\Const{c}{\List{t}}}{\Gr}}\\
    & \quad = \Apply
                {\EvalIncConst*
                  {\Const{c}}
                  {\List*{\EvalWith{t}{\Gr}}}
                  {\List*{\EvalIncWith{t}{\Gr}{\D \Gr}}}}
                {\EvalConst*
                  {\Const{c}}
                  {\List*{\EvalWith{t}{\Gr}}}}\\
    & \quad = \EvalConst
                {c}
                {\List*{\EvalIncWith{t}{\Gr}{\D \Gr}}}\\
    & \quad = \EvalConst
                {c}
                {\List*{\EvalWith{t}{\Apply*{\D \Gr}{\Gr}}}}\\
    & \quad = \EvalWith
                {\Const{c}{\List{t}}}
                {\Apply*{\D \Gr}{\Gr}}
  \end{align*}
  by \cref{def:change-evaluation} and
  the induction hypotheses on the terms $\List{t}$.
\end{optionalproof}

\subsection{Correctness of differentiation}
\label{sec:differentiate-correct}

\DeclareFixedFootnote{\EmptyEmptyNote}{%
To evaluate a closed term $t$, we need no environment entries, so
the empty environment $\EmptyEnv$ suffices:
$\EvalWith*{t}{\EmptyEnv}$ is the value of $t$ in the empty environment, and
$\;\EvalIncSmashWith*{t}{\EmptyEnv}{\EmptyEnv}$
is the value of $t$ using the change semantics, the empty environment and the empty change
environment.}

We can now
prove that the behavior of $\Eval{\Derive{t}}$ is consistent with
the behavior of $\EvalInc{t}$. This leads us to the proof of the
correctness theorem mentioned in the introduction.

The logical relation~\citep[Chapter 8]{Mitchell1996foundations}
of \emph{erasure} captures the idea that an
element of a change structure stays almost the same after we
erase all traces of dependent types from it.

\begin{definition}[Erasure]\label{def:erasure}
Let $\D v \in \Change[\Gt]v$ and $\D v' \in \Eval{\Change\Gt}$.
We say $\D v$ erases to $\D v'$, or
$\Implements[\Gt][v]{\D v}{\D v'}$,
if one of the following holds:
\begin{subdefinition}
\item $\Gt$ is a base type and $\D v=\D v'$.
\item $\Gt=\Fun{\Gs_0}{\Gs_1}$ and for all $w$, $\D w$, $\D w'$
such that $\Implements[\Gs_0][w]{\D w}{\D w'}$, we have
$\Implements[\Gs_1][\App* v w]
{\App*{\App{\D v}w}{\D w}}
{\App*{\App{\D v'}w}{\D w'}}$. \qed
\end{subdefinition}
\end{definition}

Sometimes we shall also say that $\D v \in \Change[\Gt] v$ erases
to a closed term $\HasType{\D t}{\GD t}$, in which case we mean
$\D v$ erases to $(\EvalWith{\D t}{\EmptyEnv})$.\EmptyEmptyNote

The following lemma makes precise what we meant by
``almost the same''.

\begin{lemma}\label{lem:almost-the-same}
Suppose $\Implements[\Gt][v]{\D v}{\D v'}$. If $\UPDATE'$ is the
erased version of the update operator $\UPDATE$ of the change
structure of $\Gt$ (\cref{ssec:change-types}), then
\[
v \UPDATE \D v = v \UPDATE' \D v'. \qed
\]
\end{lemma}

It turns out that $\EvalIncWith{t}$ and $\Derive{t}$ are ``almost the
same''. For closed terms, we make this precise by:

\begin{lemma}
  \label{lem:derive-correct}
If $(\HasType t \Gt)$ is closed, then $\EvalIncSmashWith*{t}\EmptyEnv\EmptyEnv$ erases to
$\Derive{t}$.
\end{lemma}

\begin{optionalproof}
  \pg{Say that we omit the proof because it's extremely boring.
    The recursion scheme seems less obvious than I'd have
    expected since it uses a logical relation but also deals with open terms.}
\end{optionalproof}

We omit for lack of space a more general version of
\cref{lem:derive-correct}, which holds also for open terms, but
requires defining erasure on environments.
The main correctness theorem is a corollary of
\cref{lem:almost-the-same,lem:derive-correct,lem:change-semantics-correct}.

\begin{theorem}[Correctness of differentiation]
\label{thm:main}
Let $\HasType{f}{\Fun \Gs \Gt}$ be a closed term of function
type. For every closed base term $\HasType{s}{\Gs}$ and for every closed change term
$\HasType{\D s}{\Change\Gs}$ such that some change
$\D v\in\Change[\Gs]{\Eval{s}}$ erases to $\D s$, we
have
\[
  \App{f}{\Update*{s}{\D s}}
\cong
  \Update{\App*{f}{s}}{\App*{\App{\Derive f}{s}}{\D s}},
\]
where $\cong$ is denotational equality ($a \cong b$ iff $\Eval{a} = \Eval{b}$).
\end{theorem}
\cref{thm:main} is a more precise restatement of
\cref{eq:correctness}. Requiring the existence of $\D v$ ensures
that $\D s$ evaluates to a change, and not to junk in
$\Eval{\Change\Gs}$.

\begin{oldSec}
\begin{parameter}
  \label{def:implements-base}
  For every base type $\Gi$ and base value $b \in \Eval{\Gi}$,
  base changes $\D b \in \Change[\Gi]{b}$ and values $v' \in
  \Eval{\Change{\Gi}}$ that implement the same change are related
  by $\Implements[\Gi][b]{\D b}{v'}$.
\end{parameter}

\begin{parameter}
  \label{lem:carry-over-base}
  If $\Implements[\Gi][b]{\D b}{v'}$,
  then $\Apply{\D b}{b} = \Apply{v'}{b}$.
\end{parameter}

\begin{parameter}
  \label{lem:diff-implements-diff-base}
  \[\Implements[\Gi][b_1]{\Diff{b_2}{b_1}}{\Diff{b_2}{b_1}}\]
\end{parameter}

\begin{definition}
  \label{def:implements}
  For every type $\Gt$ and value $v \in \Eval{\Gt}$, changes $\D
  v \in \Change[\Gt]{v}$ and values $v' \in \Eval{\Change{\Gt}}$
  that implement the same change are related by
  $\Implements[\Gt][v]{\D v}{v'}$. This relation is defined
  inductively in the structure of $\Gt$ as follows.

  \begin{itemize}
  \item $\Implements[\Gi][b]{\D b}{v'}$ 
    if and only if
    $\Implements[\Gi][b]{\D b}{v'}$.
  \item $\Implements[\Fun{\Gs}{\Gt}][f]{\D f}{f'}$
    if and only if
    \[
      \Implements[\Gs][v]
        {\D v}
        {v'}
      \implies
      \Implements[\Gt][\App*{f}{v}] 
        {\App{\App{\D f}{\D v}}{v}}
        {\App{\App{f'}{v'}}{v}}
    \]
    for all
    $v \in \Eval{\Gs}$,
    $\D v \in \Change[\Gs]{v}$, and
    $v' \in \Eval{\Change{\Gs}}$.
  \end{itemize}
\end{definition}

\begin{lemma}
  \label{lem:carry-over}
  If $\Implements[\Gt][v]{\D v}{v'}$,
  then $\Apply{\D v}{v} = \Apply{v'}{v}$.
  \tr{That should be different $\APPLY$, somehow.}
\end{lemma}
\end{oldSec}

\begin{oldSec}

\subsection{Old contents of this section}

\begin{theorem}[correctness of differentiation]
\label{thm:main-findiff}
if $s$ is a closed term of type $\Fun*\Gs\Gt$ and $t_0$, $t_1$
closed terms of type $\Gs$, then
\[
\Eval{\App s{t_1}}
=
\Eval{\Apply
{\App*{\App{\Derive s}{t_0}}{\Diff*{t_1}{t_0}}}
{\App* s{t_0}}}.
\]
\end{theorem}

Although theorem~\ref{thm:main-findiff} is quantified over
programs with all types of inputs and outputs, in practice we
care only about programs that take values of base type (i.\ e.,
numbers, bags and other non-functions) as input
and produce concrete data as output.
\yc{Here be a justification for using the term difference
operator to produce input change. Any idea how to formulate it
succinctly?}

To prove theorem~\ref{thm:main-findiff}, we need two lemmas about derivatives.

First, consider an open term $t$, its derivative
$\Derive{t}$ and an environment $\Gr$ which closes $\Derive{t}$.
Then, $\Derive{t}$ is the change of $t$ with respect to changes
to values denoted by its free variables in an environment $\Gr$.
This property can be expected to hold
%
only if the environment $\Gr$ does not assign garbage to some $\D
x$ that cannot be interpreted as a \emph{valid} change to
$\Gr(x)$. We defer defining when a change is valid for a
value; assuming this notion, we define when an environment
is \emph{consistent}:
%
\pg{variable pairs have not been defined, so I commented out the definition
%
''
if it assigns variable pairs to values valid
for each other.
''}

\begin{definition}
\label{def:consistency}
An environment $\Gr$ is consistent if $\Gr(\D x)$ is a valid
change to $\Gr(x)$ for every variable $x$.
\end{definition}

\begin{lemma}
\label{lem:freevars}
Let $t$ be a well-typed term, $\Gr$ a consistent environment, and
$\Gr'$ the updated version of $\Gr$. Then
\[
\EvalWith{ \Apply{\Derive t}{t} } \Gr = \EvalWith{t}{\Gr'}.
\]
\end{lemma}

Second, the value denoted by the
derivative responds correctly to changes to all future
arguments.

\begin{lemma}
\label{lem:future-args}
Let $t$ be a term of type $\Fun*\Gs\Gt$ and $\Gr$ a consistent
environment. If $\D v\in\Dom{\GD\Gs}$ is a valid change to
$v\in\Dom\Gs$, then
\begin{alignat*}{2}
&&&
(\EvalWith{\Apply{\Derive t}t}{\Gr})(\Apply{\D v}{v})\\
&=\;&&
\Apply
{(\EvalWith{\Derive t}{\Gr})(v)(\D v)}
{(\EvalWith{t}{\Gr})(v)}.
\end{alignat*}
\end{lemma}

Theorem~\ref{thm:main-findiff} is a consequence of lemmas
\ref{lem:freevars}~and \ref{lem:future-args} on closed terms. It
is difficult to prove the two lemmas directly. We shall encode
the idea behind lemma~\ref{lem:future-args} into a logical
relation on the semantic domains that allows us to show both
lemmas by one induction on typing judgements. A logical relation
$R$ has the characteristic that if two functions are related by
$R$, then they map $R$-related arguments to $R$-related results.
Let us call our logical relation the \emph{validity} of changes.
It makes precise the notion of meaningful changes to a value.

\begin{definition}[validity of changes]\label{def:valid}
The validity of changes is the union of relations between
$\Dom{\Gs}$ and $\Dom{\GD\Gs}$ defined inductively on the type
$\Gs$ thus:
\begin{itemize}
\item Every integer in $\Dom{\Int}$ is a valid change to every
integer in $\Dom{\Int}$.
\item Every bag in $\Dom{\Bag}$ is a valid change to every bag in
$\Dom{\Bag}$.
\item
A function $\D f \in \Dom{\Fun\Gs{\Fun{\GD\Gs}{\GD\Gt}}}$ is a
valid change to $f \in \Dom{\Fun\Gs\Gt}$ if for every
$v\in\Dom\Gs$ and every change $\D v\in\Dom{\GD\Gs}$ valid for
$v$ we have
\begin{enumerate}[(1)]
\item the result $\D f(v)(\D v)$ in the domain $\Dom{\GD\Gt}$ is
a valid change to $f(v)\in\Dom\Gt$,
\item the following equation holds:
\[
\Apply*{\D f}{f}(\Apply{\D v}{v}) = \Apply{f(v)}{\D f(v)(\D v)}.
\]
\end{enumerate}
\end{itemize}
\end{definition}

Intuitively, a valid change to a function $f$ must not only
modify the result at every point of input, but also respond to
changes to input in a manner similar to the incremental version
of $f$. We are ready to write down the induction hypothesis
strong enough to give us lemmas \ref{lem:freevars}~and
\ref{lem:future-args}.

\begin{lemma}[inductive reformulation of lemmas
\ref{lem:freevars}~and \ref{lem:future-args}]
\label{lem:hypothesis}
Let $t$ be a well-typed term, $\Gr$ a consistent environment and
$\Gr'$ the updated version of $\Gr$.
\begin{enumerate}[(i)]
\item\label{itm:freevars}
We have
$\EvalWith*{t}{\Gr'}=
\Apply{\EvalWith*{\Derive t}\Gr}{\EvalWith*{t}\Gr}$.
\item\label{itm:future-args}
$\EvalWith*{\Derive t}\Gr$ is a valid change to
$\EvalWith*{t}\Gr$.
\end{enumerate}
\end{lemma}

\begin{proof}[Proof fragment]
We prove it by induction on a typing judgement of the term $t$.
Here we show only the most nontrivial case where $t=\Lam x s$.

Suppose $\Gs\r\Gt$ is the type of $t$. Choose arbitrary $v\in
\Dom\Gs$ and let $\D v\in\Dom{\GD\Gs}$ be a valid change to $v$.
Define these shorthands:
\begin{align*}
f     & = \mean[\Gr]{\Lam{x} s} \\
\D f  & = \mean[\Gr]{\Lam{x}{\Lam{\D x}{\Derive s}}} \\
v'    & = \Apply{\D v}v,\\
\Gr_1 & = \Gr[x\mapsto v,\D x\mapsto\D v]
\end{align*}

For part~\itref{itm:freevars}, let
\begin{align*}
\Gr_2  & = \Gr[x\mapsto v,\D x\mapsto \Diff vv],\\
\Gr_2' & = \Gr'[x\mapsto v',\D x\mapsto \Diff{v'}{v'}].
\end{align*}
It is clear that $\Gr_2'$ is the updated version of $\Gr_2$. By
induction hypothesis on $s$,
\begin{align*}
\mean*[\Gr']t(v)
& = \mean[\Gr_2']s\\
& = \Apply{\mean*[\Gr_2]{\Derive s}}{\mean*[\Gr_2]s}\\
& = \Apply{\D f(v)(\Diff vv)}{f(v)}\\
& = \Apply*{\mean*[\Gr]{\Derive t}}{\mean*[\Gr]t}(v).
\end{align*}
Since $v$ is arbitrary, $\mean*[\Gr']t$ and
$\Apply*{\mean*[\Gr]{\Derive t}}{\mean*[\Gr]t}$ are equal as
functions.

For part~\itref{itm:future-args},
notice that
\begin{align*}
\D f(v')(\Diff{v'}{v'})
& = \mean[\Gr[x\mapsto v',\D x\mapsto\Diff{v'}{v'}]]{\Derive
s},\\
f(v')
& = \mean[\Gr[x\mapsto v']]s,
\end{align*}
and by induction hypothesis on $s$ and lemma \pg{was ref{lem:apply-diff}},
\begin{align*}
(\Apply{\D f}f)(\Apply{\D v}v)
& = (\Apply{\D f}f)(v')\\
& = \Apply{\D f(v')(\Diff{v'}{v'})}{f(v')}\\
& = \mean[\Gr'[x\mapsto\Apply{\Diff*{v'}{v'}}{v'}]]{s}\\
& = \mean[\Gr'[x\mapsto\Apply{\D v}{v}]]s\\
& = \Apply{\mean*[\Gr_1]{\Derive s}}
          {\mean*[\Gr_1]s}\\
& = \Apply{\D f(v)(\D v)}{f(v)}.
\end{align*}
Together with the validity of $\D f(v)(\D v)$ as a change to
$f(v)$ given by the induction hypothesis on $s$, we obtain
part~\itref{itm:future-args} of the lemma.
\end{proof}

%The reader may find a more complete version of the development of
%this section in appendix~\ref{sec:STLC-correct}.

Although we use a fixed set of primitives in the formalization,
the proof of the main correctness theorem does not depend on the
primitives beyond that their derivatives satisfy the
theorem themselves. It is possible to generalize the most
important induction steps, those on abstractions and
applications, to calculi with any suitable primitives. Thus, the
choice of primitives is unimportant; it serves mostly as an
understanding aid. A planned future work is to express this idea
in Agda by making our calculi and proofs parametric in term
constructors. \yc{or simply parametric in terms of primitives?}
\yc{This paragraph is intended to discourage objections of our
choice of primitives. Where is the best place to put it?}

\yc{If we have examples involving merge and sum alone, it is
possible to solve them right here.}

\end{oldSec}



\subsection{Plugins}\label{ssec:plugin}

Both our correctness proof and the differentiation framework (which is the 
basis for our implementation) are parametric in the plugin. 
Instantiating the differentiation framework requires a \emph{differentiation plugin};
instantiating the correctness proof for it  requires a \emph{proof           plugin}, containing additional definitions and lemmas.

To allow executing and differentiating $\Gl$-terms, a differentiation plugin must
provide:
\begin{itemize}
\item base types, and for each base type $\Gi$, the erased change structure of $\Gi$ as specified in
\cref{fig:change-operations},
\item primitives, and for each primitive $c$, the term $\Derive{c}$.
\end{itemize}
\begin{examples}
With bags of numbers as a primitive type, and a change structure
erased from $\ChangeStruct{\Bag S}$ (defined in
\cref{ssec:change-structures}), the derivative of $\MERGE$ is
easy to write down:
\[
\Derive{\MERGE}=\Lam{u}{\Lam{\D u}{\Lam{v}{\Lam{\D v}{\Merge{\D u}{\D v}}}}}
\]
In other words, the change to the merge of two bags is the merge of changes to
each bag.
\end{examples}

For each base type $\Gi$, a proof plugin must provide:
\begin{itemize}
\item a semantic domain $\Eval{\Gi}$,
\item a change structure $\ChangeStruct{C}_\Gi$ such that $\forall v. \Change[\Gi]{v} \subseteq \Eval{\Change{\Gi}}$,
\item a proof that $\ChangeStruct{C}_\Gi$ erases to the corresponding erased change structure in the differentiation plugin.
\end{itemize}
For each primitive $\HasType c \Gt$, the proof plugin must provide:
\begin{itemize}
\item its value $\Eval{c}$ in the domain $\Eval{\Gt}$,

\item its derivative $\EvalIncSmashWith*{c}{\EmptyEnv}{\EmptyEnv}$\EmptyEmptyNote{} in the change set of $\Gt$,
\item a proof that $\EvalIncSmashWith*{c}{\EmptyEnv}{\EmptyEnv}$ erases to the term $\Derive{c}$.
\end{itemize}

To show that the interface for proof plugins
can be implemented, we wrote a small proof plugin with
integers and bags of integers\yc{link to agda}.
To show that differentiation plugins are practicable, we 
have implemented the transformation and a differentiation plugin
which allows the incrementalization of non-trivial programs.
This is presented in the next section.
