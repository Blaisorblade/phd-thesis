% From http://tihlde.org/~eivindw/latex-listings-for-scala/
% "define" Scala
%Keyword list taken from the scaladoc definition.
\lstdefinelanguage{scala}{
  morekeywords={%
	  abstract,case,catch,class,def,do,else,extends,%
	  false,final,finally,for,forSome,if,implicit,import,lazy,%
	  match,new,null,object,override,package,private,protected,%
	  return,sealed,super,this,throw,trait,true,try,type,%
	  val,var,while,with,yield},
  otherkeywords={=>,<-,<\%,<:,>:,\#,@,scala>},
  sensitive=true,
  morecomment=[l]{//},
  morecomment=[n]{/*}{*/},
  morestring=[b]",
  morestring=[b]',
  morestring=[b]"""
}[keywords,comments,strings]

% activate the language and predefine settings
\lstset{%
    tabsize=2,%
    basicstyle=\scriptsize\ttfamily,%
    commentstyle=\itshape,%
    keywordstyle=\bfseries,%
    identifierstyle=,% nothing
    stringstyle=\ttfamily, % typewriter type for strings
    mathescape=true,%
    escapechar=£,%
    numberstyle=\tiny,%
}

%Setup whitespace in listings
\lstset{%
showstringspaces=false, %
%columns=fullflexible,		% enable kerning, lose column alignment.
breaklines=false,
breakatwhitespace=false,
breakautoindent=false,
keepspaces
%keepspaces was described at: http://tex.stackexchange.com/questions/41954/listings-bug-space-after-literate-replacement-lost-with-spaceflexible-fullflexi
}

\newcommand{\codesize}{\footnotesize}

% Command for in-text code snippets, if needed.
\newcommand{\co}[1]{%
    \lstinline[basicstyle=\codesize\ttfamily]^#1^}
