% Emacs, this is -*- latex -*-!
%% ODER: format ==         = "\mathrel{==}"
%% ODER: format /=         = "\neq "
%
%
\makeatletter
\@ifundefined{lhs2tex.lhs2tex.sty.read}%
  {\@namedef{lhs2tex.lhs2tex.sty.read}{}%
   \newcommand\SkipToFmtEnd{}%
   \newcommand\EndFmtInput{}%
   \long\def\SkipToFmtEnd#1\EndFmtInput{}%
  }\SkipToFmtEnd

\newcommand\ReadOnlyOnce[1]{\@ifundefined{#1}{\@namedef{#1}{}}\SkipToFmtEnd}
\usepackage{amstext}
\usepackage{amssymb}
\usepackage{stmaryrd}
\DeclareFontFamily{OT1}{cmtex}{}
\DeclareFontShape{OT1}{cmtex}{m}{n}
  {<5><6><7><8>cmtex8
   <9>cmtex9
   <10><10.95><12><14.4><17.28><20.74><24.88>cmtex10}{}
\DeclareFontShape{OT1}{cmtex}{m}{it}
  {<-> ssub * cmtt/m/it}{}
\newcommand{\texfamily}{\fontfamily{cmtex}\selectfont}
\DeclareFontShape{OT1}{cmtt}{bx}{n}
  {<5><6><7><8>cmtt8
   <9>cmbtt9
   <10><10.95><12><14.4><17.28><20.74><24.88>cmbtt10}{}
\DeclareFontShape{OT1}{cmtex}{bx}{n}
  {<-> ssub * cmtt/bx/n}{}
\newcommand{\tex}[1]{\text{\texfamily#1}}	% NEU

\newcommand{\Sp}{\hskip.33334em\relax}


\newcommand{\Conid}[1]{\mathit{#1}}
\newcommand{\Varid}[1]{\mathit{#1}}
\newcommand{\anonymous}{\kern0.06em \vbox{\hrule\@width.5em}}
\newcommand{\plus}{\mathbin{+\!\!\!+}}
\newcommand{\bind}{\mathbin{>\!\!\!>\mkern-6.7mu=}}
\newcommand{\rbind}{\mathbin{=\mkern-6.7mu<\!\!\!<}}% suggested by Neil Mitchell
\newcommand{\sequ}{\mathbin{>\!\!\!>}}
\renewcommand{\leq}{\leqslant}
\renewcommand{\geq}{\geqslant}
\usepackage{polytable}

%mathindent has to be defined
\@ifundefined{mathindent}%
  {\newdimen\mathindent\mathindent\leftmargini}%
  {}%

\def\resethooks{%
  \global\let\SaveRestoreHook\empty
  \global\let\ColumnHook\empty}
\newcommand*{\savecolumns}[1][default]%
  {\g@addto@macro\SaveRestoreHook{\savecolumns[#1]}}
\newcommand*{\restorecolumns}[1][default]%
  {\g@addto@macro\SaveRestoreHook{\restorecolumns[#1]}}
\newcommand*{\aligncolumn}[2]%
  {\g@addto@macro\ColumnHook{\column{#1}{#2}}}

\resethooks

\newcommand{\onelinecommentchars}{\quad-{}- }
\newcommand{\commentbeginchars}{\enskip\{-}
\newcommand{\commentendchars}{-\}\enskip}

\newcommand{\visiblecomments}{%
  \let\onelinecomment=\onelinecommentchars
  \let\commentbegin=\commentbeginchars
  \let\commentend=\commentendchars}

\newcommand{\invisiblecomments}{%
  \let\onelinecomment=\empty
  \let\commentbegin=\empty
  \let\commentend=\empty}

\visiblecomments

\newlength{\blanklineskip}
\setlength{\blanklineskip}{0.66084ex}

\newcommand{\hsindent}[1]{\quad}% default is fixed indentation
\let\hspre\empty
\let\hspost\empty
\newcommand{\NB}{\textbf{NB}}
\newcommand{\Todo}[1]{$\langle$\textbf{To do:}~#1$\rangle$}

\EndFmtInput
\makeatother
%
%
%
%
%
%
% This package provides two environments suitable to take the place
% of hscode, called "plainhscode" and "arrayhscode". 
%
% The plain environment surrounds each code block by vertical space,
% and it uses \abovedisplayskip and \belowdisplayskip to get spacing
% similar to formulas. Note that if these dimensions are changed,
% the spacing around displayed math formulas changes as well.
% All code is indented using \leftskip.
%
% Changed 19.08.2004 to reflect changes in colorcode. Should work with
% CodeGroup.sty.
%
\ReadOnlyOnce{polycode.fmt}%
\makeatletter

\newcommand{\hsnewpar}[1]%
  {{\parskip=0pt\parindent=0pt\par\vskip #1\noindent}}

% can be used, for instance, to redefine the code size, by setting the
% command to \small or something alike
\newcommand{\hscodestyle}{}

% The command \sethscode can be used to switch the code formatting
% behaviour by mapping the hscode environment in the subst directive
% to a new LaTeX environment.

\newcommand{\sethscode}[1]%
  {\expandafter\let\expandafter\hscode\csname #1\endcsname
   \expandafter\let\expandafter\endhscode\csname end#1\endcsname}

% "compatibility" mode restores the non-polycode.fmt layout.

\newenvironment{compathscode}%
  {\par\noindent
   \advance\leftskip\mathindent
   \hscodestyle
   \let\\=\@normalcr
   \let\hspre\(\let\hspost\)%
   \pboxed}%
  {\endpboxed\)%
   \par\noindent
   \ignorespacesafterend}

\newcommand{\compaths}{\sethscode{compathscode}}

% "plain" mode is the proposed default.
% It should now work with \centering.
% This required some changes. The old version
% is still available for reference as oldplainhscode.

\newenvironment{plainhscode}%
  {\hsnewpar\abovedisplayskip
   \advance\leftskip\mathindent
   \hscodestyle
   \let\hspre\(\let\hspost\)%
   \pboxed}%
  {\endpboxed%
   \hsnewpar\belowdisplayskip
   \ignorespacesafterend}

\newenvironment{oldplainhscode}%
  {\hsnewpar\abovedisplayskip
   \advance\leftskip\mathindent
   \hscodestyle
   \let\\=\@normalcr
   \(\pboxed}%
  {\endpboxed\)%
   \hsnewpar\belowdisplayskip
   \ignorespacesafterend}

% Here, we make plainhscode the default environment.

\newcommand{\plainhs}{\sethscode{plainhscode}}
\newcommand{\oldplainhs}{\sethscode{oldplainhscode}}
\plainhs

% The arrayhscode is like plain, but makes use of polytable's
% parray environment which disallows page breaks in code blocks.

\newenvironment{arrayhscode}%
  {\hsnewpar\abovedisplayskip
   \advance\leftskip\mathindent
   \hscodestyle
   \let\\=\@normalcr
   \(\parray}%
  {\endparray\)%
   \hsnewpar\belowdisplayskip
   \ignorespacesafterend}

\newcommand{\arrayhs}{\sethscode{arrayhscode}}

% The mathhscode environment also makes use of polytable's parray 
% environment. It is supposed to be used only inside math mode 
% (I used it to typeset the type rules in my thesis).

\newenvironment{mathhscode}%
  {\parray}{\endparray}

\newcommand{\mathhs}{\sethscode{mathhscode}}

% texths is similar to mathhs, but works in text mode.

\newenvironment{texthscode}%
  {\(\parray}{\endparray\)}

\newcommand{\texths}{\sethscode{texthscode}}

% The framed environment places code in a framed box.

\def\codeframewidth{\arrayrulewidth}
\RequirePackage{calc}

\newenvironment{framedhscode}%
  {\parskip=\abovedisplayskip\par\noindent
   \hscodestyle
   \arrayrulewidth=\codeframewidth
   \tabular{@{}|p{\linewidth-2\arraycolsep-2\arrayrulewidth-2pt}|@{}}%
   \hline\framedhslinecorrect\\{-1.5ex}%
   \let\endoflinesave=\\
   \let\\=\@normalcr
   \(\pboxed}%
  {\endpboxed\)%
   \framedhslinecorrect\endoflinesave{.5ex}\hline
   \endtabular
   \parskip=\belowdisplayskip\par\noindent
   \ignorespacesafterend}

\newcommand{\framedhslinecorrect}[2]%
  {#1[#2]}

\newcommand{\framedhs}{\sethscode{framedhscode}}

% The inlinehscode environment is an experimental environment
% that can be used to typeset displayed code inline.

\newenvironment{inlinehscode}%
  {\(\def\column##1##2{}%
   \let\>\undefined\let\<\undefined\let\\\undefined
   \newcommand\>[1][]{}\newcommand\<[1][]{}\newcommand\\[1][]{}%
   \def\fromto##1##2##3{##3}%
   \def\nextline{}}{\) }%

\newcommand{\inlinehs}{\sethscode{inlinehscode}}

% The joincode environment is a separate environment that
% can be used to surround and thereby connect multiple code
% blocks.

\newenvironment{joincode}%
  {\let\orighscode=\hscode
   \let\origendhscode=\endhscode
   \def\endhscode{\def\hscode{\endgroup\def\@currenvir{hscode}\\}\begingroup}
   %\let\SaveRestoreHook=\empty
   %\let\ColumnHook=\empty
   %\let\resethooks=\empty
   \orighscode\def\hscode{\endgroup\def\@currenvir{hscode}}}%
  {\origendhscode
   \global\let\hscode=\orighscode
   \global\let\endhscode=\origendhscode}%

\makeatother
\EndFmtInput
%
%
%
% First, let's redefine the forall, and the dot.
%
%
% This is made in such a way that after a forall, the next
% dot will be printed as a period, otherwise the formatting
% of `comp_` is used. By redefining `comp_`, as suitable
% composition operator can be chosen. Similarly, period_
% is used for the period.
%
\ReadOnlyOnce{forall.fmt}%
\makeatletter

% The HaskellResetHook is a list to which things can
% be added that reset the Haskell state to the beginning.
% This is to recover from states where the hacked intelligence
% is not sufficient.

\let\HaskellResetHook\empty
\newcommand*{\AtHaskellReset}[1]{%
  \g@addto@macro\HaskellResetHook{#1}}
\newcommand*{\HaskellReset}{\HaskellResetHook}

\global\let\hsforallread\empty

\newcommand\hsforall{\global\let\hsdot=\hsperiodonce}
\newcommand*\hsperiodonce[2]{#2\global\let\hsdot=\hscompose}
\newcommand*\hscompose[2]{#1}

\AtHaskellReset{\global\let\hsdot=\hscompose}

% In the beginning, we should reset Haskell once.
\HaskellReset

\makeatother
\EndFmtInput


% https://github.com/conal/talk-2015-essence-and-origins-of-frp/blob/master/mine.fmt
% Complexity notation:






% If an argument to a formatting directive starts with let, lhs2TeX likes to
% helpfully prepend a space to the let, even though that's seldom desirable.
% Write lett to prevent that.













































% Hook into forall.fmt:
% Add proper spacing after forall-generated dots.











% We shouldn't use /=, that means not equal (even if it can be overriden)!







% XXX



%  format `stoup` = "\blackdiamond"






% Cancel the effect of \; (that is \thickspace)



% Use as in |vapply vf va (downto n) v|.
% (downto n) is parsed as an application argument, so we must undo the produced
% spacing.

% indexed big-step eval
% without environments
% big-step eval
% change big-step eval








% \, is 3mu, \! is -3mu, so this is almost \!\!.


\def\deriveDefCore{%
\begin{align*}
  \ensuremath{\Derive{\lambda (\Varid{x}\typcolon\sigma)\to \Varid{t}}} &= \ensuremath{\lambda (\Varid{x}\typcolon\sigma)\;(\Varid{dx}\typcolon\Delta \sigma)\to \Derive{\Varid{t}}} \\
  \ensuremath{\Derive{\Varid{s}\;\Varid{t}}} &= \ensuremath{\Derive{\Varid{s}}\;\Varid{t}\;\Derive{\Varid{t}}} \\
  \ensuremath{\Derive{\Varid{x}}} &= \ensuremath{\Varid{dx}} \\
  \ensuremath{\Derive{\Varid{c}}} &= \ensuremath{\DeriveConst{\Varid{c}}}
\end{align*}
}


% Drop unsightly numbers from function names. The ones at the end could be
% formatted as subscripts, but not the ones in the middle.


\chapter{More on our formalization}
\section{Mechanizing plugins modularly and limitations}
\label{sec:modularity-limits}
Next, we discuss our mechanization of language plugins in Agda, and its
limitations. For the concerned reader, we can say these limitations affect
essentially how modular our proofs are, and not their cogency.

In essence, it's not possible to express in Agda the correct interface for
language plugins, so some parts of language plugins can't be modularized as desirable.
Alternatively, we can mechanize any fixed language plugin together with
the main formalization, which is not truly modular, or mechanize a core
formulation parameterized on a language plugin, which that runs into a few
limitations, or encode plugins so they can be modularized and deal with encoding
overhead.

This section requires some Agda knowledge not provided here, but
we hope that readers familiar with both Haskell and Coq will be
able to follow along.

Our mechanization is divided into multiple Agda modules. Most
modules have definitions that depend on language plugins,
directly or indirectly. Those take definitions from language
plugins as \emph{module parameters}.

For instance, STLC object types are formalized through Agda type
\ensuremath{\Conid{Type}}, defined in module \ensuremath{\Conid{\Conid{Parametric}.\Conid{Syntax}.Type}}. The latter is
parameterized over \ensuremath{\Conid{Base}}, the type of base types.

For instance, \ensuremath{\Conid{Base}} can support a base type of integers, and a
base type of bags of elements of type \ensuremath{\iota} (where \ensuremath{\iota\typcolon\Conid{Base}}). Simplifying a few details, our definition is equivalent
to the following Agda code:
\begin{hscode}\SaveRestoreHook
\column{B}{@{}>{\hspre}l<{\hspost}@{}}%
\column{3}{@{}>{\hspre}l<{\hspost}@{}}%
\column{5}{@{}>{\hspre}l<{\hspost}@{}}%
\column{11}{@{}>{\hspre}c<{\hspost}@{}}%
\column{11E}{@{}l@{}}%
\column{12}{@{}>{\hspre}c<{\hspost}@{}}%
\column{12E}{@{}l@{}}%
\column{14}{@{}>{\hspre}l<{\hspost}@{}}%
\column{15}{@{}>{\hspre}l<{\hspost}@{}}%
\column{32}{@{}>{\hspre}l<{\hspost}@{}}%
\column{48}{@{}>{\hspre}l<{\hspost}@{}}%
\column{E}{@{}>{\hspre}l<{\hspost}@{}}%
\>[B]{}\mathbf{module}\;\Conid{\Conid{Parametric}.\Conid{Syntax}.Type}\;(\Conid{Base}\typcolon\Conid{Set})\;\mathbf{where}{}\<[E]%
\\
\>[B]{}\hsindent{3}{}\<[3]%
\>[3]{}\mathbf{data}\;\Conid{Type}\typcolon\Conid{Set}\;\mathbf{where}{}\<[E]%
\\
\>[3]{}\hsindent{2}{}\<[5]%
\>[5]{}\Varid{base}{}\<[11]%
\>[11]{}\typcolon{}\<[11E]%
\>[14]{}\;(\iota\typcolon\Conid{Base}){}\<[48]%
\>[48]{}\to \Conid{Type}{}\<[E]%
\\
\>[3]{}\hsindent{2}{}\<[5]%
\>[5]{}\text{\textunderscore}\mathord{\Rightarrow}\text{\textunderscore}{}\<[11]%
\>[11]{}\typcolon{}\<[11E]%
\>[14]{}\;(\sigma\typcolon\Conid{Type}){}\<[32]%
\>[32]{}\to (\tau\typcolon\Conid{Type})\to \Conid{Type}{}\<[E]%
\\[\blanklineskip]%
\>[B]{}\mbox{\onelinecomment  Elsewhere, in plugin:}{}\<[E]%
\\[\blanklineskip]%
\>[B]{}\mathbf{data}\;\Conid{Base}\typcolon\Conid{Set}\;\mathbf{where}{}\<[E]%
\\
\>[B]{}\hsindent{3}{}\<[3]%
\>[3]{}\Varid{baseInt}{}\<[12]%
\>[12]{}\typcolon{}\<[12E]%
\>[15]{}\;\Conid{Base}{}\<[E]%
\\
\>[B]{}\hsindent{3}{}\<[3]%
\>[3]{}\Varid{baseBag}{}\<[12]%
\>[12]{}\typcolon{}\<[12E]%
\>[15]{}\;(\iota\typcolon\Conid{Base})\to \Conid{Base}{}\<[E]%
\\
\>[B]{}\mbox{\onelinecomment  ...}{}\<[E]%
\ColumnHook
\end{hscode}\resethooks

But with these definitions, we only have bags of elements of base type. If
\ensuremath{\iota} is a base type, \ensuremath{\Varid{base}\;(\Varid{baseBag}\;\iota)} is the type of bags with elements
of type \ensuremath{\Varid{base}\;\iota}. Hence, we have bags of elements of base type. But we don't
have a way to encode \ensuremath{\Conid{Bag}\;\tau} if \ensuremath{\tau} is an arbitrary non-base type, such as
\ensuremath{\Varid{base}\;\Varid{baseInt}\Rightarrow\Varid{base}\;\Varid{baseInt}} (the encoding of object type \ensuremath{\mathbb{Z}\to \mathbb{Z}}).
%
Can we do better? If we ignore modularization, we can define types through the
following mutually recursive datatypes:

\begin{hscode}\SaveRestoreHook
\column{B}{@{}>{\hspre}l<{\hspost}@{}}%
\column{3}{@{}>{\hspre}l<{\hspost}@{}}%
\column{5}{@{}>{\hspre}l<{\hspost}@{}}%
\column{11}{@{}>{\hspre}c<{\hspost}@{}}%
\column{11E}{@{}l@{}}%
\column{14}{@{}>{\hspre}l<{\hspost}@{}}%
\column{17}{@{}>{\hspre}l<{\hspost}@{}}%
\column{32}{@{}>{\hspre}l<{\hspost}@{}}%
\column{48}{@{}>{\hspre}l<{\hspost}@{}}%
\column{E}{@{}>{\hspre}l<{\hspost}@{}}%
\>[B]{}\textbf{mutual}{}\<[E]%
\\
\>[B]{}\hsindent{3}{}\<[3]%
\>[3]{}\mathbf{data}\;\Conid{Type}\typcolon\Conid{Set}\;\mathbf{where}{}\<[E]%
\\
\>[3]{}\hsindent{2}{}\<[5]%
\>[5]{}\Varid{base}{}\<[11]%
\>[11]{}\typcolon{}\<[11E]%
\>[14]{}\;(\iota\typcolon\Conid{Base}\;\Conid{Type}){}\<[48]%
\>[48]{}\to \Conid{Type}{}\<[E]%
\\
\>[3]{}\hsindent{2}{}\<[5]%
\>[5]{}\text{\textunderscore}\mathord{\Rightarrow}\text{\textunderscore}{}\<[11]%
\>[11]{}\typcolon{}\<[11E]%
\>[14]{}\;(\sigma\typcolon\Conid{Type}){}\<[32]%
\>[32]{}\to (\tau\typcolon\Conid{Type})\to \Conid{Type}{}\<[E]%
\\[\blanklineskip]%
\>[B]{}\hsindent{3}{}\<[3]%
\>[3]{}\mathbf{data}\;\Conid{Base}\typcolon\Conid{Set}\;\mathbf{where}{}\<[E]%
\\
\>[3]{}\hsindent{2}{}\<[5]%
\>[5]{}\Varid{baseInt}{}\<[14]%
\>[14]{}\typcolon{}\<[17]%
\>[17]{}\;\Conid{Base}{}\<[E]%
\\
\>[3]{}\hsindent{2}{}\<[5]%
\>[5]{}\Varid{baseBag}{}\<[14]%
\>[14]{}\typcolon{}\<[17]%
\>[17]{}\;(\iota\typcolon\Conid{Type})\to \Conid{Base}{}\<[E]%
\ColumnHook
\end{hscode}\resethooks

So far so good, but these types have to be defined together. We
can go a step further by defining:

\begin{hscode}\SaveRestoreHook
\column{B}{@{}>{\hspre}l<{\hspost}@{}}%
\column{3}{@{}>{\hspre}l<{\hspost}@{}}%
\column{5}{@{}>{\hspre}l<{\hspost}@{}}%
\column{11}{@{}>{\hspre}c<{\hspost}@{}}%
\column{11E}{@{}l@{}}%
\column{14}{@{}>{\hspre}l<{\hspost}@{}}%
\column{17}{@{}>{\hspre}l<{\hspost}@{}}%
\column{32}{@{}>{\hspre}l<{\hspost}@{}}%
\column{48}{@{}>{\hspre}l<{\hspost}@{}}%
\column{E}{@{}>{\hspre}l<{\hspost}@{}}%
\>[B]{}\textbf{mutual}{}\<[E]%
\\
\>[B]{}\hsindent{3}{}\<[3]%
\>[3]{}\mathbf{data}\;\Conid{Type}\typcolon\Conid{Set}\;\mathbf{where}{}\<[E]%
\\
\>[3]{}\hsindent{2}{}\<[5]%
\>[5]{}\Varid{base}{}\<[11]%
\>[11]{}\typcolon{}\<[11E]%
\>[14]{}\;(\iota\typcolon\Conid{Base}\;\Conid{Type}){}\<[48]%
\>[48]{}\to \Conid{Type}{}\<[E]%
\\
\>[3]{}\hsindent{2}{}\<[5]%
\>[5]{}\text{\textunderscore}\mathord{\Rightarrow}\text{\textunderscore}{}\<[11]%
\>[11]{}\typcolon{}\<[11E]%
\>[14]{}\;(\sigma\typcolon\Conid{Type}){}\<[32]%
\>[32]{}\to (\tau\typcolon\Conid{Type})\to \Conid{Type}{}\<[E]%
\\[\blanklineskip]%
\>[B]{}\hsindent{3}{}\<[3]%
\>[3]{}\mathbf{data}\;\Conid{Base}\;(\Conid{Type}\typcolon\Conid{Set})\typcolon\Conid{Set}\;\mathbf{where}{}\<[E]%
\\
\>[3]{}\hsindent{2}{}\<[5]%
\>[5]{}\Varid{baseInt}{}\<[14]%
\>[14]{}\typcolon{}\<[17]%
\>[17]{}\;\Conid{Base}\;\Conid{Type}{}\<[E]%
\\
\>[3]{}\hsindent{2}{}\<[5]%
\>[5]{}\Varid{baseBag}{}\<[14]%
\>[14]{}\typcolon{}\<[17]%
\>[17]{}\;(\iota\typcolon\Conid{Type})\to \Conid{Base}\;\Conid{Type}{}\<[E]%
\ColumnHook
\end{hscode}\resethooks

Here, \ensuremath{\Conid{Base}} takes the type of object types as a parameter, and
\ensuremath{\Conid{Type}} uses \ensuremath{\Conid{Base}\;\Conid{Type}} to tie the recursion. This definition
still works, but only as long as \ensuremath{\Conid{Base}} and \ensuremath{\Conid{Type}} are defined together.

If we try to separate the definitions of \ensuremath{\Conid{Base}} and \ensuremath{\Conid{Type}} into
different modules, we run into trouble.
\begin{hscode}\SaveRestoreHook
\column{B}{@{}>{\hspre}l<{\hspost}@{}}%
\column{3}{@{}>{\hspre}l<{\hspost}@{}}%
\column{5}{@{}>{\hspre}l<{\hspost}@{}}%
\column{11}{@{}>{\hspre}c<{\hspost}@{}}%
\column{11E}{@{}l@{}}%
\column{12}{@{}>{\hspre}c<{\hspost}@{}}%
\column{12E}{@{}l@{}}%
\column{14}{@{}>{\hspre}l<{\hspost}@{}}%
\column{15}{@{}>{\hspre}l<{\hspost}@{}}%
\column{32}{@{}>{\hspre}l<{\hspost}@{}}%
\column{48}{@{}>{\hspre}l<{\hspost}@{}}%
\column{E}{@{}>{\hspre}l<{\hspost}@{}}%
\>[B]{}\mathbf{module}\;\Conid{\Conid{Parametric}.\Conid{Syntax}.Type}\;(\Conid{Base}\typcolon\Conid{Set}\to \Conid{Set})\;\mathbf{where}{}\<[E]%
\\
\>[B]{}\hsindent{3}{}\<[3]%
\>[3]{}\mathbf{data}\;\Conid{Type}\typcolon\Conid{Set}\;\mathbf{where}{}\<[E]%
\\
\>[3]{}\hsindent{2}{}\<[5]%
\>[5]{}\Varid{base}{}\<[11]%
\>[11]{}\typcolon{}\<[11E]%
\>[14]{}\;(\iota\typcolon\Conid{Base}\;\Conid{Type}){}\<[48]%
\>[48]{}\to \Conid{Type}{}\<[E]%
\\
\>[3]{}\hsindent{2}{}\<[5]%
\>[5]{}\text{\textunderscore}\mathord{\Rightarrow}\text{\textunderscore}{}\<[11]%
\>[11]{}\typcolon{}\<[11E]%
\>[14]{}\;(\sigma\typcolon\Conid{Type}){}\<[32]%
\>[32]{}\to (\tau\typcolon\Conid{Type})\to \Conid{Type}{}\<[E]%
\\[\blanklineskip]%
\>[B]{}\mbox{\onelinecomment  Elsewhere, in plugin:}{}\<[E]%
\\[\blanklineskip]%
\>[B]{}\mathbf{data}\;\Conid{Base}\;(\Conid{Type}\typcolon\Conid{Set})\typcolon\Conid{Set}\;\mathbf{where}{}\<[E]%
\\
\>[B]{}\hsindent{3}{}\<[3]%
\>[3]{}\Varid{baseInt}{}\<[12]%
\>[12]{}\typcolon{}\<[12E]%
\>[15]{}\;\Conid{Base}\;\Conid{Type}{}\<[E]%
\\
\>[B]{}\hsindent{3}{}\<[3]%
\>[3]{}\Varid{baseBag}{}\<[12]%
\>[12]{}\typcolon{}\<[12E]%
\>[15]{}\;(\iota\typcolon\Conid{Type})\to \Conid{Base}\;\Conid{Type}{}\<[E]%
\ColumnHook
\end{hscode}\resethooks

Here, \ensuremath{\Conid{Type}} is defined for an arbitrary function on types \ensuremath{\Conid{Base}\typcolon\Conid{Set}\to \Conid{Set}}. However, this definition is rejected by Agda's
\emph{positivity checker}. Like Coq, Agda forbids defining
datatypes that are not strictly positive, as they can introduce
inconsistencies.
\pg{Add the ``Bad'' example from Agda's wiki, with link for credit.}

The above definition of \ensuremath{\Conid{Type}} is \emph{not} strictly positive,
because we could pass to it as argument \ensuremath{\Conid{Base}\mathrel{=}\lambda \tau\to (\tau\to \tau)} so that \ensuremath{\Conid{Base}\;\Conid{Type}\mathrel{=}\Conid{Type}\to \Conid{Type}}, making \ensuremath{\Conid{Type}} occur in
a negative position. However, the actual uses of \ensuremath{\Conid{Base}} we are
interested in are fine. The problem is that we cannot inform the
positivity checker that \ensuremath{\Conid{Base}} is supposed to be a strictly
positive type function, because Agda doesn't supply the needed
expressivity.

This problem is well-known. It could be solved if Agda function spaces supported
positivity annotations,\footnote{As discussed in
  \url{https://github.com/agda/agda/issues/570} and
  \url{https://github.com/agda/agda/issues/2411}.} or by encoding a universe of
strictly-positive type constructors. This encoding is not fully transparent and
adds hard-to-hide noise to
development~\citep{Schwaab2013modular}.\footnote{Using pattern synonyms and
  \texttt{DISPLAY} pragmas might successfully hide this noise.}
Few alternatives remain:
\begin{itemize}
\item We can forbid types from occurring in base types, as we did in our
  original paper~\citep{CaiEtAl2014ILC}. There we did not discuss at all a
  recursive definition of base types.
\item We can use the modular mechanization, disable positivity checking and risk
  introducing inconsistencies. We tried this successfully in a branch of that
  formalization. We believe we did not introduce inconsistencies in that branch
  but have no hard evidence.
\item We can simply combine the core formalization with a sample plugin. This is
  not truly modular because the core modularization can only be reused by
  copy-and-paste. Moreover, in dependently-typed languages the content of a
  definition can affect whether later definitions typecheck, so alternative
  plugins using the same interface might not typecheck.\footnote{Indeed, if
    \ensuremath{\Conid{Base}} were not strictly positive, the application \ensuremath{\Conid{Type}\;\Conid{Base}} would be
    ill-typed as it would fail positivity checking, even though \ensuremath{\Conid{Base}\typcolon\Conid{Set}\to \Conid{Set}} does not require \ensuremath{\Conid{Base}} to be strictly positive.}
\end{itemize}

Sadly, giving up on modularity altogether appears the more conventional choice.
Either way, as we claimed at the outset, these modularity concerns only limit
the modularity of the mechanized proofs, not their cogency.
