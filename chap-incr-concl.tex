% Emacs, this is -*- latex -*-!

% \pg{Plan for things that complete the original paper's story: add them by
%   revising that text and in that chapter.}
\chapter{Conclusion and future work}
\label{ch:incr-conclusion}
In databases, standard finite differencing technology allows incrementalizing
programs in a specific domain-specific first-order language of collection
operations.
Unlike other approaches, finite differencing transforms programs to programs,
which can in turn be transformed further.

Differentiation (\cref{sec:intro}) transforms higher-order programs to incremental ones called \emph{derivatives}, as long as
one can provide incremental support for primitive types and operations.

Case studies in this thesis consider support for several such primitives. We first
study a setting restricted to \emph{self-maintainable} derivatives
(\cref{sec:applying}). Then we study a more general setting where it is possible
to remember inputs and intermediate results from one execution to the next,
thanks to a transformation to \emph{cache-transfer style} (\cref{ch:cts}). In all cases, we are
able to deliver order-of-magnitude speedups on non-trivial examples; moreover,
incrementalization produces programs in standard languages that are subject to
further optimization and code transformation.

Correctness of incrementalization appeared initially surprising, so we devoted
significant effort to formal correctness proofs, either on paper or in
mechanized proof assistants.
The original correctness proof of differentiation, using a set-theoretic
denotational semantics~\citep{CaiEtAl2014ILC}, was a significant milestone, but
since then we have
simplified the proof to a logical relations proof that defines when a change is
valid from a source to a destination, and proves that differentiation produces
valid changes (\cref{ch:derive-formally}). Valid changes witness the difference between
sources and destinations; since changes can be \emph{nil}, change validity
arises as a generalization of the concept of \emph{logical equivalence} and
parametricity for a language (at least in terminating languages)
(\cref{ch:diff-parametricity-system-f}). Crucially,
changes are not just witnesses: operation $\oplus$ takes a change and its
source to the change destination. One can consider further operations, that give
rise to an algebraic structure that we call \emph{change structure} (\cref{ch:change-theory}).

Based on this simplified proof, in this thesis we generalize correctness to
further languages
using big-step operational semantics and (step-indexed) logical relations (\cref{ch:bsos}).
Using operational semantics we reprove correctness for simply-typed
$\lambda$-calculus, then add support for recursive functions (which would
require domain-theoretic methods when using denotational semantics), and finally
extend the proof to untyped $\lambda$-calculus. Building on a variant of this
proof (\cref{sec:sound-derive}), we show that conversion to cache-transfer-style
also preserves correctness.

Based on a different formalism for logical equivalence and
parametricity~\citep{Bernardy2011realizability}, we sketch variants of the
transformation and correctness proofs for simply-typed $\lambda$-calculus with
type variables and then for full System F (\cref{ch:diff-parametricity-system-f}).

\paragraph{Future work}
\label{ch:incr-conclusion-futwork}
To extend differentiation to System F, we must consider changes where source and
destination have different types. This generalization of changes makes change
operations much more flexible: it becomes possible to define a combinator
language for change structures, and it appears possible to define nontrivial
change structures for algebraic datatypes using this combinator language.

We conjecture such a combinator language will allow programmers to define
correct change structures out of simple, reusable components.

Incrementalizing primitives correctly remains at present a significant
challenge. We provide tools to support this task by formalizing equational
reasoning, but it appears necessary to provide more tools to programmers, as
done by \citet{Liu00}. Conceivably, it might be possible to build such a
rewriting tool on top of the rewriting and automation support of theorem provers
such as Coq.

% Incrementalization b
% In this part, we have discussed incrementalization by static

% \section{Primitives}
% \label{sec:fw-primitives}