% Thanks to \frontmatter, this is unnumbered.
% https://tex.stackexchange.com/a/244746/1340
\chapter{Abstract}

In modern programming languages, queries on in-memory collections are often more
expensive than needed.
While database queries can be readily optimized, collection queries often don't
translate trivially to databases, as modern programming languages are far more
expressive than databases, supporting both nested data and first-class functions.

Collection queries can be optimized and incrementalized by hand, but this
reduces modularity, and is often too error-prone to be feasible or to enable
maintenance of resulting programs.
In this thesis, we optimize and incrementalize such collection queries to free
programmers from such burdens.
Resulting programs are expressed in the same core language, so that they can be
subjected to other standard optimizations.

To enable optimizing collection queries which occur inside programs, we develop a
staged variant of the Scala collection API that \emph{reifies} queries as ASTs.
On top of this interface, we adapt domain-specific optimizations from the fields
of programming languages and databases; among others, we rewrite queries to use
indexes chosen by programmers. Thanks to the use of indexes we show significant
speedups in our experimental evaluation, with an average of 12x and a maximum of
12800x.

To incrementalize higher-order programs by program transformation, we extend
\emph{finite differencing} \citep{Paige82FDC,Blakeley:1986:EUM,Gupta99MMV} and
develop the first approach to incrementalization by program transformation for
higher-order programs. Base programs are transformed to \emph{derivatives},
programs that transform input \emph{changes} to output changes.
We prove that our incrementalization approach is correct: We develop the theory
underlying this approach for simply-typed and untyped $\lambda$-calculus, and
discuss extensions to System F\@.

Derivatives often need to reuse results produced by base programs: to enable
such reuse, we extend work by \citet{Liu95} to higher-order programs, and
develop and prove correct a program transformation, converting higher-order
programs to \emph{cache-transfer-style}.

For efficient incrementalization, it is necessary to choose and incrementalize
by hand appropriate primitive operations. We incrementalize a significant subset
of collection operations and perform case studies, showing order-of-magnitude speedups
both in practice and in asymptotic complexity.
% To incrementalize programs efficiently, derivatives
% To use our approach a certain domain, designers must provide
% relies on efficient primitives

% \begin{abstract}
% \end{abstract}