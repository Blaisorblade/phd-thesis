

%selecting queries
%\begin{table*}
%\begin{tabular}{ll}\toprule
%Identifier & Description \\ \midrule
%PROTECTED\_FIELD % Findbugs: CI\_CONFUSED\_INHERITANCE
%%	& % Idealized: 4
%	& Class is final but declares protected field \\
%NO\_CLONE % Findbugs: CN\_IDOM
%%	& % Idealized: 9
%	&  Class implements Cloneable but does not define or use clone method \\
%SUPER\_CLONE\_MISSING % Findbugs: CN\_IDIOM\_NO\_SUPER\_CALL
%%	& % Idealized: 11
%	& The clone method does not call super.clone() \\
%NOT\_CLONEABLE % Findbugs: CN\_IMPLEMENTS\_CLONE\_BUT\_NOT\_CLONEABLE
%%	& % Idealized: 5
%	& Class defines clone() but doesn't implement Cloneable\\
%COVARIANT\_COMPARETO % Findbugs CO\_ABSTRACT\_SELF \& CO\_SELF\_NO\_OBJECT 
%%	& % Idealized: 7
%	& Covariant compareTo() method defined\\
%GC\_CALL % Findbugs:  DM\_GC
%%	& % Idealized: 12
%	& Explicit garbage collection; extremely dubious except in benchmarking code\\
%RUN\_FINALIZERS\_ON\_EXIT % Findbugs DM\_RUN\_FINALIZERS\_ON\_EXIT
%%	& % Idealized: 12
%	& Method invokes dangerous method runFinalizersOnExit\\
%COVARIANT\_EQUALS %Findbugs: EQ\_ABSTRACT\_SELF 
%%	& % Idealized: 4
%	& Abstract class defines covariant equals() method \\
%FINALIZER\_NOT\_PROTECTED % Findbugs: FI_PUBLIC_SHOULD_BE_PROTECTED
%%	& % Idealized: 6
%	& Finalizer should be protected, not public\\
%%NO\_SUITABLE\_CONSTRUCTOR% Findbugs: SE\_NO\_SUITABLE\_CONSTRUCTOR 
%%	& % Idealized: 7
%%	& Class is Serializable but its superclass doesn't define a void constructor\\
%UNUSED\_PRIVATE\_FIELD % Findbugs: UUF\_UNUSED\_FIELD 
%%	& % Idealized: 23
%	& The value of a private field is not read\\
%DONT\_CATCH\_IMSE  %Findbugs: IMSE\_DONT\_CATCH\_IMSE 
%%	& % Idealized: 5
%	& Dubious catching of IllegalMonitorStateException \\\bottomrule
%\end{tabular}
%\nocaptionrule\caption{Implemented Analyses}
%\label{table:implemented-analyses}
%\end{table*}


\newcommand{\captionEvalTable}{%
As in in \cref{sec:implemenationsandspeedups}, (1) denotes the modular Scala
implementation, (2) the hand-optimized Scala one, and ($3^-$), ($3^o$), ($3^x$)
refer to the {\LoS} implementation when run, respectively, without
optimizations, with optimizations, with optimizations and indexing.
Queries marked with the $R$ superscript were selected by random sampling.}
\newcommand{\tablerowsize}{\scriptsize}
\begin{sidewaystable}[ph!]
\centering
\input{\graphPath{EvalTable}}
\nocaptionrule\caption{Performance results. \captionEvalTable}
\label{table:performance}
\end{sidewaystable}



\begin{table}[h]
  \centering
  \footnotesize
\input{\graphPath{EvalSummaryTable}}
\nocaptionrule\caption{Average performance ratios.
This table summarizes all interesting performance ratios across all queries,
using the geometric mean~\citep{Fleming86}.
The meaning of speedups is discussed in \cref{sec:implemenationsandspeedups}.}
\label{table:performanceAvg}
\end{table}

\begin{table}[h!]
\begin{tabular}{p{7cm}r}\toprule
Abstraction & Used \\ \midrule
% calculate class hierarchy & 5 \\
All fields in all class files	& 4\\
All methods in all class files	& 3\\
All method bodies in all class files	& 3\\
All instructions in all method bodies and their bytecode index	& 5\\
Sliding window (size $n$) over all instructions (and their index) &	3\\
\bottomrule
\end{tabular}\\
\nocaptionrule\caption{Description of abstractions removed during hand-optimization and number of queries where the abstraction is used (and optimized away).}
\label{table:implemented-abstractions}
\end{table}

\begin{extraEval}
\begin{table*}[tb]
\centering
\begin{tabular}{l*{3}{r@{}c@{}l}r*{2}{r@{}c@{}l}}\toprule
Name&\multicolumn{3}{c}{Base impl.\ (in ms)}&\multicolumn{3}{c}{Modular impl}&\multicolumn{3}{c}{Optimiz.\ time}&IS&\multicolumn{3}{c}{OS}&\multicolumn{3}{c}{OS-Opt}\\\midrule
\input{\graphPath{table}}
\bottomrule
\end{tabular}
\nocaptionrule\caption{Old performance results table}
\label{table:performanceOld}
\end{table*}
\end{extraEval}
