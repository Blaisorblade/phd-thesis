While the basic idea of deep embedding is well known, it is not obvious how to realize deep embedding
when considering the following additional goals:

\begin{itemize}
\item To support users adopting \LoS, a generic \LoS\ query should share the ``look and feel'' of the ordinary collections API: In particular, query syntax should remain mostly unchanged. In our case, we want to preserve  Scala's \emph{for-comprehension}\footnote{Also known as \emph{for expressions}~\citep[Ch.~23]{Odersky11book}.} syntax and its notation
 for anonymous functions.
 %IDE services for native Scala queries (such as syntax highlighting or
 %reference resolving) should also work with \LoS\ queries.
 % Paolo: That seems irrelevant, this is guaranteed by all deep embeddings.
 \item Again to support users adopting \LoS, a generic \LoS\ query should not only share the syntax of the ordinary collections API; it should also be well-typed if and only if the corresponding ordinary query is well-typed.
 This is particularly challenging in the Scala collections
 library due to its deep integration with advanced type-system features, such as higher-kinded generics
 and implicit objects~\citep{odersky2009fighting}. For instance, calling \code{map} on a \code{List} will return a \code{List}, and calling \code{map} on a \code{Set}
 will return a \code{Set}. Hence the object-language representation and the
 transformations thereof should be as ``typed'' as possible. This precludes, among others, a first-order
 representation of object-language variables as strings.
 \item \LoS\ should be interoperable with ordinary Scala code and Scala collections. For instance, it should
 be possible to call normal non-reified functions within a \LoS\ query, or mix native Scala queries and \LoS\ queries.
%KO: commented this out because we now have the RW section early
%\item \LINQ\ also reifies queries by relying on built-in support for deep embedding (in the form of expression trees); we want instead to express deep embedding as a library using general-purpose language constructs of more general applicability.
 \item The performance of \LoS\ queries should be reasonable even without optimizations. A non-optimized \LoS\ query
 should not be dramatically slower than a native Scala query. Furthermore, it should be possible to create new queries at run time and execute them without excessive overhead.
 This goal limits the options of applicable interpretation or compilation techniques.
\end{itemize}
\chk{these goals are not really justified, but that may be too late to change}
 
We think that these challenges characterize deep embedding of queries on collections as a \emph{critical} case study~\citep{flyvbjerg06five} for DSL embedding.
That is, it is so challenging that embedding techniques successfully used in this case are likely to be successful on a broad range of other DSLs.
From the case study, we report the successes and failures of achieving these goals in \LoS. 
