Programmers
are often concerned with domain-specific optimizations which distract them from the problem at hand and should instead be automated.
Consider for instance the following piece of Scala code, which maps two functions over a collection of integers \code{v}:

\begin{lstlisting}
v.map(x => x + 1).map(x => x * 3)
\end{lstlisting}

Exploiting algebraic identities, we can optimize such code using collections.
For instance, the two map operations above can be fused into a single one to avoid constructing an intermediate collection:

\begin{lstlisting}
v.map(x => (x + 1) * 3)
\end{lstlisting}

This optimization is specific to the domain of collections. In general, domain-specific optimizations are typically not performed by compilers or virtual machines, because they require domain knowledge
about the behavior and invariants of the code, something which is beyond the capabilities of general-purpose static analysis. For instance, the 
optimization above is only behavior-preserving, if the functions mapped over the collection do not perform order-dependent side effects,
%\pg{TR: static analysis could allow optimizing this example. Also discuss map fusion in related work.}
which is complex to determine in general. Namely, verifying that a function is pure requires verifying that all the invoked functions are also pure, but already understanding which functions are invoked can be difficult, especially when late-bound function calls are involved.
% I wrote 'late-bound function calls' instead of 'virtual method calls' to include the same problem in functional programs.

Nowadays, mostly programmers have to perform domain-specific optimizations by hand. In our example, this does not seem too bad; but 
when considering further optimizations, such as indexing, reordering and fusion of filters, and computing hash joins
instead of Cartesian products, then it becomes clear that hand-coding such optimizations can lead to complicated
and redundant code, because the same optimization patterns are applied again and again. Hand-coding such optimizations also tends to lead to brittle and \emph{non-modular} code. For instance, consider a program which 
computes \code{w = v.map(x => x + 1)}, only to pass \code{w} to another function which computes
\code{z = w.map(x => x * 3)}. If \code{w} is not used further, it gets built just to be transformed again. Again, we would like to apply map fusion, but this time the code crosses modularity boundaries. Hence, in a modular structure, a developer might not even see optimization potential.
In contrast, an automated domain-specific optimizer can inline the function to fuse code from different modules together
 and spot cross-module optimization opportunities.
%\pg{TR: Maybe connect to inlining?}
%\pg{Have a forward reference, in the paper summary, with a more detailed explanation of the problem.}

Such optimizations are not possible within most ordinary collections API, such as the Scala collections API~\cite{odersky2009fighting}, because they
do not reify queries on collections. Instead, each API function is defined compositionally in terms
of the meaning of the other API functions.
We regard the Scala collections API as a \emph{embedded domain-specific language} (EDSL); this style of compositional DSL implementation by libraries is often
called \emph{shallow embedding}%
%~\chk{add reference} %PG: I know it's crazy, but nobody gives a reference for this term.
. The alternative, which we pursue, is a \emph{deep embedding} of the DSL, where programs written in the DSL (in our case: 
queries on collections) are reified as data or, more specifically, as abstract syntax trees, which can be subject to non-compositional semantics. Among others, they can then be analyzed, optimized, transformed, interpreted, or compiled.

In this paper we present \LoS, a deep embedding of the Scala collections EDSL with the goal of supporting more flexible semantics and especially automatic optimization.
We chose to use the Scala language and its collections EDSL because (i) Scala has several features that make it an attractive host language for DSL embedding \citep{Odersky11book} and (ii) because it has a particularly sophisticated collections library with precisely-typed interfaces \citep{odersky2009fighting}, which makes it particularly challenging to embed the library deeply.

%We target Scala collections EDSL because improved support for DSL embedding is among Scala design goals.

While the basic idea of deep embedding is well known, it is not obvious how to realize deep embedding
when considering the following additional goals:

\begin{itemize}
\item To support users adopting \LoS, a generic \LoS\ query should share the ``look and feel'' of the ordinary collections API: In particular, query syntax should remain mostly unchanged. In our case, we want to preserve  Scala's \emph{for-comprehension}\footnote{Also known as \emph{for expressions}~\citep[Ch.~23]{Odersky11book}.} syntax and its notation
 for anonymous functions.
 %IDE services for native Scala queries (such as syntax highlighting or
 %reference resolving) should also work with \LoS\ queries.
 % Paolo: That seems irrelevant, this is guaranteed by all deep embeddings.
 \item Again to support users adopting \LoS, a generic \LoS\ query should not only share the syntax of the ordinary collections API; it should also be well-typed if and only if the corresponding ordinary query is well-typed.
 This is particularly challenging in the Scala collections
 library due to its deep integration with advanced type-system features, such as higher-kinded generics
 and implicit objects~\citep{odersky2009fighting}. For instance, calling \code{map} on a \code{List} will return a \code{List}, and calling \code{map} on a \code{Set}
 will return a \code{Set}. Hence the object-language representation and the
 transformations thereof should be as ``typed'' as possible. This precludes, among others, a first-order
 representation of object-language variables as strings.
 \item \LoS\ should be interoperable with ordinary Scala code and Scala collections. For instance, it should
 be possible to call normal non-reified functions within a \LoS\ query, or mix native Scala queries and \LoS\ queries.
%KO: commented this out because we now have the RW section early
%\item \LINQ\ also reifies queries by relying on built-in support for deep embedding (in the form of expression trees); we want instead to express deep embedding as a library using general-purpose language constructs of more general applicability.
 \item The performance of \LoS\ queries should be reasonable even without optimizations. A non-optimized \LoS\ query
 should not be dramatically slower than a native Scala query. Furthermore, it should be possible to create new queries at run time and execute them without excessive overhead.
 This goal limits the options of applicable interpretation or compilation techniques.
\end{itemize}
\chk{these goals are not really justified, but that may be too late to change}
 
We think that these challenges characterize deep embedding of queries on collections as a \emph{critical} case study~\citep{flyvbjerg06five} for DSL embedding.
That is, it is so challenging that embedding techniques successfully used in this case are likely to be successful on a broad range of other DSLs.
From the case study, we report the successes and failures of achieving these goals in \LoS. 


Overall, we make the following contributions:


\begin{itemize}
\item We illustrate modularity limitations when manipulating collections, caused by the lack of domain-specific optimizations. Conversely, we illustrate how domain-specific optimizations lead to more readable and more modular code~(\cref{sec:motivation}).
\item We present the design~(\cref{sec:solution}) and implementation~(\cref{sec:implementation}) of \LoS, an embedded DSL for queries on collections in Scala.
\item We discuss how Scala supports expressing the interface~(Section~\ref{sec:intfScala}).
%\LoS\ makes heavy usage of advanced Scala features to preserve the ``look and feel'' of native Scala queries
% and is hence a significant showcase for these features~(\cref{sec:implementation}).
 \item We evaluate \LoS\ by re-implementing several code analyses of the Findbugs tool~\citep{DBLP:journals/sigplan/HovemeyerP04}.
 The resulting code is shorter and/or more efficient, in some cases by orders of magnitude~(\cref{sec:evaluation}).
 \item We discuss the degree to which \LoS\ fulfills our original design goal and extract lessons for
 both future DSL developers and general-purpose language designers. %\chk{none of these abbreviations was introducted.} %DSL was introduced, I avoid GPL
% In particular, we propose areas in which the support of Scala
% for embedding DSLs should be improved~(\cref{sec:discussion}).
\end{itemize}

% PG: I integrated the outline into the contribution list, I like it better.
%The remainder of this work is structured as follows. First, we discuss related work. 
%In \cref{sec:intro} we give an informal overview of \LoS. \Cref{sec:implementation} describes the implementation of \LoS. \Cref{sec:evaluation} presents the result of our evaluation. \cref{sec:discussion} reflects on the results and discusses implications for object- and meta-language design. \Cref{sec:concl} concludes.

%%%%%%%%
% %Keep the intro within the first page, or however short.
% %
% %Detail more about DSLs in general later, not in the intro.
% %
% %1 parag. DSL + embedding are important to reduce the abstraction gap and …. (citations). (Domain-level developers are mentioned up to here.)
% %2 parag. ds-optims are important
% %3. => deep embedding  (multiple semantics as a side point). Scala is tailored for this even for deep embedding (lots of citations).
% %4. We want to do a case study embedding collections, because it's critical case study, challenges developers (?) and tools (mostly languages), and can hugely benefit from optimizations.
% %(I don't talk so much any more about developers; this is a challenge for the DSL developer anyway).
% %(Only later mention error reporting and the domain-level developer (ie. user))
% %5. = Sec. 1.1 (intro to the case study): shorten. Our goal is to embed a Scala-like collection API similarly to \LINQ\ without [hacking the compiler] and optimizing it.
% %% This might be enough to discuss \LINQ\ only at the end.
% %6. Mention FindBugs as evaluation goal.
% %7. We did it and we report novel encoding, experience on this (anticipate conclusions shortly). (Some performance numbers would also be good). "In selected cases, we yield a speedup of …". (I can be more honest here but shortly).
% %8. list of contributions
% %No outline, unless it fits within the contributions.
% % Note: here or later, we could distinguish between domain-specific compiler extensions and more general ones.

% % Good paragraph

% One of the design goals for this EDSL is that embedded queries should be
% as close as possible to their native equivalent, to simplify
% transforming a native query into an embedded one. Ideally, querying a
% ``smart'' collection instead of a native one should be enough to ensure
% to produce a representation of the query instead of executing it;
% however, a query representation needs to be then converted explicitly to
% its result. However, Scala does not support natively expression trees,
% which are used in C\# to achieve this goal. We instead provide an
% emulation for them, based on user-defined implicit conversions, which we
% later describe. The surface syntax is quite close to Scala in many
% regards, but some areas show significant limitations; in other cases,
% providing a usable surface syntax is made more complex by limitations of
% Scala's type inference; one of our goals is to not require the user to
% supply more type annotations than it would need if expression trees were
% natively supported.
