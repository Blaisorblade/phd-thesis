% Emacs, this is -*- latex -*-!
%% ODER: format ==         = "\mathrel{==}"
%% ODER: format /=         = "\neq "
%
%
\makeatletter
\@ifundefined{lhs2tex.lhs2tex.sty.read}%
  {\@namedef{lhs2tex.lhs2tex.sty.read}{}%
   \newcommand\SkipToFmtEnd{}%
   \newcommand\EndFmtInput{}%
   \long\def\SkipToFmtEnd#1\EndFmtInput{}%
  }\SkipToFmtEnd

\newcommand\ReadOnlyOnce[1]{\@ifundefined{#1}{\@namedef{#1}{}}\SkipToFmtEnd}
\usepackage{amstext}
\usepackage{amssymb}
\usepackage{stmaryrd}
\DeclareFontFamily{OT1}{cmtex}{}
\DeclareFontShape{OT1}{cmtex}{m}{n}
  {<5><6><7><8>cmtex8
   <9>cmtex9
   <10><10.95><12><14.4><17.28><20.74><24.88>cmtex10}{}
\DeclareFontShape{OT1}{cmtex}{m}{it}
  {<-> ssub * cmtt/m/it}{}
\newcommand{\texfamily}{\fontfamily{cmtex}\selectfont}
\DeclareFontShape{OT1}{cmtt}{bx}{n}
  {<5><6><7><8>cmtt8
   <9>cmbtt9
   <10><10.95><12><14.4><17.28><20.74><24.88>cmbtt10}{}
\DeclareFontShape{OT1}{cmtex}{bx}{n}
  {<-> ssub * cmtt/bx/n}{}
\newcommand{\tex}[1]{\text{\texfamily#1}}	% NEU

\newcommand{\Sp}{\hskip.33334em\relax}


\newcommand{\Conid}[1]{\mathit{#1}}
\newcommand{\Varid}[1]{\mathit{#1}}
\newcommand{\anonymous}{\kern0.06em \vbox{\hrule\@width.5em}}
\newcommand{\plus}{\mathbin{+\!\!\!+}}
\newcommand{\bind}{\mathbin{>\!\!\!>\mkern-6.7mu=}}
\newcommand{\rbind}{\mathbin{=\mkern-6.7mu<\!\!\!<}}% suggested by Neil Mitchell
\newcommand{\sequ}{\mathbin{>\!\!\!>}}
\renewcommand{\leq}{\leqslant}
\renewcommand{\geq}{\geqslant}
\usepackage{polytable}

%mathindent has to be defined
\@ifundefined{mathindent}%
  {\newdimen\mathindent\mathindent\leftmargini}%
  {}%

\def\resethooks{%
  \global\let\SaveRestoreHook\empty
  \global\let\ColumnHook\empty}
\newcommand*{\savecolumns}[1][default]%
  {\g@addto@macro\SaveRestoreHook{\savecolumns[#1]}}
\newcommand*{\restorecolumns}[1][default]%
  {\g@addto@macro\SaveRestoreHook{\restorecolumns[#1]}}
\newcommand*{\aligncolumn}[2]%
  {\g@addto@macro\ColumnHook{\column{#1}{#2}}}

\resethooks

\newcommand{\onelinecommentchars}{\quad-{}- }
\newcommand{\commentbeginchars}{\enskip\{-}
\newcommand{\commentendchars}{-\}\enskip}

\newcommand{\visiblecomments}{%
  \let\onelinecomment=\onelinecommentchars
  \let\commentbegin=\commentbeginchars
  \let\commentend=\commentendchars}

\newcommand{\invisiblecomments}{%
  \let\onelinecomment=\empty
  \let\commentbegin=\empty
  \let\commentend=\empty}

\visiblecomments

\newlength{\blanklineskip}
\setlength{\blanklineskip}{0.66084ex}

\newcommand{\hsindent}[1]{\quad}% default is fixed indentation
\let\hspre\empty
\let\hspost\empty
\newcommand{\NB}{\textbf{NB}}
\newcommand{\Todo}[1]{$\langle$\textbf{To do:}~#1$\rangle$}

\EndFmtInput
\makeatother
%
%
%
%
%
%
% This package provides two environments suitable to take the place
% of hscode, called "plainhscode" and "arrayhscode". 
%
% The plain environment surrounds each code block by vertical space,
% and it uses \abovedisplayskip and \belowdisplayskip to get spacing
% similar to formulas. Note that if these dimensions are changed,
% the spacing around displayed math formulas changes as well.
% All code is indented using \leftskip.
%
% Changed 19.08.2004 to reflect changes in colorcode. Should work with
% CodeGroup.sty.
%
\ReadOnlyOnce{polycode.fmt}%
\makeatletter

\newcommand{\hsnewpar}[1]%
  {{\parskip=0pt\parindent=0pt\par\vskip #1\noindent}}

% can be used, for instance, to redefine the code size, by setting the
% command to \small or something alike
\newcommand{\hscodestyle}{}

% The command \sethscode can be used to switch the code formatting
% behaviour by mapping the hscode environment in the subst directive
% to a new LaTeX environment.

\newcommand{\sethscode}[1]%
  {\expandafter\let\expandafter\hscode\csname #1\endcsname
   \expandafter\let\expandafter\endhscode\csname end#1\endcsname}

% "compatibility" mode restores the non-polycode.fmt layout.

\newenvironment{compathscode}%
  {\par\noindent
   \advance\leftskip\mathindent
   \hscodestyle
   \let\\=\@normalcr
   \let\hspre\(\let\hspost\)%
   \pboxed}%
  {\endpboxed\)%
   \par\noindent
   \ignorespacesafterend}

\newcommand{\compaths}{\sethscode{compathscode}}

% "plain" mode is the proposed default.
% It should now work with \centering.
% This required some changes. The old version
% is still available for reference as oldplainhscode.

\newenvironment{plainhscode}%
  {\hsnewpar\abovedisplayskip
   \advance\leftskip\mathindent
   \hscodestyle
   \let\hspre\(\let\hspost\)%
   \pboxed}%
  {\endpboxed%
   \hsnewpar\belowdisplayskip
   \ignorespacesafterend}

\newenvironment{oldplainhscode}%
  {\hsnewpar\abovedisplayskip
   \advance\leftskip\mathindent
   \hscodestyle
   \let\\=\@normalcr
   \(\pboxed}%
  {\endpboxed\)%
   \hsnewpar\belowdisplayskip
   \ignorespacesafterend}

% Here, we make plainhscode the default environment.

\newcommand{\plainhs}{\sethscode{plainhscode}}
\newcommand{\oldplainhs}{\sethscode{oldplainhscode}}
\plainhs

% The arrayhscode is like plain, but makes use of polytable's
% parray environment which disallows page breaks in code blocks.

\newenvironment{arrayhscode}%
  {\hsnewpar\abovedisplayskip
   \advance\leftskip\mathindent
   \hscodestyle
   \let\\=\@normalcr
   \(\parray}%
  {\endparray\)%
   \hsnewpar\belowdisplayskip
   \ignorespacesafterend}

\newcommand{\arrayhs}{\sethscode{arrayhscode}}

% The mathhscode environment also makes use of polytable's parray 
% environment. It is supposed to be used only inside math mode 
% (I used it to typeset the type rules in my thesis).

\newenvironment{mathhscode}%
  {\parray}{\endparray}

\newcommand{\mathhs}{\sethscode{mathhscode}}

% texths is similar to mathhs, but works in text mode.

\newenvironment{texthscode}%
  {\(\parray}{\endparray\)}

\newcommand{\texths}{\sethscode{texthscode}}

% The framed environment places code in a framed box.

\def\codeframewidth{\arrayrulewidth}
\RequirePackage{calc}

\newenvironment{framedhscode}%
  {\parskip=\abovedisplayskip\par\noindent
   \hscodestyle
   \arrayrulewidth=\codeframewidth
   \tabular{@{}|p{\linewidth-2\arraycolsep-2\arrayrulewidth-2pt}|@{}}%
   \hline\framedhslinecorrect\\{-1.5ex}%
   \let\endoflinesave=\\
   \let\\=\@normalcr
   \(\pboxed}%
  {\endpboxed\)%
   \framedhslinecorrect\endoflinesave{.5ex}\hline
   \endtabular
   \parskip=\belowdisplayskip\par\noindent
   \ignorespacesafterend}

\newcommand{\framedhslinecorrect}[2]%
  {#1[#2]}

\newcommand{\framedhs}{\sethscode{framedhscode}}

% The inlinehscode environment is an experimental environment
% that can be used to typeset displayed code inline.

\newenvironment{inlinehscode}%
  {\(\def\column##1##2{}%
   \let\>\undefined\let\<\undefined\let\\\undefined
   \newcommand\>[1][]{}\newcommand\<[1][]{}\newcommand\\[1][]{}%
   \def\fromto##1##2##3{##3}%
   \def\nextline{}}{\) }%

\newcommand{\inlinehs}{\sethscode{inlinehscode}}

% The joincode environment is a separate environment that
% can be used to surround and thereby connect multiple code
% blocks.

\newenvironment{joincode}%
  {\let\orighscode=\hscode
   \let\origendhscode=\endhscode
   \def\endhscode{\def\hscode{\endgroup\def\@currenvir{hscode}\\}\begingroup}
   %\let\SaveRestoreHook=\empty
   %\let\ColumnHook=\empty
   %\let\resethooks=\empty
   \orighscode\def\hscode{\endgroup\def\@currenvir{hscode}}}%
  {\origendhscode
   \global\let\hscode=\orighscode
   \global\let\endhscode=\origendhscode}%

\makeatother
\EndFmtInput
%
%
%
% First, let's redefine the forall, and the dot.
%
%
% This is made in such a way that after a forall, the next
% dot will be printed as a period, otherwise the formatting
% of `comp_` is used. By redefining `comp_`, as suitable
% composition operator can be chosen. Similarly, period_
% is used for the period.
%
\ReadOnlyOnce{forall.fmt}%
\makeatletter

% The HaskellResetHook is a list to which things can
% be added that reset the Haskell state to the beginning.
% This is to recover from states where the hacked intelligence
% is not sufficient.

\let\HaskellResetHook\empty
\newcommand*{\AtHaskellReset}[1]{%
  \g@addto@macro\HaskellResetHook{#1}}
\newcommand*{\HaskellReset}{\HaskellResetHook}

\global\let\hsforallread\empty

\newcommand\hsforall{\global\let\hsdot=\hsperiodonce}
\newcommand*\hsperiodonce[2]{#2\global\let\hsdot=\hscompose}
\newcommand*\hscompose[2]{#1}

\AtHaskellReset{\global\let\hsdot=\hscompose}

% In the beginning, we should reset Haskell once.
\HaskellReset

\makeatother
\EndFmtInput


% https://github.com/conal/talk-2015-essence-and-origins-of-frp/blob/master/mine.fmt
% Complexity notation:






% If an argument to a formatting directive starts with let, lhs2TeX likes to
% helpfully prepend a space to the let, even though that's seldom desirable.
% Write lett to prevent that.













































% Hook into forall.fmt:
% Add proper spacing after forall-generated dots.











% We shouldn't use /=, that means not equal (even if it can be overriden)!







% XXX



%  format `stoup` = "\blackdiamond"






% Cancel the effect of \; (that is \thickspace)



% Use as in |vapply vf va (downto n) v|.
% (downto n) is parsed as an application argument, so we must undo the produced
% spacing.

% indexed big-step eval
% without environments
% big-step eval
% change big-step eval








% \, is 3mu, \! is -3mu, so this is almost \!\!.


\def\deriveDefCore{%
\begin{align*}
  \ensuremath{\Derive{\lambda (\Varid{x}\typcolon\sigma)\to \Varid{t}}} &= \ensuremath{\lambda (\Varid{x}\typcolon\sigma)\;(\Varid{dx}\typcolon\Delta \sigma)\to \Derive{\Varid{t}}} \\
  \ensuremath{\Derive{\Varid{s}\;\Varid{t}}} &= \ensuremath{\Derive{\Varid{s}}\;\Varid{t}\;\Derive{\Varid{t}}} \\
  \ensuremath{\Derive{\Varid{x}}} &= \ensuremath{\Varid{dx}} \\
  \ensuremath{\Derive{\Varid{c}}} &= \ensuremath{\DeriveConst{\Varid{c}}}
\end{align*}
}


% Drop unsightly numbers from function names. The ones at the end could be
% formatted as subscripts, but not the ones in the middle.


% \chapter{Incrementalizing programs}

% \chapter{Incrementalizing primitives}
% \section{A generalized group fold}
% \pg{Status: very tentative}
% We have discussed how groups can induce change structures in one way (with
% change type |da| equal to base type |a| and with the support of the group).%
% \footnote{Technically, to consider a group on programs, we need to use groups in
%   the syntactic category of types and programs.}

% A more general way takes a group uses a base type |a|, a group |G = (da, +, -,
% e)|, and a function |`oplus` : a -> da -> a| and defines a change structure
% where:
% \begin{itemize}
% \item the change type is constantly |da| for all elements of |a|;
% \item the associative change composition operation |`ocompose` : da -> da -> da|
%   is given by the group operation |+|;
% \item the nil change is the neutral group element |e|;
% \item change application is given by |`oplus`|.
% \end{itemize}

% In such a change structure we obtain a few peculiar extra members:
% \begin{itemize}
% \item the unary group inverse operation |-| doubles as a change negation
%   operation, such that |dx `ocompose` (- dx) = nil|.
% \end{itemize}

% We can additionally require an operation |elLift: A -> DA| lifting elements to changes.

% Given such a change structure, we can define a \emph{group fold} over a
% collection type constructor |T|: |groupFold: (Group da, ChangeStruct a da) => (a
% -> da) -> T a -> da|. Such an operation would generalize the one used by
% \citet{CaiEtAl2014ILC}, dropping the requirement that |a = da|. Thanks to change negation, a
% derivative |dgroupFold| could react to the removal of collection elements.

% \chapter{Incrementalization beyond ILC}
% \label{ch:inc-more-programs}
% \pg{Only a very rough sketch for now}

% In previous chapters we have shown how to apply finite differencing to
% programs with first-class functions. With finite differencing we can
% incrementalize a few programs, but for others we run into problems:

% \begin{enumerate}
% \item The incremental computation does not have accesss to intermediate results
%   from the original computation.
% \item Since function changes are represented as functions, they cannot be
%   inspected at runtime, preventing a few optimizations. For instance, applying a
%   derivative to a nil change always produces a nil change, but we never take
%   advantage of this to optimize derivatives, except sometimes at compile time.
% \item Applying a derivative to a nil change always produce a nil change, but we
%   never take advantage of this to optimize derivatives, except sometimes at
%   compile time.
% \end{enumerate}

% To show these limitations
% % \item No support for change composition: there is no direct way to compose a
% %   sequence of changes |dx1, dx2, dx3, ...| across |x0, x1, x2, x3, ...| and
% %   produce a single change, except by applying all those changes and computing a
% %   difference with |x0 `oplus` dx1 `oplus` dx2 `oplus` dx3|.

% % \item Change structures must provide a difference operation, even though most
% %   often we are not supposed to use it.

% \chapter{Misc to integrate}
% \subsection{Applying function changes to composed changes}
% \pg{Move somewhere else. About lists?}
% \pg{No subscript for |`ocompose`| here?}

% \pg{Status: this is still text where I figure this out.}

% To implement change composition |da1 `ocompose` da2|, we can represent changes
% by lists of individual or \emph{atomic} changes. Change composition is then just
% concatenation. We can then extend functions on atomic changes to functions on
% changes. Next, we show how.

% Imagine, in the setting of \citet{CaiEtAl2014ILC} extended with change composition, applying
% one function change |df| to a composed argument change |da1 `ocompose` da2|,
% where |f1 `oplus` df = f2|, |a1 `oplus` da1 = a2| and |a2 `oplus` da2 = a3|. We
% want to compute incrementally the difference |db| between |f2 a3| and |f1 a1|.
% The change |db1 = df a1 da1| goes from |f1 a1| to |f2 a2|. With a change |db2|
% going from |f2 a2| to |f2 a3|, we can assemble the desired change as |db = db1
% `ocompose` db2|.

% However, to compute |db2| incrementally we cannot apply |df| to |da2|: |df a2
% da2| gives the difference between |f1 a2| and |f2 a3|, but |db2| should be the
% difference between |f2 a2| and |f2 a3|. Hence, we need to compute a nil change
% for |f2|.

% Doing that efficiently from the available elements is hard. However, with
% defunctionalized changes this becomes much easier (assuming the function code
% stays the same). Given the code of |f1| and |f2| we can produce a derivative for
% the base function; we can then combine that derivative with an updated
% environment. \pg{continue and clarify}
% % Why are these methods around?
% % \begin{code}
% % derPreDFun1 :: (ChangeStruct a, NilChangeStruct env) => Code env a b -> env -> Dt^env -> a -> Dt^a -> Dt^b
% % derPostDFun1 :: (ChangeStruct a, NilChangeStruct env) => Code env a b -> env -> Dt^env -> a -> Dt^a -> Dt^b
% % \end{code}

% % \subsection{Replacing functions without necessarily recomputing}
% % If a function change replaces a base function with a different one, in general
% % we must simply call the new function and produce a replacement change. However,
% % this can be wasteful if we replace a function by a similar one.
% %
% % Luckily we can add special cases to |derApplyDFun1|. To get started, suppose for
% % instance we replace |(+1)| by |(+2)|. Instead of replacing all the results, we
% % can add a specialized pattern match producing an equivalent change that however
% % is not a replacement one:
% %
% % \begin{code}
% % derApplyDFun1 :: (ChangeStruct a, NilChangeStruct env) =>
%   % Code env a b -> env -> DCode env a b -> Dt^env -> a -> Dt^a -> Dt^b
% % derApplyDFun1 (P Add1 _()) (DP (Replace Add2) ()) = +1 -- As a change
% % ...
% % derApplyDFun1 (P _f _env) (DP (Replace newF) newEnv) = oreplace (newF newEnv)
% % \end{code}
% %
% % This is enabled by defunctionalizing both base functions and changes.

% \section{Mapping changeable functions over sequences}
% \label{sec:map-seq}
% \pg{By far not done.}

% In this section we demonstrate how we can incrementalize by hand |map| and
% |concatMap| over collections, in a way that is compatible with and allows for
% ``deep'' changes to single sequence elements.

% We assume from previous sections knowledge of:
% \begin{itemize}
% \item cache-passing style
% \item defunctionalized function changes
% \end{itemize}

% and assume, for simplicity, the following type of sequence changes:
% \pg{single or atomic?}
% \begin{code}
% data SeqSingleChange a
%   =  Insert    { idx :: Int, x :: a }
%   |  Remove    { idx :: Int }
%   |  ChangeAt  { idx :: Int, dx :: Dt ^ a }
% data SeqChange a = Seq (SeqSingleChange a)
% type Dt (Seq a) = SeqChange a
% \end{code}

% Let us incrementalize |ys = map f xs : Seq b|, with |f : a -> b| and |xs : Seq
% a|. We want to compute the output change |dys| when |xs| changes by |dxs| and
% |f| changes by |df|.

% If |df| is not a nil change we must apply it to each element of |xs| to compute
% changes to each element of |ys|.\pg{Write definition using change composition.}
% Since looping over |xs| would take linear time,
% we want to avoid it if posssible: we need to detect whether |df| is a nil change
% or not. Sometimes we can detect this statically\pg{reference to section}, but
% more often we only detect this dynamically. So we assume that |Dt (a -> b)|
% supports detecting nil changes via |isNil|.

% Let us first assume that |df| is a nil change; let us moreover assume that |dxss
% : Dt (Seq a)| contains only one simple change |dxs : SeqSingleChange a|. If
% |dxs| simply adds or removes an element |x|, we can simply add or remove the
% corresponding element |y = f x| from |ys|.
% %
% Assume now |dxs| changes an element, that is, |dxs = ChangeAt idx dx| where |idx|
% is the index of the changed element |x = xs !! idx|, and |dx : Dt ^ el| is the
% element change. The output change will then be |ChangeAt idx dy| where |dy = df x
% dx|. Producing this change is a self-maintainable process only if |df| is itself
% self-maintainable.

% Since |f| might produce intermediate results that are needed by the derivative,
% we should use a cache-passing version of |f|, and adapt |map| accordingly. And
% |df| might not be self-maintainable, so we better make the base input available
% to |dmap| by storing it in a cache for |map|. Hence we use as interface:

% \begin{code}
% data MapCache a b cache = ...
% getFc :: MapCache a b cache -> Fun a b cache
% getCaches :: MapCache a b cache -> Seq cache
% mapC :: Fun a b cache -> Seq a -> (Seq b, MapCache a b cache)

% dmapC ::
%   DFun a b cache1 cache2 -> Dt (Seq a) ->
%   MapCache a b cache -> (Dt (Seq b), MapCache a b cache)
% \end{code}

% So the implementation of |dmapCSingle| for an element change is:
% \pg{typecheck this}
% \begin{code}
% dmapCSingle ::
%   DFun a b cache1 cache2 -> SeqSingleChange a ->
%   MapCache a b cache -> (Dt (Seq b), MapCache a b cache)
% dmapCSingle dfc (ChangeAt idx dx) mapCache1 = (singleton (ChangeAt idx dy), mapCache2)
%   where
%     caches1 = getCaches mapCache1
%     cx1 = caches1 `index` idx
%     (dy, cx2) = dfc dx cx
%     caches2 = update idx cx2 caches1
%     mapCache2 = mapCache1 { getCaches = caches2 }
% \end{code}
% where we assume the following interface for operating on sequences (as supplied by |Data.Seq|):
% \begin{code}
% singleton :: a -> Seq a
% index :: Seq a -> Int -> a
% update :: Int -> a -> Seq a -> Seq a
% \end{code}

% % \chapter{Static caching}
% % \label{sec:static-caching}
% % \pg{Write it!}

% % \section{Motivating Cache-passing Style}

% % \pg{It'd be nice to type the smart approach to cache type variables. We can't
% %   generally. It would be good to characterize when it can be used, but we don't
% %   do that either. Instead, we just show examples of what would be possible.}
% % %
% % \pg{We conjecture that we can type using free type variables:
% %   \begin{itemize}
% %   \item we can probably thread type variables where possible and use the packing trick elsewhere.
% % \end{itemize}
% % }

% % \pg{In fact, assume a combinator |mapInt : (Int -> Int) -> List Int -> List
% %   Int|. We can't prove that |mapInt f| uses its argument as we expect, maybe it
% %   does nothing on the list elements, or maybe it maps |inc| on (some of) them.
% %   Many such behaviors are allowed by its type; but our translation turns these
% %   |mapInt| variants with the same type into functions with different cache
% %   types. Mapping different functions over different elements produces.}


