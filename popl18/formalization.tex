\section{Formalization}
\label{sec:formalization}
In this section, we formalize differentiation with static caching and formally
prove its soundness with respect to differentiation alone. We furthermore
present a novel proof of correctness for differentiation itself.

To simplify the proof, we encode many invariants of input and output programs by
defining input and output languages with a suitably restricted abstract syntax:
Restricting the input language simplifies the transformations, while restricting
the output languages simplifies the semantics. Since base programs and
derivatives have different shapes, we define different syntactic categories of
\emph{base terms} and \emph{change terms}.
% following the example of...

We define and prove sound three transformations, across two languages: first, we
present a variant of differentiation without caching (as defined by
\citet{CaiEtAl2014ILC}), going from base terms of language $\source$ to change
terms of $\source$, and we prove it sound with respect to non-incremental
evaluation.
%
Second, we define differentiation with static caching as a pair of
transformations going from base terms of $\source$: the first transformation
produces CTS versions of base functions as base terms in $\target$,
and the second transformation produces CTS derivatives as change terms
in $\target$.

As source language for differentiation with static caching,
we use a core language named~$\source$.  This
language is a common target of a compiler front-end for a functional
programming language: in this language, terms are written in
$\lambda$-lifted and in a variant of A-normal form~\cite{sabry1993reasoning}, hence every
intermediate result is named. Hence intermediate results can be reused
and stored in a cache.

The target language for differentiation with static caching is
named~$\target$. Programs in this target language are also
$\lambda$-lifted and in a variant of A-normal form. But additionally, in these
programs every toplevel function $\tf$ produces a cache which is to be
conveyed to the derivatives of $\tf$.

The syntax and semantics of the source and target languages are
presented in Sections~\ref{sec:sourcelanguage}
and~\ref{sec:targetlanguage}. The transformation is described in
Section~\ref{sec:transformation} and its proof of soundness
is given in Section~\ref{sec:transformation-soundness}. Most of
the formalization have been mechanized in Coq (the proofs of some
straightforward lemmas are left for future work).


\subsection{The source language \source}
\label{sec:sourcelanguage}

\input{\poplPath{source-definition-syntax}}

\paragraph{Syntax}
The syntax of~$\source$ is given in
Figure~\ref{fig:source-definition}. Our source language allows
representing both \emph{base} terms~$\sterm$ and \emph{change}
terms~$\idterm$.

Our syntax for base terms~$\sterm$ represents $\lambda$-lifted
programs in A'-normal form (\cref{sec:motivating-example}).
So a term can be either a bound variable~$\tx$, or a {\bf let}
binding~$\ty$ in subterm~$\sterm$ to either a new
tuple~$\stuple{\many\tx}$ ($\slet{\ty = \stuple{\many\tx}}\sterm$), or
the result of calling function~$\tf$ on argument~$\tx$
($\slet{\ty = \sapp{\tf}{\tx}}{\sterm}$).  Both~$\tf$ and~$\tx$ are
variables to be looked up in the environment.  Terms cannot contain
$\lambda$-abstractions as they have been $\lambda$-lifted to top-level
definitions, which we encode as closures in the initial environments.
%
Unlike standard ANF we add no special syntax for function
calls in tail position (see \cref{sec:tail-calls} for a discussion
about this limitation).
%
We often need to inspect the result of
a function call~$\tf\tx$, which is not a valid term in our syntax.
To alleviate this, we write~``$\scall\tf\tx$''
for ``$\slet{\ty = \tf\;\tx}\ty$'' where $\ty$ is
chosen fresh.

Our syntax for change terms~$\idterm$ mimicks the syntax for base
terms except that (i) each function call is immediately followed by
the evaluation of its derivative and (ii) the final value returned by
a change term is a change variable~$\tdx$. As we will see, this
\emph{adhoc} syntax of change terms is tailored to capture the
programs produced by differentiation.  We only allows
$\alpha$-renamings that maintain the invariant that the definition
of~$\tx$ is immediately followed by the definition of~$\tdx$: if~$\tx$
is renamed to~$\ty$ then $\tdx$ must be renamed to~$\tdy$.

\paragraph{Semantics}

The closed values for the base terms are composed of closures, tuples
of values, constants and primitives. A closure is a pair of an
evaluation environment~$\sectx$ and a $\lambda$-abstraction closed
with respect to~$\sectx$. The set of available constants is left
abstract. It may contain usual first-order constants like integers. We
also leave abstract the primitives $\sprim$ like
\textbf{\texttt{if-then-else}} or projections of tuple components. As
usual, environments~$\sectx$ map variables to closed values. With no
loss of generality, we assume that all bound variables are distinct.

\input{\poplPath{source-definition-base-semantics}}

In Figure~\ref{fig:source-definition-base-terms-semantics}, the base
terms are evaluated using a step-indexed big-step semantics whose
judgment~$\seval\sectx\sterm{n}\sclosedvalue$ is read ``Under the
environment~$\sectx$, the base term~$\sterm$ evaluates into the closed
value~$\sclosedvalue$ in~$n$~steps.''
%
\rulename{SVar} looks up into~$\sectx$ to return the closed value of a
variable~$\tx$.
%
\rulename{STuple} looks up for~$\many\tx$ into~$\sectx$ to evaluate
the tuple~$\stuple{\many\tx}$, binds it to~$\ty$ and continues with
the next term.
%
\rulename{SPrimitiveCall} assumes that~$\tf$ is mapped to a
primitive~$\sprim$ in~$\sectx$ and that $\sprim$ has an associated
interpretation realized as a function~$\tdelta\sprim\param$ from
closed values to closed values. The evaluation of this interpretation
function applied to the value of~$\tx$ is bound to~$\ty$ in~$\sectx$
in order to evaluate the remaining term~$\sterm$ in~$n$ steps. In that
case, the evaluation of the $\textbf{let}$-binding takes~$n + 1$~steps.
%
\rulename{SClosureCall} assumes that~$\tf$ is mapped
to a closure in~$\sectx$ and evaluates the body of this closure in $n$
steps in the closure environment~$\sectx_\tf$ extended with the value
of its argument~$\tx$ found in~$\sectx$. The resulting value is bound
to~$\ty$ in $\sectx$ to evaluate the remaining term $\sterm$ in~$m$
steps. Hence, the whole term evaluates in~$n + m + 1$~steps.

A closed change value is either a closure change, a tuple change,
a literal change, a replacement change or a nil change.
%
A closure change is a pair made of an evaluation environment~$\iectx$
and a~$\lambda$-abstraction expecting a value and a change value as
arguments to evaluate a change term into an output change value. An
evaluation environment~$\iectx$ follows the same structure as
$\textbf{let}$-bindings of the change terms: it binds variables to
closed values and each variable~$\tx$ is immediately followed by a
binding for its associated change variable~$\tdx$. As with
$\textbf{let}$-bindings of change terms, $\alpha$-renamings in an
environment~$\iectx$ must rename~$\tdx$ into~$\tdy$ if~$\tx$ is
renamed into~$\ty$. With no loss of generality, we assume that all
bound term variables are distinct in these environments.  We define
the \textit{original environment}~$\oldenv\iectx$ of a change
environment~$\iectx$ by induction over~$\iectx$:
\[
\begin{array}{rcl}
\oldenv\sectxempty & = & \sectxempty \\
\oldenv{\iectxcons\iectx{\tx = \sclosedvalue; \tdx = \icloseddvalue}}
& = & \iectxcons{\oldenv{\iectx}}{\tx = \sclosedvalue}
\end{array}
\]

The \textit{new environment}~$\newenv\iectx$ of a change
environment~$\iectx$ is defined by induction over~$\iectx$:
\[
\begin{array}{rcl}
\newenv\sectxempty & = & \sectxempty \\
\newenv{\iectxcons\iectx{\tx = \sclosedvalue; \tdx = \icloseddvalue}}
& = & \iectxcons{\newenv{\iectx}}{\tx = \sclosedvalue \oplus \icloseddvalue}
\end{array}
\]

This last rule makes use of an operation $\oplus$ to update a value
with a change, which may fail at runtime. Indeed, the application of a
change is a partial function written
``$\sclosedvalue \oplus \icloseddvalue$'' which is defined as follows:
\[
  \begin{array}{rclcl}
    \sconst & \oplus & \idconst & = & \delta_{\oplus} (\sconst, \idconst)
    \\
    \sclosedvalue_1 & \oplus & \replace{\sclosedvalue_2} & = & \sclosedvalue_2
    \\
    \sclosedvalue & \oplus & \inil & = & \sclosedvalue
    \\
    \sclosure{\sectx}{\slam\tx\sterm} & \oplus &
    \iclosure\iectx{\ilam{\tx\,\tdx}{\idterm}} & = &
    \sclosure{(\sectx \oplus \iectx)}{\slam\tx\sterm}
    \\
    \stuple{\sclosedvalue_1, \ldots, \sclosedvalue_n}
    & \oplus
    & \stuple{\icloseddvalue_1, \ldots, \icloseddvalue_n}
    & =
    & \stuple{\sclosedvalue_1 \oplus \icloseddvalue_1, \ldots, \sclosedvalue_n \oplus \icloseddvalue_n}
  \end{array}
\]
\noindent where
\[
    (\tectxcons\sectx{\tx = \sclosedvalue}) \oplus
    (\tectxcons\iectx{\tx = \sclosedvalue; \tdx = \icloseddvalue})
    =
    (\tectxcons{(\sectx \oplus \iectx)}{\tx = (\sclosedvalue \oplus \icloseddvalue)})
\]

Nil and replacement changes can be used to update both
constants, tuples, primitives and closures, while tuple change can
only update tuples, literal change can only update literals and
closure changes can only update closures. Indeed, $\oplus$ on literals
is defined \textit{via} some interpretation function~$\delta_{\oplus}$. A
replacement change overrides the current value~$\sclosedvalue$ with a
new one~$\sclosedvalue'$.
Change application for a closure
ignores~$\idterm$ instead of combining it with $\sterm$ somehow. This may seem
surprising, but we only need $\oplus$ to behave well for valid changes (as shown
by \cref{lemma:crel-oplus}): for valid closure changes,
$\idterm$ must behave similarly to $\iderive\sterm$ anyway,\pg{Not formal, not
  visible in the definition, but true for a suitable meaning of ``similarly'';
  but I'm clearly not claiming a theorem, just giving a hint for clarity.}
so only environment updates matter. This definition also avoids having to modify
terms at runtime, which would have been difficult to implement safely.
\pg{$\oplus$ is not used by derivatives, so commenting this part out
A change application for a closure
ignores~$\idterm$, which may seems strange at first but is needed to
but is needed to
get guarantees about the evaluation of derivatives.}
% With this restriction, we avoid the case of a new
% closure code that would refer to a variable undefined in the closure
% environment.
% \pg{I can imagine fixes to that problem, so I'd propose to stick to the
% stronger arguments.}

\input{\poplPath{source-definition-change-semantics}}

In Figure~\ref{fig:source-definition-change-terms-semantics}, change
terms are evaluated using a big-step semantics whose
judgment~$\seval\iectx\idterm{}\icloseddvalue$ is read ``Under the
environment $\iectx$, the change term $\idterm$ evaluates into the
closed change value $\icloseddvalue$.''
%
$\rulename{SDVar}$ looks up into $\iectx$ to return a value for $\tdx$.
%
$\rulename{SDTuple}$ builds a tuple out of the values of~$\many\tx$
and a change tuple out of the change values of~$\many\tdx$ as found
in the environment~$\iectx$.
%
There are four rules to evaluate $\textbf{let}$-binding depending on
the nature of~$\iectxlookup\iectx\tdf$. These four rules
systematically recomputes the value~$\sclosedvalue_\ty$ of~$\ty$ in
the original environment. They differ in the way they compute the
change~$\tdy$ to~$\ty$.

If $\iectxlookup\iectx\tdf$ is a replacement,
$\rulename{SDReplaceCall}$ applies. Replacing the value of~$\tf$
in the environment forces the recomputation of~$\tf\,\tx$ from
scratch in the new environment. The resulting value~$\sclosedvalue'_\ty$
is the new value which must replace~$\sclosedvalue_\ty$: therefore
$\tdy$ equals~$\replace{\sclosedvalue'_\ty}$ in that case.

If $\iectxlookup\iectx\tdf$ is a nil change. We have two rules
depending on the nature of $\iectxlookup\iectx\tf$. If
$\iectxlookup\iectx\tf$ is a closure, $\rulename{SDClosureNil}$
applies and in that case the nil change of~$\iectxlookup\iectx\tf$ is
the exact same closure. Hence, to compute $\tdy$, we must reevaluate
this closure applied to the updated
argument~$\iectxlookup{\newenv\iectx}\tx$ to a
value~$\sclosedvalue'_\ty$ and bind~$\tdy$ to
$\replace{\sclosedvalue'_\ty}$. In other words, this rule is
equivalent to $\rulename{SDReplaceCall}$ in the case where a closure
is replaced by itself.
%
If $\iectxlookup\iectx\tf$ is a primitive, $\rulename{SDPrimitiveNil}$
applies. The nil change of a primitive~$\sprim$ is its derivative
which interpretation is realized by a
function~$\Delta_\sprim(\param)$.  The evaluation of this function on
the input value and the input change leads to the
change~$\icloseddvalue_\ty$ bound to $\tdy$.

If $\iectxlookup\iectx\tf$ is a closure change
$\iclosure{\iectx_\tf}{\ilam{\tx\,\tdx}{\idterm_\tf}}$,
$\rulename{SDClosureChange}$ applies. The
change~$\icloseddvalue_\ty$ results from the evaluation
of~$\idterm_\tf$ in the closure change environment~$\iectx_\tf$
augmented with an input value for~$\tx$ and a change value
for~$\tdx$. Again, let us recall that we will maintain the
invariant that the term~$\idterm_\tf$ behaves as the derivative
of~$\tf$ so this rule can be seen as the invokation of~$\tf$'s
derivative.

\paragraph{Expressiveness}

A closure in the initial environment can be used to represent a
top-level definition. Since environment entries can point to
primitives, we need no syntax to directly represent calls of
primitives in the syntax of base terms. To encode in our syntax a
program with top-level definitions and a term to be evaluated
representing the entry point, one can produce a term~$\sterm$
representing the entry point together with an environment~$\sectx$
containing as values any top-level definitions, primitives and
constants used in the program.

Our formalization does not model directly $n$-ary functions, but they can be
encoded through unary functions and tuples. This encoding does not support
currying efficiently, but we discuss possible solutions in
\cref{sec:nary-abstraction}.

Control operators, like recursion combinators or branching, can be
introduced as primitive operations as well. If the branching condition changes,
expressing the output change in general requires replacement changes. Similarly
to branching we can add tagged unions.

\subsection{Static differentiation in \source}
Previous work~\citep{CaiEtAl2014ILC} defines differentiation for simply-typed
$\lambda$-calculus terms.
Figure~\ref{fig:differentiation} shows differentiation for $\source$'s syntax.

\input{\poplPath{differentiation}}
Differentiating a base term~$\sterm$ produces a change
term~$\iderive{\sterm}$, its \emph{derivative}.
%
Differentiating final result variable~$\tx$ produces its change variable~$\tdx$.
Differentiation copies each binding of an intermediate result $\ty$ to the
output and adds a new bindings for its change $\tdy$.
%
If $\ty$ is bound to tuple~$\stuple{\many\tx}$, then $\tdy$ will
be bound to the change tuple $\stuple{\many\tdx}$.
If $\ty$ is bound to function application $\sapp\tf{\tx}$, then $\tdy$ will be
bound to the application of function change $\tdf$ to input $\tx$ and its change
$\tdx$.

Evaluating $\iderive{\sterm}$ recomputes all intermediate results
computed by $\sterm$. This recomputation will be avoided through cache-transfer
style in section \cref{sec:transformation}.

The original transformation for static differential of
$\lambda$-terms~\citep{CaiEtAl2014ILC} has three cases which
we recall here:
\[
  \begin{array}{rcl}
    \oderive\tx & = & \tdx \\
    \oderive{t \, u} & = & \oderive{t}\,u\,\oderive{u} \\
    \oderive{\lambda \tx. t} & = & \lambda \tx\,\tdx. \oderive{t}
  \end{array}
\]

Even though the first two cases of the original transformation are
easily mapped into the two cases of our variant, one may ask where the
third case is realized now. Actually, this third case occurs while we
transform the initial environment. Indeed, we will assume that the
closures of the environment of the source program have been adjoined a
derivative. More formally, we suppose that the derivative of $\sterm$
is evaluated under an environment~$\iderive\sectx$ obtained as
follows:
\[
  \begin{array}{rcl}
    \iderive{\bullet} & = & \bullet \\
    \iderive{\sectxcons\sectx{\tf = \sclosure{\sectx_f}{\lambda \tx. \sterm}}}
    &=&
    \sectxcons{\iderive\sectx}{\tf = \sclosure{\sectx_f}{\lambda \tx. \sterm},
    \tdf = \sclosure{\iderive{\sectx_f}}{\lambda \tx\,\tdx. \iderive{\sterm}}}
    \\
    \iderive{\sectxcons\sectx{\tx = \sclosedvalue}}
    &=&
    \sectxcons{\iderive\sectx}{\tx = \sclosedvalue, \tdx = \inil}
    \qquad \textrm{(If $\sclosedvalue$ is not a closure.)}
  \end{array}
\]
\subsection{A new soundness proof for static differentiation}
\label{sec:sound-derive}

As in \citet{CaiEtAl2014ILC}'s development, static differentiation is
only sound on input changes that are \emph{valid}. However, our
definition of validity needs to be significantly different.
%
\citeauthor{CaiEtAl2014ILC} prove soundness for a strongly normalizing
simply-typed $\lambda$-calculus using denotational semantics. We generalize this
result to an untyped and Turing-complete language using purely syntactic
arguments. In both scenarios, a function change is only valid if it turns valid
input changes into valid output changes, so validity is a logical relation.
Since standard logical relations only apply to typed languages, we turn to
\emph{step-indexed} logical relations.

\subsubsection{Validity as a step-indexed logical relation}
To state and prove soundness of differentiation, we define validity by
introducing a ternary step-indexed relation over base values, changes and
updated values, following previous work on step-indexed
logical relations~\citep{Ahmed2006stepindexed,Acar08}. Experts might notice
small differences in our step-indexing, as mentioned in
%\pg{Maybe move this into sec:cts-rw.}
\cref{sec:sourcelanguage}, but they do not affect the substance of the proof.
% \pg{We are not using biorthogonality: that notion embeds contextual equivalence
%   inside the definition of equivalence.}
%biorthogonality~\citep{pitts2010step}.
%
We write
\[\vvcrel {\sclosedvalue_2} k {\sclosedvalue_1} \icloseddvalue\]
\noindent and say that ``$\icloseddvalue$ is a valid change from
$\sclosedvalue_1$ to $\sclosedvalue_2$, up to $k$ steps'' to mean that
$\icloseddvalue$ is a change from $\sclosedvalue_1$ to
$\sclosedvalue_2$ and that $\icloseddvalue$ is a \emph{valid}
description of the differences between~$\sclosedvalue_1$ 
and~$\sclosedvalue_2$, with validity tested with up to~$k$~steps.
%
To justify this intuition of validity, we later state and prove two
lemmas: a valid change from~$\sclosedvalue_1$ to~$\sclosedvalue_2$
goes indeed from~$\sclosedvalue_1$ to~$\sclosedvalue_2$
(Lemma~\ref{lemma:crel-oplus}), and if a change is valid up to~$k$~steps it
is also valid up to fewer steps (Lemma~\ref{lemma:vvcrel-antimono}).

\begin{lemma}[$\oplus$ agrees with validity]
  \label{lemma:crel-oplus}
  If
  $\vvcrel {\sclosedvalue_2} k {\sclosedvalue_1} \icloseddvalue$ then
  $\sclosedvalue_1 \oplus \icloseddvalue = \sclosedvalue_2$.
\end{lemma}
\begin{lemma}[Downward-closure]
  \label{lemma:vvcrel-antimono}
  If $N \ge n$, then $\vvcrel{\sclosedvalue_2} {N} {\sclosedvalue_1} \icloseddvalue$ implies
  $\vvcrel{\sclosedvalue_2} {n} {\sclosedvalue_1} \icloseddvalue$.
\end{lemma}

\input{\poplPath{differentiation-step-indexed-relation}}

As usual with step-indexed logical relations, validity is defined by
well-founded induction over naturals ordered by $<$. To show this, it helps to
observe that evaluation always takes at least one step.

Validity is formally defined by cases in
Figure~\ref{fig:step-index-relation-between-values-and-changes}; we
describe in turn each case. First, a constant change $\idconst$ is a
valid change from $\sconst$ to
$\sconst \oplus \idconst = \delta_{\oplus} (\sconst, \idconst)$.
Since the function $\delta_\oplus$ is partial, the relation only holds
for the constant changes $\idconst$ which are valid changes for
$\sconst$.
%
Second, a replacement change~${\replace{\sclosedvalue_2}}$ is always a valid
change from any value~$\sclosedvalue_1$ to $\sclosedvalue_2$.
%
Third, a nil change is a valid change between any value and itself.
%
Fourth, a tuple change is valid up to step~$n$, if each of its components
is valid up to any step strictly less than~$k$.
%
Fifth, we define validity for closure changes. Roughly
speaking, this statement means that a closure change is valid if (i)
its environment change~$\iectx$ is valid for the original closure
environment~$\sectx_1$ and for the new closure environment~$\sectx_2$;
(ii) when applied to related values, the closure \textit{bodies}~$\sterm_1$
and~$\sterm_2$ are related by~$\idterm$. The validity relation between
terms is defined as follows:
\[
\begin{array}{l}
  \vvcrelterm{\sectx_1}{\iectx}{\sectx_2}{n}{\sterm_1}{\idterm}{\sterm_2} \\
  \textrm{if and only if }
  \forall k < n, \sclosedvalue_1, \sclosedvalue_2, \\
  \quad\seval{\sectx_1}{\sterm_1}{k}{\sclosedvalue_1} \textrm{ and }
  \seval{\sectx_2}{\sterm_2}{}{\sclosedvalue_2} \\
  \quad\textrm{implies that } \exists \icloseddvalue,
  \seval{\iectx}{\idterm}{}{\icloseddvalue} \wedge
  \vvcrel{\sclosedvalue_2}{n - k}{\sclosedvalue_1}{\icloseddvalue}
\end{array}
\]

\noindent
We extend this relation from values to environments by defining a
judgment $\vvcrel {\sectx_2} k {\sectx_1} \iectx$ defined as follows:
\begin{mathpar}
  \infer{}{
    \vvcrel \bullet k \bullet \bullet
  }

  \infer{
    \vvcrel {\sectx_2} k {\sectx_1} \iectx
    \\
    \vvcrel {\sclosedvalue_2} k {\sclosedvalue_1} \icloseddvalue
  }{
    \vvcrel
    {\tectxcons{\sectx_2}{\tx = \sclosedvalue_2}}
    k
    {(\tectxcons{\sectx_1}{\tx = \sclosedvalue_1})}
    {(\tectxcons\iectx{\tx = \sclosedvalue_1; \tdx = \icloseddvalue})}
  }
\end{mathpar}

The above lemmas about validity for values extend to environments.
\begin{lemma}[$\oplus$ agrees with validity, for environments]
  \label{lemma:crel-oplus-env}
  If
  $\vvcrel {\sectx_2} k {\sectx_1} \iectx$ then
  $\sectx_1 \oplus \iectx = \sectx_2$.
\end{lemma}
\begin{lemma}[Downward-closure, for environments]
  \label{lemma:vvcrel-antimono-env}
  If $N \ge n$, then $\vvcrel{\sectx_2} {N} {\sectx_1} \iectx$ implies
  $\vvcrel{\sectx_2} {n} {\sectx_1} \iectx$.
\end{lemma}

Finally, for both values, terms and environments, omitting the step
count~$k$ from validity means validity holds for all~$k$s. That is,
for instance,
$\vvcrel {\sclosedvalue_2} {} {\sclosedvalue_1} \icloseddvalue$ means
$\vvcrel {\sclosedvalue_2} {k} {\sclosedvalue_1} \icloseddvalue$ for
all $k$.

\subsubsection{Soundness of differentiation}

\yrg{This section will be expanded a bit to give a bit more insight about the proof.}

We can state a soundness theorem for differentiation without
mentioning step-indexes. Instead of proving it directly, we must first
prove a more technical statement
(Lemma~\ref{lemma:fundamental-property}) that mentions step-indexes
explicitly.
%
\begin{theorem}[Soundness of differentiation in $\source$]
  \label{thm:sound-derive}
  If $\iectx$ is a valid change environment from base environment $\sectx_1$ to
  updated environment $\sectx_2$, that is $\vvcrel {\sectx_2} {} {\sectx_1}\iectx$,
  %
  and if $\sterm$ converges both in the base and updated environment, that is
  $\sneval{\sectx_1}\sterm{\sclosedvalue_1}$ and
  $\sneval{\sectx_2}\sterm{\sclosedvalue_2}$,
  %
  then $\iderive\sterm$ evaluates under the change environment~$\iectx$ to a valid
  change $\icloseddvalue$ between base result ${\sclosedvalue_1}$ and updated result
  ${\sclosedvalue_2}$, that is
  $\sneval\iectx{\iderive\sterm}{\icloseddvalue}$,
  $\vvcrel{\sclosedvalue_2} {} {\sclosedvalue_1}
  \icloseddvalue$ and
  $\sclosedvalue_1 \oplus \icloseddvalue = \sclosedvalue_2$.
\end{theorem}

\begin{lemma}[Fundamental Property]
  \label{lemma:fundamental-property}
For each $n$,
if $\vvcrel{\sectx_2}{n}{\sectx_1}{\iectx}$ then
$\vvcrelterm{\sectx_1}{\iectx}{\sectx_2}{n}{\sterm}{\iderive{\sterm}}{\sterm}$.
\end{lemma}

\subsection{The target language \target}
\label{sec:targetlanguage}

In this section, we present the target language of a transformation
that extends static differentiation with a static caching mechanism.
As said earlier, the functions of $\target$ not only compute their
output but they also returns a cache containing the intermediate
values that have contributed to that output. This cache is transmitted
to derivatives to build changes on top of intermediate values with no
recomputation and derivatives update the caches according to their
input changes. 

\paragraph{Syntax}
The syntax of $\target$ is defined in
Figure~\ref{fig:target-definition-syntax}. The base terms of $\target$
follow the same shape as the $A$-normal $\lambda$-lifted terms of
$\source$ except that a $\bf let$-binding for a function
application~$\tf\,\tx$ now introduces an extra \textit{cache
  identifier}~$\tcacheid\ty\tf\tx$ in replacement for the
output~$\ty$. The syntax of cache identifiers is non standard: it can
be seen as a triple that refers to the value identifiers $\tf$, $\tx$
and $\ty$. Hence, an $\alpha$-renaming of one of these three
identifiers must refresh the cache identifier accordingly. The syntax
$(\tx, \tcache)$ for the result term makes explicit that a cache
$\tcache$ is systematically returned.
% A similar alpha-renaming issue must be made about the formal
% cache argument of caching-derivatives. In the Coq development,
% the positional approach to binders makes the problem disappear.

\begin{figure}[!tb]
  %\begin{multicols}{2}
  \footnotesize
%  \figsubtitle{Syntax}
  \[
  \begin{array}{rclr}
    \multicolumn{4}{r}{\syntaxclass{Base terms}} \\
    \tterm
    & ::= &
    \tlet{\ty, \tcacheid{\ty}{\tf}{\tx} = \tapp{\tf}{\tx}}{\tterm}
    & \rem{Call} \\
    & \mid &
    \tlet{\ty = \stuple{\many\tx}}
    & \rem{Tuple} \\
    & \mid & (\tx, \tcache)
    & \rem{Result} \\
    \nextline
    \multicolumn{4}{r}{\syntaxclass{Cache terms}} \\
    \tcache
    & ::= & \temptycache
    & \rem{Empty} \\
    & \mid & \tcachecons\tcache{\tcacheid{\ty}{\tf}{\tx}}
    & \rem{Sub-cache} \\
    & \mid & \tcachecons\tcache\tx
    & \rem{Cached value} \\
    \nextline
    \multicolumn{4}{r}{\syntaxclass{Change terms}} \\
    \tdterm
    & ::= &
    \tlet{\tdy, \tcacheid{\ty}{\tf}{\tx} = \tdapp{\tdf}{\tdx}{\tcacheid{\ty}{\tf}{\tx}}}{\tdterm}
    & \rem{Call} \\
    & \mid &
    \tlet{\tdy = \stuple{\many\tdx}}{\tdterm}
    & \rem{Tuple} \\
    & \mid & (\tdx, \tupdcache)
    & \rem{Result} \\
    \nextline
    \multicolumn{4}{r}{\syntaxclass{Cache updates}} \\
    \tupdcache
    & ::= & \temptycache
    & \rem{Empty} \\
    & \mid & \tcachecons\tupdcache\tcacheid{\ty}{\tf}{\tx}
    & \rem{Sub-cache} \\
    & \mid & \tcachecons\tupdcache{(\tchange\tx\tdx)}
    & \rem{Updated value} \\
    \nextline
    \multicolumn{4}{r}{\syntaxclass{Closed values}} \\
    \tclosedvalue
    & ::= & \tclosure\tectx{\tlam{\tx}{\tterm}}
    & \rem{Closure} \\
    & \mid & \stuple{\many\tclosedvalue}
    & \rem{Tuple} \\
    & \mid & \tconst
    & \rem{Literal} \\
    & \mid & \tprim
    & \rem{Primitive} \\
    \nextline
    \multicolumn{4}{r}{\syntaxclass{Cache values}} \\
    \tcachedclosedvalues
    & ::= & \temptycache
    & \rem{Empty} \\
    & \mid & \tcachecons\tcachedclosedvalues\tcachedclosedvalues
    & \rem{Sub-cache} \\
    & \mid & \tcachecons\tcachedclosedvalues\tclosedvalue
    & \rem{Cached value} \\
    \nextline
    \multicolumn{4}{r}{\syntaxclass{Change values}} \\
    \tcloseddvalue
    & ::= & \tclosure\tdectx{\tlam{\tdx\,\tcache}{\tdterm}}
    & \rem{Closure} \\
    & \mid & \stuple{\many\tcloseddvalue}
    & \rem{Tuple} \\
    & \mid & \tdconst
    & \rem{Literal} \\
    & \mid & \tdnil
    & \rem{Nil} \\
    & \mid & \replace\tclosedvalue
    & \rem{Replacement} \\
    \nextline
    \multicolumn{4}{r}{\syntaxclass{Base definitions}} \\
    \tvdef
    & ::= & \tx = \tclosedvalue
    & \rem{Value definition} \\
    & \mid & \tcacheid{\ty}{\tf}{\tx} = \tcachedclosedvalues
    & \rem{Cache definition} \\
    \nextline
    \multicolumn{4}{r}{\syntaxclass{Change definitions}} \\
    \tdvdef
    & ::= & \tvdef
    & \rem{Base} \\
    & \mid & \tdx = \tcloseddvalue
    & \rem{Change} \\
    \nextline
    \multicolumn{4}{r}{\syntaxclass{Evaluation environments}} \\
    \tectx
    & ::= & \tectxcons{\tectx}{\tvdef}
    & \rem{Binding} \\
    & \mid & \tectxempty
    & \rem{Empty} \\
    \nextline
    \multicolumn{4}{r}{\syntaxclass{Change environments}} \\
    \tdectx
    & ::= & \tectxcons{\tdectx}{\tdvdef}
    & \rem{Binding} \\
    & \mid & \tectxempty
    & \rem{Empty} \\
  \end{array}
  \]
%\end{multicols}
\caption{Target language $\target$ (syntax).}
  \label{fig:target-definition-syntax}
\end{figure}

\begin{figure}[htb]
    \small
  \begin{mathpar}
    \small
    \\
    \figsubtitle{Evaluation of base terms}{
      $\sneval{\tectx}{\tterm}{(\tclosedvalue, \tcachedclosedvalues)}$
    }
    \\

    \infer[\rulename{TResult}]{
      \tectxlookup\tectx\tx = \tclosedvalue \\
      \sneval{\tectx}{\tcache}{\tcachedclosedvalues}
    }{
      \sneval{\tectx}{(\tx, \tcache)}{(\tclosedvalue, \tcachedclosedvalues)}
    }

    \infer[\rulename{TTuple}]{
      \sneval
      {\tectxcons{\tectx}{\ty = \tectxlookup\tectx{\many\tx}}}
      {\tterm}
      {(\tclosedvalue, \tcachedclosedvalues)}
    }{
      \sneval
      {\tectx}
      {\tlet{\ty = \stuple{\many\tx}}{\tterm}}
      {(\tclosedvalue, \tcachedclosedvalues)}
    }

    \infer[\rulename{TClosureCall}]{
      \tectxlookup\tectx\tf = \tclosure{\tectx'}{\tlam{\tx'}{\tterm'}} \\\\
      \sneval
      {\tectxcons{\tectx'}{\tx' = \tectxlookup\tectx\tx}}{\tterm'}{(\tclosedvalue', \tcachedclosedvalues')} \\\\
      \sneval
      {\tectxcons\tectx{\ty = \tclosedvalue'; \tcacheid\ty\tf\tx = \tcachedclosedvalues'}}
      {\tterm}
      {(\tclosedvalue, \tcachedclosedvalues)}
    }{
      \sneval
      {\tectx}
      {\tlet{\ty, \tcacheid\ty\tf\tx = \tf\;\tx}{\tterm}}
      {(\tclosedvalue, \tcachedclosedvalues)}
    }

    \infer[\rulename{TPrimitiveCall}]{
      \tectxlookup\tectx\tf = \sconst
      \\
      \tdelta\sconst{\sectxlookup\tectx\tx} = (\tclosedvalue', \tcachedclosedvalues')
      \\
      \sneval
        {\sectxcons{\tectx}{\ty = \tclosedvalue'; \tcacheid\ty\tf\tx = \tcachedclosedvalues'}}
        {\tterm}
        {\sclosedvalue}
    }{
      \sneval\tectx
      {\slet{\ty, \tcacheid\ty\tf\tx = \sapp\tf\tx}{\tterm}}
      {(\tclosedvalue, \tcachedclosedvalues)}
    }

    \\
    \figsubtitle{Evaluation of caches}{
      $\sneval{\tectx}{\tcache}{\tcachedclosedvalues}$
    }
    \\

    \infer[\rulename{TEmptyCache}]{
    }{
      \sneval{\tectx}{\temptycache}{\temptycache}
    }

    \infer[\rulename{TCacheVar}]{
      \tectxlookup\tectx\tx = \tclosedvalue \\
      \sneval{\tectx}{\tcache}{\tcachedclosedvalues}
    }{
      \sneval{\tectx}{\tcachecons\tcache\tx}{\tcachecons\tcachedclosedvalues\tclosedvalue}
    }

    \infer[\rulename{TCacheSubCache}]{
      \tectxlookup\tectx{\tcacheid\ty\tf\tx} = \tcachedclosedvalues' \\
      \sneval{\tectx}{\tcache}{\tcachedclosedvalues}
    }{
      \sneval
      {\tectx}
      {\tcachecons\tcache{\tcacheid\ty\tf\tx}}
      {\tcachecons\tcachedclosedvalues{\tcachedclosedvalues'}}
    }
  \end{mathpar}
  \caption{Target language $\target$ (semantics of base terms and caches).}
  \label{fig:target-definition-base-semantics}
\end{figure}

\begin{figure}
    \small
  \begin{mathpar}
    \\
    \figsubtitle{Evaluation of change terms}{
      $\sneval{\tdectx}{\tdterm}{(\tcloseddvalue, \tcachedclosedvalues)}$
    }
    \\

    \infer[\rulename{TDResult}]{
      \tectxlookup\tdectx\tdx = \tcloseddvalue \\
      \sneval{\tdectx}{\tupdcache}{\tcachedclosedvalues}
    }{
      \sneval
      {\tdectx}
      {(\tdx, \tupdcache)}
      {(\tcloseddvalue, \tcachedclosedvalues)}
    }

    \infer[\rulename{TDTuple}]{
      \sneval
      {\tectxcons{\tdectx}{\tdy=\tectxlookup\tdectx{\many\tdx}}}
      {\tdterm}
      {(\tcloseddvalue, \tcachedclosedvalues)}
    }{
      \sneval
      {\tdectx}
      {\tlet
        {\tdy = \stuple{\many\tdx}}
        {\tdterm}
        {(\tcloseddvalue, \tcachedclosedvalues)}}
    }

    \infer[\rulename{TDReplaceCall}]{
      \tectxlookup\tdectx\tdf = \replace{\tclosedvalue_\tf} \\\\
      \sneval{\newenv{\tdectx}}{\scall\tf\tx}{(\tclosedvalue', \tcachedclosedvalues')} \\\\
      \sneval
      {\tectxcons\tdectx{\tdy = \replace{\tclosedvalue'}; \tcacheid\ty\tf\tx = \tcachedclosedvalues'}}
      {\tdterm}
      {(\tcloseddvalue, \tcachedclosedvalues)}
    }{
      \sneval
      {\tdectx}
      {\tlet{\tdy, \tcacheid\ty\tf\tx = \tdf\;\tdx\;\tcacheid\ty\tf\tx}{\tdterm}}
      {(\tcloseddvalue, \tcachedclosedvalues)}
    }

    \infer[\rulename{TDClosureNil}]{
      \tectxlookup\tdectx{\tf,\tdf} = \tclosure{\tdectx_\tf}{\slam{\tx}{\tterm_f}}, \inil \\\\
      \seval
      {\sectxcons{\tdectx_f}{\tx = \sectxlookup{\newenv\tdectx}\tx}}{\tterm_f}{}
      {(\tclosedvalue'_\ty, \tcachedclosedvalues')}
      \\\\
      \seval
      {\tectxcons\tdectx
        {\tdy = \replace{{\tclosedvalue_\ty}'} ; \tcacheid\ty\tf\tx = \tcachedclosedvalues'}}
      {\tdterm}
      {}
      {(\tcloseddvalue, \tcachedclosedvalues)}
    }{
      \sneval
      {\tdectx}
      {\tlet{\tdy, \tcacheid\ty\tf\tx = \tdf\;\tdx\;\tcacheid\ty\tf\tx}{\tdterm}}
      {(\tcloseddvalue, \tcachedclosedvalues)}
    }

    \infer[\rulename{TDPrimitiveNil}]{
      \sectxlookup\tdectx{\tf,\tdf} = \tprim, \inil \\
      \sectxlookup\tdectx{\tx, \tdx} = \tclosedvalue_\tx, \tcloseddvalue_\tx \\
      \seval
      {\iectxcons
        \iectx
        {\tdy, \tcacheid\ty\tf\tx =
          \tddelta
          \tprim
          {\tclosedvalue_\tx,\tcloseddvalue_\tx, \sectxlookup\tdectx{\tcacheid\ty\tf\tx}}}}
      {\tdterm}
      {}
      {(\tcloseddvalue, \tcachedclosedvalues)}
    }{
      \sneval
      {\tdectx}
      {\tlet{\tdy, \tcacheid\ty\tf\tx = \tdf\;\tdx\;\tcacheid\ty\tf\tx}{\tdterm}}
      {(\tcloseddvalue, \tcachedclosedvalues)}
    }

    \infer[\rulename{TDClosureChange}]{
      \sectxlookup\tdectx{\tdf} = \tclosure{\tdectx_\tf}{\tlam{\tdx\,\tcache}{\tdterm_\tf}} \\
      \matchcache
      {\sectxlookup\tdectx{\tcacheid\ty\tf\tx}}
      {\tcache}
      {\tdectx'}
      \\
      \seval{\iectxcons{\tdectx_\tf}{
          \tdx = \sectxlookup\tdectx{\tdx};
          \tdectx'
      }}
      {\tdterm_\tf}{}{(\tcloseddvalue_\ty, \tcachedclosedvalues')} \\
      \seval
      {\iectxcons\tdectx{\tdy = \tcloseddvalue_\ty, \tcacheid\ty\tf\tx = \tcachedclosedvalues'}}
      {\tdterm}{}
      {(\tcloseddvalue, \tcachedclosedvalues)}
    }{
      \sneval
      {\tdectx}
      {\tlet{\tdy, \tcacheid\ty\tf\tx = \tdf\;\tdx\;\tcacheid\ty\tf\tx}{\tdterm}}
      {(\tcloseddvalue, \tcachedclosedvalues)}
    }

    % \\
    % \figsubtitle{Evaluation of cache updates}{
    %   $\sneval{\tectx}{\tupdcache}{\tcachedclosedvalues}$
    % }
    % \\

    % \infer[\rulename{TUpdateEmpty}]{
    % }{
    %   \sneval{\tdectx}{\temptycache}{\temptycache}
    % }

    % \infer[\rulename{TUpdateCachedValue}]{
    %   \tectxlookup\tdectx{\tx, \tdx} = \tclosedvalue, \tcloseddvalue \\
    %   \sneval{\tdectx}{\tupdcache}{\tcachedclosedvalues}
    % }{
    %   \sneval{\tdectx}
    %   {\tcachecons\tupdcache{(\tx\oplus\tdx)}}
    %   {\tcachecons\tcachedclosedvalues(\tclosedvalue\oplus\tcloseddvalue)}
    % }

    % \infer[\rulename{TUpdateSubCache}]{
    %   \tectxlookup\tdectx{\tcacheid\ty\tf\tx} = \tcachedclosedvalues' \\
    %   \sneval{\tdectx}{\tupdcache}{\tcachedclosedvalues}
    % }{
    %   \sneval
    %   {\tdectx}
    %   {\tcachecons\tupdcache{\tcacheid\ty\tf\tx}}
    %   {\tcachecons\tcachedclosedvalues{\tcachedclosedvalues'}}
    % }

    \\
    \figsubtitle{Binding of caches}{
      $\matchcache\tcachedclosedvalues\tcache\tdectx$
    }
    \\

    \infer[\rulename{TMatchEmptyCache}]{
    }{
      \matchcache\temptycache\temptycache\temptycache
    }

    \infer[\rulename{TMatchCachedValue}]{
      \matchcache
      {\tcachedclosedvalues}
      {\tcache}
      {\tdectx}
    }{
      \matchcache
      {\tcachecons\tcachedclosedvalues{\tclosedvalue}}
      {\tcachecons\tcache{\tx}}
      {\tectxcons\tdectx{(\tx = \tclosedvalue)}}
    }

    \infer[\rulename{TMatchSubCache}]{
      \matchcache
      {\tcachedclosedvalues}
      {\tcache}
      {\tdectx}
    }{
      \matchcache
      {\tcachecons\tcachedclosedvalues{\tcachedclosedvalues'}}
      {\tcachecons\tcache{\tcacheid\ty\tf\tx}}
      {\tectxcons\tdectx{(\tcacheid\ty\tf\tx = \tcachedclosedvalues')}}
    }
  \end{mathpar}
  \caption{Target language $\target$ (semantics of change terms and cache updates).}
  \label{fig:target-definition-change-semantics}
\end{figure}


The syntax for caches has three cases: a cache can be empty, a cache
can contain a value or another cache. In other words, a cache is a
tree-like data structure which is isomorphic to an execution trace
containing both immediate values and the execution traces of the
function calls issued during the evaluation.

The syntax for change terms of~$\target$ witnesses the CTS
discipline followed by the derivatives: to determine~$\tdy$, the
derivative of~$\tf$ evaluated at point $\tx$ with change $\tdx$ is
expecting the cache related to the evaluation of~$\ty$ in the base
term. The derivative returns the updated cache which contains the
intermediate results that would be gathered by the evaluation
of~$\tf\,(\tx \oplus \tdx)$. The result term of every change term
returns a cache update $\tupdcache$ in addition to the computed
change.

The syntax for cache updates is almost the same as the one for caches
except each value identifier~$\tx$ of the input cache is adjusted with
its related change~$\tdx$.

\paragraph{Semantics}
An evaluation environment $\tectx$ of $\target$ contains not only
values but also cache values. The syntax for values $\tclosedvalue$
includes closures, tuples, primitives and constants. The syntax for
cache values $\tcachedclosedvalues$ mimicks the one for cache terms.
The evaluation of change terms expects the evaluation
environments~$\tdectx$ to also include bindings for change values.

There are five kinds of change values: closure changes, tuple changes,
literal changes, nil changes and replacements.  Closure changes embed
an environment $\tdectx$ as well as a code pointer for a function
waiting for both a base value $\tx$ and a cache $\tcache$. Notice that
we abusively reuse the same syntax~$\tcache$ to deconstruct and to
construct caches. There other changes are similar to the ones found
in~$\source$.

The base terms of the language are evaluated using a big-step
semantics defined in Figure~\ref{fig:target-definition-base-semantics}.
The judgment~``$\sneval{\tectx}{\tterm}{(\tclosedvalue,
  \tcachedclosedvalues)}$'' is read~``Under the evaluation environment
$\tectx$, the base term $\tterm$ evaluates into a value
$\tclosedvalue$ and a cache $\tcachedclosedvalues$''. The auxiliary
judgment ``$\sneval{\tectx}{\tcache}{\tcachedclosedvalues}$''
specifies the construction of caches.
%
The rule~\rulename{TResult} not only looks into the environment
for the return value~$\tclosedvalue$ but it also evaluates
the returned cache~$\tcache$.
%
The rule~\rulename{TTuple} is similar to the rule of the source
language since no cache is produced by the allocation of a tuple.
%
The rule~\rulename{TClosureCall} works exactly
as~\rulename{SClosureCall} except that the cache value returned by the
closure is bound to the cache identifier $\tcacheid\ty\tf\tx$. The
same remark applies to~\rulename{TPrimitiveCall} with respect
to~\rulename{SPrimitiveCall}.
%
The rule~\rulename{TEmptyCache} evaluates an empty cache term into an
empty cache value. The rule~\rulename{TCacheVar} computes the value of
the cache term~$\tcachecons\tcache\tx$ by appending the value of the
variable~$\tx$ to the cache value~$\tcachedclosedvalues$ computed for
the cache term~$\tcache$. Similarly, the
rule~\rulename{TCacheSubCache} appends the cache value of a cache
named~$\tcacheid\ty\tf\tx$ to the cache value~$\tcachedclosedvalues$
computed for~$\tcache$.

The change terms of the target language are also evaluated using a
big-step semantics defined in
Figure~\ref{fig:target-definition-change-semantics}. The
judgment~``$\sneval{\tdectx}{\tdterm}{(\tcloseddvalue,
  \tcachedclosedvalues)}$'' is read~``Under the evaluation
environment~$\tectx$, the change term $\tdterm$ evaluates into a
change value $\tcloseddvalue$ and an updated cache
$\tcachedclosedvalues$.''. The first auxiliary
judgment~``$\sneval{\tdectx}{\tupdcache}{\tcachedclosedvalues}$''
defines the evaluation of cache update terms. We omit the
rules for this judgment since it is similar to the one for cache terms,
except that cached values are computed by~$\tx \oplus \tdx$, not simply~$\tx$.
%
The final auxiliary
judgment~``$\matchcache\tcachedclosedvalues\tcache\tdectx$'' describes
how a cache pattern $\tcache$ binds a cache
value~$\tcachedclosedvalues$ to produce a change
environment~$\tdectx$.

The rule~\rulename{TDResult} returns the final change value of a
computation as well as a updated cache resulting from the evaluation
of the cache update term~$\tupdcache$.
%
The rule~\rulename{TDTuple} is similar to its counterpart in the
source language, except that no tuple is built for~$\ty$ as it has
already been pushed in the environment by the cache.

As for~$\source$, there are four rules to deal with {\bf
  let}-bindings depending on the shape of~the change bound to~$\tdf$
in the environment.
%
If $\tdf$ is bound to a replacement, the rule~\rulename{TDReplaceCall} applies.
In that case, we reevaluate the function call in the updated
environment~$\newenv{\tdectx}$~(defined similarly as in the source
language). This evaluation leads to a new value~$\tclosedvalue'$
which replaces the original one as well as an updated cache
for~$\tcacheid\ty\tf\tx$.

If $\tdf$ is bound to a nil change and $\tf$ is bound to a closure, the
rule~\rulename{TDClosureNil} applies. This rule mimicks again its
counterpart in the source language passing with the difference that
only the resulting change and the updated cache are bound in the
environment.

If $\tdf$ is bound to a nil change and $\tf$ is bound to a primitive, the
rule~\rulename{TDPrimitiveNil} applies. The derivative of the
primitive is invoked with the value of $\tx$, its change value and the
cache of the original call to this primitive. The semantics of this
derivative is given by a builtin function~$\tddelta\sprim\param$, as
in the source language.

If $\tdf$ is bound to a closure change and $\tf$ is bound to a
closure, the rule~\rulename{TDClosureNil} applies. The body of the
closure change is evaluated under the closure change environment
extended with the value of the formal argument~$\tdx$ and with the
environment resulting from the binding of the original cache value
to the variables occuring in the cache~$\tcache$. This evaluation
leads to a change and an updated cache bound in the environment
to continue with the evaluation of the rest of the term.

% The rule~\rulename{TUpdateEmpty} is the base case for cache
% update.
% %
% The rule~\rulename{TUpdateCachedValue} updates the cached
% value~$\tclosedvalue$ of~$\tx$ with the value of $\tdx$, a
% change~$\tcloseddvalue$, by applying the change application
% primitive~$\oplus$.
% %
% The rule~\rulename{TUpdateSubCache} extracts the value
% of~$\tcacheid\ty\tf\tx$ from the environment~$\tdectx$ to append it to
% the updated cache~$\tcachedclosedvalues$ coming from the evaluation of
% the cache update~$\tupdcache$.

% The rule~\rulename{TMatchEmptyCache} matches an empty cache pattern
% against an empty cache value, which binds nothing.
% %
% The rule~\rulename{TMatchCachedValue} matches a cache which ends with
% value identifier~$\tx$ against a cache value which ends with a
% value~$\tclosedvalue$. The resulting environment binds~$\tx$
% to~$\tclosedvalue$. The rule~\rulename{TMatchSubCache} is similar
% except that it binds a cache identifier instead of value identifier.
% %



\subsection{CTS conversion from \source to \target}
\label{sec:transformation}

\begin{figure}[htb]
  \small
  \newcommand\vskipBeforeCatTitle{\\[-0.2em]}
  \begin{minipage}{\linewidth}
    \begin{alignmath}*{c}
      \begin{array}[t]{rcl}
      \categorytitle{CTS translation of toplevel definitions}{$\compile{\tf = \slam{\tx}{\sterm}}$}
      \compile{\tf = \slam{\tx}{\sterm}}
      & = &
            \tf = \slam{\tx}{\tterm}, \\
      &  &\tdf = \slam{\tdx\, \tcache}{\derive{\tcachecons\temptycache{(\tchange{\tx}{\tdx})}}\sterm} \\
      \side{\textrm{where } (\tcache, \tterm) = \compileterm{\tcachecons\temptycache\tx}\sterm}\\
      \nextline
      %
      \vskipBeforeCatTitle
      \categorytitle{CTS differentiation of terms}{$\derive{\tupdcache}{\sterm}$}
      %
      \derive{\tupdcache}{\slet{\ty = \sapp{\tf}{\tx}}{\sterm}}
      & = &
            \tquote{\tlet{\tdy, \tcacheid{\ty}{\tf}{\tx} = \tdapp{\tdf}{\tdx}{\tcacheid{\ty}{\tf}{\tx}}}{\tterm}}
      \\
      \side{\textrm{where }\tterm = \derive{(\tupdcache\;(\tchange\ty\tdy)\;\tcacheid{\ty}{\tf}{\tx})}{\sterm}} \\
      \nextline

      \derive{\tupdcache}{\slet{\ty = \stuple{\many\tx}}{\sterm}}
      & = &
            \tquote{\tlet{\tdy = \stuple{\many\tdx}}{\tterm}}
      \\
      \side{\textrm{where }\tterm = \derive{(\tupdcache\;(\tchange\ty\tdy))}{\sterm}} \\
      \nextline

      \derive{\tupdcache}{\tx}
      & = &
            \tquote{(\tdx, \tupdcache)}
            \\\nextline\vskipBeforeCatTitle
      %
      \categorytitle{CTS translation of terms}{$\compileterm\tcache\sterm$}
      %
      \compileterm{\tcache}{\slet{\ty = \tapp{\tf}{\tx}}{\sterm}}
      & = &
      (\tcache', \tquote{\tlet{\ty, \tcacheid{\ty}{\tf}{\tx} = \sapp{\tf}{\tx}}{\tterm}}) \\
      \side{\textrm{where }(\tcache', \tterm)
      = \compileterm{(\tcache\;\ty\;\tcacheid{\ty}{\tf}{\tx})}{\sterm}} \\
      \nextline
      %
      \compileterm{\tcache}{\slet{\ty = \stuple{\many\tx}}{\sterm}}
      & = &
      (\tcache', \tquote{\tlet{\ty = \stuple{\many\tx}}{\tterm}}) \\
      \side{\textrm{where }(\tcache', \tterm)
      = \compileterm{(\tcache\;\ty)}{\sterm}} \\
      \nextline
      \compileterm{\tcache}{\tx}
      & = &
            (\tcache, \tquote{(\tx, \tcache)})
    \end{array}
\end{alignmath}
\end{minipage}
\caption{Cache-Transfer Style (CTS) conversion from $\source$ to $\target$.}
\label{fig:differentiation-and-static-caching}
\end{figure}


CTS conversion from $\source$ to $\target$ is
defined in Figure~\ref{fig:differentiation-and-static-caching}. It comprises
CTS differentiation $\mathcal{D}{(\param)}$, from
$\source$ base terms to $\target$ change terms,
and CTS translation $\mathcal{C}(\param)$, from $\source$
$\target$, which is overloaded over top-level
definitions~$\compile{\tf = \slam{\tx}{\sterm}}$ and
terms~$\compileterm{\tcache}{\sterm}$.

By the first rule, $\compile{\param}$ maps each source toplevel definition
``$\tf = \slam{\tx}{\sterm}$'' to the compiled code of the
function $\tf$ and to the derivative~$\tdf$ of~$\tf$ expressed in the
target language~$\target$. These target definitions are generated by a
first call to the compilation function~$\compileterm{\tcachecons\temptycache\tx}{\sterm}$:
it returns both~$\tterm$, the compiled body of $\tf$ and the
cache term~$\tcache$ which contains the names of the intermediate
values computed by the evaluation of $\tterm$. This cache
term~$\tcache$ is used as a cache pattern to define the second
argument of the derivative of $\tf$. That way, we make sure that the
shape of the cache expected by $\tdf$ is consistent with the shape of
the cache produced by $\tf$. Derivative body~$\tdf$ is
computed by derivation call~$\derive{(\tx \oplus \tdx)}{\sterm}$.

CTS translation on terms, $\compileterm{\tcache}{\sterm}$, accepts a term
$\sterm$ and a cache term
$\tcache$. This cache is a fragment of output code: in tail
position ($\sterm = \tx$), it
generates code to return both the result $\tx$ and the cache $\tcache$. When the
transformation visits {\bf let}-bindings, it outputs extra bindings for
caches~$\tcacheid\ty\tf\tx$, and appends all variables newly bound in the output
to the cache used when visiting the {\bf let}-body.

Similarly to $\compileterm{\tcache}{\sterm}$, CTS
derivation~$\derive{\tupdcache}{\sterm}$ accepts a cache update
$\tupdcache$ to return in tail position. While cache terms record \emph{intermediate
results}, cache updates record \emph{result updates}. For {\bf let}-bindings,
to update $\ty$ by change $\tdy$, CTS derivation appends
to $\tupdcache$ term $\ty \oplus \tdy$ to replace $\ty$.

\subsection{Soundness of CTS conversion}
\label{sec:transformation-soundness}

In this section, we outline definitions and main lemmas needed to prove CTS
conversion sound.
The proof is based on a
mostly straightforward simulation in lock-step, but two subtle points emerge.
First, we must relate $\source$ environments that do not contain caches, with
$\target$ environments that do.
Second, while evaluating CTS derivatives, the evaluation environment mixes
caches from the base computation and updated caches computed by the derivatives.

\paragraph{Evaluation commutes with compilation of base terms}

\begin{figure}
  \iftoggle{poplForThesis}{\small}{\vspace{-3em}\footnotesize}
  \newcommand\categorytitle[1]{\\[-0.6em]\multicolumn{3}{r}{\textit{#1}} \\[0.5em]}%
  \iftoggle{poplForThesis}{}{\begin{multicols}{2}}
    \[
    \begin{array}{rcl}
      %
      \categorytitle{CTS translation of values $\compile\sclosedvalue$}
      %
    \compile{\sclosure\sectx{\slam{\tx}{\sterm}}}
      & = &
            \tclosure{\compile\sectx}{\slam{\tx}{\tterm}} \\
      \side{\textrm{where } (\tcache, \tterm) = \compileterm{\tcachecons\temptycache\tx}\sterm}\\
      \\
      \compile{\sconst}
      & = &
            \sconst \\
      \nextline
      %
      \categorytitle{CTS translation of change values $\compile\icloseddvalue$}
      %
      \compile{\sclosure\sdectx{\slam{\tx\,\tdx}{\iderive\sterm}}}
      & = &
            \tclosure{\compile\sdectx}{\slam{\tdx\, \tcache}{\derive{\tcachecons\temptycache{(\tchange{\tx}{\tdx})}}\sterm}} \\
      \side{\textrm{where } (\tcache, \tterm) = \compileterm{\tcachecons\temptycache\tx}\sterm}\\
      \\
      \compile{\replace\sclosedvalue}
      & = &
            \replace{\compile\sclosedvalue}
      \\
      \nextline
      \compile{\tdconst}
      & = &
          \tdconst
    \end{array}
  \]

  \[
    \begin{array}{rcl}
      \nextline
      \categorytitle{CTS translation of value environments $\compile\sectx$}
      %
      \compile{\sectxempty}
      & = &
            \tectxempty
      \\
      \nextline
      \compile{\sectxcons{\sectx}{\tx = \sclosedvalue}}
      & = &
            \sectxcons{\compile\sectx}{\tx = \compile\sclosedvalue}
      \\
      \nextline
      \categorytitle{CTS translation of change environments $\compile\iectx$}
      %
      \compile{\sectxempty}
      & = &
            \tectxempty
      \\
      \nextline
      \compile{\sectxcons{\iectx}{\tx = \sclosedvalue, \tdx = \icloseddvalue}}
      & = &
            \sectxcons{\compile\iectx}{\tx = \compile\sclosedvalue, \tdx = \compile\icloseddvalue}
    \end{array}
  \]
  \iftoggle{poplForThesis}{}{\end{multicols}}
\caption{Extending CTS translation to values, change values, environments and change environments.}
\label{fig:differentiation-and-static-caching-continued}
\end{figure}


Figure~\ref{fig:differentiation-and-static-caching-continued}
extends CTS translation to values, change values, environments and
change environments. CTS translation commutes with our semantics, as shown by
next lemma:

\begin{lemma}[Evaluation commutes with CTS translation of base terms]
  \-\\
  For all $\sectx, \sterm$ and $\sclosedvalue$, such that
  $\seval\sectx\sterm{}\sclosedvalue$, and
  for all $\tcache$, there exists $\tcachedclosedvalues$,
  $\seval
  {\compile{\sectx}}{\compileterm\tcache\sterm}
  {}
  {(\compile{\sclosedvalue}, \tcachedclosedvalues)}$.
\end{lemma}

Stating a corresponding lemma for CTS translation of derived terms is trickier.
If the derivative of $\sterm$ evaluates correctly in some environment
(that is $\seval{\iectx}{\iderive\sterm}{}{\icloseddvalue}$), CTS derivative
$\derive\tupdcache\sterm$ cannot be evaluated in environment $\compile\iectx$.
A CTS derivative can only evaluate against environments containing cache values
from the base computation, but no cache values appear in $\compile\iectx$!

As a fix we enrich~$\compile\iectx$ with the values of a cache~$\tcache$, using
the judgment defined in Figure~\ref{fig:envwithcache}.
Judgment~$\envwithcache\tectx\tcache{\tectx'}$ is read ``The target
change environment~$\tectx'$ extends~$\tectx$ with the values of
cache~$\tcache$.'' We introduce a similar
judgment~$\newenvwithcache\tectx\tcache{\tectx'}$ to extend a change
environment~$\tectx$ with the updated values of the cache~$\tcache$.
The rules of this judgment are omitted for lack of space but they
can be obtained by replacing $\oldenv\tectx$ with $\newenv\tectx$
in the rule for cache computation in Figure~\ref{fig:envwithcache}.

\begin{lemma}[Evaluation commutes with compilation of derivatives]
  \-\\
  Let $\tcache$ be such that
  $(\tcache, \_) = \compileterm{\bullet}{\sterm}$.
  For all $\iectx, \sterm$ and $\icloseddvalue$,
  if
  $\seval{\iectx}{\iderive\sterm}{}{\icloseddvalue}$,
  and $\envwithcache{\compile\iectx}{\tcache}{\tdectx}$,
  then
  $\seval{\tdectx}{\derive{\tupdcache}{\sterm}}{}{(\compile\icloseddvalue, \tcachedclosedvalues)}$.
\end{lemma}

Notice that the proof of this lemma is not immediate since during the
evaluation of~$\derive{\tupdcache}{\sterm}$ the new caches replace the
old caches. In the Coq development, we enforce a physical separation
between the part of the environment containing the old caches and the
one containing the new caches, and we maintain the invariant that the
second part of the environment corresponds to the remaining part of the
term.

\begin{figure}
  \footnotesize

  \begin{mathpar}
    \infer{}{
      \envwithcache{\tectx}{\temptycache}{\tectx}
    }

    \infer{
      \envwithcache{\tectx}{\tcache}{\tectx'}
    }{
      \envwithcache{\tectx}{\tcachecons\tcache\tx}{\tectx'}
    }

    \infer{
      \envwithcache{\tectx}{\tcache}{\tectx'}
      \\
      \sneval{\oldenv\tectx}{\tlet{\ty, \tcacheid\ty\tf\tx = \tf\tx}{(\ty, \tcacheid\ty\tf\tx)}}{(\_, \tcachedclosedvalues)}
    }{
      \envwithcache{\tectx}
      {\tcachecons\tcache{\tcacheid\ty\tf\tx}}
      {\tectxcons{\tectx'}{\tcacheid\ty\tf\tx = \tcachedclosedvalues}}
    }
\end{mathpar}

\caption{Extension of an environment with cache values $\envwithcache{\tectx}{\tcache}{\tectx'}$}
\label{fig:envwithcache}
\end{figure}

\paragraph{Soundness of the transformation}
Finally, we can state soundness of CTS differentiation relative to differentiation.
The theorem says that (a) the CTS derivative $\derive\tcache\sterm$ computes the
CTS translation $\compile{\icloseddvalue}$ of the
change computed by the standard derivative $\iderive\sterm$; (b) the updated
cache $\tcachedclosedvalues'$ produced by
the CTS derivative coincides with the cache produced by the CTS-translated base
term $\tterm$ in the updated environment $\newenv{\tectx'}$.
Since we require a correct cache via condition (b), we can use this cache
to invoke the CTS derivative on further changes, as described
in~\cref{sec:motivating-example}.

\begin{theorem}[Soundness of differentiation with static caching wrt differentiation]
  \-\\
  \label{thm:soundness-compiled-changesfinal}
  Let $\tcache$ and $\tterm$ be such that 
  $(\tcache, \tterm) = \compileterm{\bullet}{\sterm}$.
  For all $\iectx, \sterm$ and $\icloseddvalue$,
  $\seval{\iectx}{\iderive\sterm}{}{\icloseddvalue}$,
  for $\tectx$ and $\tectx'$ such that
  $\envwithcache{\compile\iectx}{\tcache}{\tectx}$
  and $\newenvwithcache{\compile\iectx}{\tcache}{\tectx'}$,
  $\seval{\oldenv\tectx}{\tterm}{}{(\tclosedvalue, \tcachedclosedvalues)}$ and
  $\seval{\newenv{\tectx'}}{\tterm}{}{(\tclosedvalue', \tcachedclosedvalues')}$,
  then
  $
  \seval{\tectx}{\derive\tcache\sterm}{}{(\compile{\icloseddvalue}, \tcachedclosedvalues')}
  $.
\end{theorem}
