\section{Formalization}
\label{sec:formalization}
In this section, we formalize differentiation with static caching and formally
prove its soundness with respect to differentiation alone. We furthermore
present a novel proof of correctness for differentiation itself.

To simplify the proof, we encode many invariants of input and output programs by
defining input and output languages with a suitably restricted abstract syntax:
Restricting the input language simplifies the transformations, while restricting
the output languages simplifies the semantics. Since base programs and
derivatives have different shapes, we define different syntactic categories of
\emph{base terms} and \emph{change terms}.
% following the example of...

We define and prove sound three transformations, across two languages: first, we
present a variant of differentiation without caching (as defined by
\citet{CaiEtAl2014ILC}), going from base terms of language $\source$ to change
terms of $\source$, and we prove it sound with respect to non-incremental
evaluation.
%
Second, we define differentiation with static caching as a pair of
transformations going from base terms of $\source$: the first transformation
produces CTS versions of base functions as base terms in $\target$,
and the second transformation produces CTS derivatives as change terms
in $\target$.

As source language for differentiation with static caching,
we use a core language named~$\source$.  This
language is a common target of a compiler front-end for a functional
programming language: in this language, terms are written in
$\lambda$-lifted and in a variant of A-normal form~\cite{sabry1993reasoning}, hence every
intermediate result is named. Hence intermediate results can be reused
and stored in a cache.

The target language for differentiation with static caching is
named~$\target$. Programs in this target language are also
$\lambda$-lifted and in a variant of A-normal form. But additionally, in these
programs every toplevel function $\tf$ produces a cache which is to be
conveyed to the derivatives of $\tf$.

The syntax and semantics of the source and target languages are
presented in Sections~\ref{sec:sourcelanguage}
and~\ref{sec:targetlanguage}. The transformation is described in
Section~\ref{sec:transformation} and its proof of soundness
is given in Section~\ref{sec:transformation-soundness}. Most of
the formalization have been mechanized in Coq (the proofs of some
straightforward lemmas are left for future work).


\input{\poplPath{source-language}}
\input{\poplPath{static-differentiation}}
\input{\poplPath{static-differentiation-soundness}}
\input{\poplPath{target-language}}
\input{\poplPath{transformation}}
\input{\poplPath{soundness}}