\section{Formalization}
\label{sec:formalization}
In this section, we formalize cache-transfer-style (CTS) differentiation and formally
prove its soundness with respect to differentiation. We furthermore
present a novel proof of correctness for differentiation itself.

To simplify the proof, we encode many invariants of input and output programs by
defining input and output languages with a suitably restricted abstract syntax:
Restricting the input language simplifies the transformations, while restricting
the output languages simplifies the semantics. Since base programs and
derivatives have different shapes, we define different syntactic categories of
\emph{base terms} and \emph{change terms}.
% following the example of...

We define and prove sound three transformations, across two languages: first, we
present a variant of differentiation (as defined by
\citet{CaiEtAl2014ILC}), going from base terms of language $\source$ to change
terms of $\source$, and we prove it sound with respect to non-incremental
evaluation.
%
Second, we define CTS conversion as a pair of
transformations going from base terms of $\source$: CTS translation
produces CTS versions of base functions as base terms in $\target$,
and CTS differentiation produces CTS derivatives as change terms
in $\target$.

As source language for CTS differentiation,
we use a core language named~$\source$.  This
language is a common target of a compiler front-end for a functional
programming language: in this language, terms are written in
$\lambda$-lifted and in A'-normal form (A'NF), so that every
intermediate result is named and can be reused
and stored in a cache by CTS conversion (as described in
\cref{sec:cts-motivation}).

The target language for CTS differentiation is
named~$\target$. Programs in this target language are also
$\lambda$-lifted and in A'NF. But additionally, in these
programs every toplevel function $\tf$ produces a cache which is to be
conveyed to the derivatives of $\tf$.

The rest of this section is organized as follows.
\Cref{sec:sourcelanguage} presents
syntax and semantics of source language~$\source$.
\Cref{sec:stat-diff-source} defines differentiation, and \cref{sec:sound-derive}
proves it correct.
\Cref{sec:targetlanguage} presents
syntax and semantics of target language~$\source$.
\Cref{sec:transformation} define CTS conversion, and
\cref{sec:transformation-soundness} proves it correct.
Most of the formalization has been mechanized in Coq (the proofs of some
straightforward lemmas are left for future work).


\input{\poplPath{source-language}}
\input{\poplPath{static-differentiation}}
\input{\poplPath{static-differentiation-soundness}}
\input{\poplPath{target-language}}
\input{\poplPath{transformation}}
\input{\poplPath{soundness}}