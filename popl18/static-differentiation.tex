\subsection{Static differentiation in \source}
Previous work~\citep{CaiEtAl2014ILC} defines differentiation for simply-typed
$\lambda$-calculus terms.
Figure~\ref{fig:differentiation} shows differentiation for $\source$'s syntax.

\input{\poplPath{differentiation}}
Differentiating a base term~$\sterm$ produces a change
term~$\iderive{\sterm}$, its \emph{derivative}.
%
Differentiating final result variable~$\tx$ produces its change variable~$\tdx$.
Differentiation copies each binding of an intermediate result $\ty$ to the
output and adds a new bindings for its change $\tdy$.
%
If $\ty$ is bound to tuple~$\stuple{\many\tx}$, then $\tdy$ will
be bound to the change tuple $\stuple{\many\tdx}$.
If $\ty$ is bound to function application $\sapp\tf{\tx}$, then $\tdy$ will be
bound to the application of function change $\tdf$ to input $\tx$ and its change
$\tdx$.

Evaluating $\iderive{\sterm}$ recomputes all intermediate results
computed by $\sterm$. This recomputation will be avoided through cache-transfer
style in section \cref{sec:transformation}.

The original transformation for static differential of
$\lambda$-terms~\citep{CaiEtAl2014ILC} has three cases which
we recall here:
\[
  \begin{array}{rcl}
    \oderive\tx & = & \tdx \\
    \oderive{t \, u} & = & \oderive{t}\,u\,\oderive{u} \\
    \oderive{\lambda \tx. t} & = & \lambda \tx\,\tdx. \oderive{t}
  \end{array}
\]

Even though the first two cases of the original transformation are
easily mapped into the two cases of our variant, one may ask where the
third case is realized now. Actually, this third case occurs while we
transform the initial environment. Indeed, we will assume that the
closures of the environment of the source program have been adjoined a
derivative. More formally, we suppose that the derivative of $\sterm$
is evaluated under an environment~$\iderive\sectx$ obtained as
follows:
\[
  \begin{array}{rcl}
    \iderive{\bullet} & = & \bullet \\
    \iderive{\sectxcons\sectx{\tf = \sclosure{\sectx_f}{\lambda \tx. \sterm}}}
    &=&
    \sectxcons{\iderive\sectx}{\tf = \sclosure{\sectx_f}{\lambda \tx. \sterm},
    \tdf = \sclosure{\iderive{\sectx_f}}{\lambda \tx\,\tdx. \iderive{\sterm}}}
    \\
    \iderive{\sectxcons\sectx{\tx = \sclosedvalue}}
    &=&
    \sectxcons{\iderive\sectx}{\tx = \sclosedvalue, \tdx = \inil}
    \qquad \textrm{(If $\sclosedvalue$ is not a closure.)}
  \end{array}
\]