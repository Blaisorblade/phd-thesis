\subsection{The target language \target}
\label{sec:targetlanguage}

In this section, we present the target language of a transformation
that extends static differentiation with CTS conversion.
As said earlier, the functions of $\target$ not only compute their
output but they also returns a cache containing the intermediate
values that have contributed to that output. This cache is transmitted
to derivatives to build changes on top of intermediate values with no
recomputation and derivatives update the caches according to their
input changes.

\paragraph{Syntax}
The syntax of $\target$ is defined in
Figure~\ref{fig:target-definition-syntax}. The base terms of $\target$
follow the same shape as the $A$-normal $\lambda$-lifted terms of
$\source$ except that a $\bf let$-binding for a function
application~$\tf\,\tx$ now introduces an extra \textit{cache
  identifier}~$\tcacheid\ty\tf\tx$ in replacement for the
output~$\ty$. The syntax of cache identifiers is non-standard: it can
be seen as a triple that refers to the value identifiers $\tf$, $\tx$
and $\ty$. Hence, an $\alpha$-renaming of one of these three
identifiers must refresh the cache identifier accordingly. The syntax
$(\tx, \tcache)$ for the result term makes explicit that a cache
$\tcache$ is systematically returned.
% A similar alpha-renaming issue must be made about the formal
% cache argument of caching-derivatives. In the Coq development,
% the positional approach to binders makes the problem disappear.

\input{\poplPath{target-definition}}

The syntax for caches has three cases: a cache can be empty, or it can
prepend a value or a cache variable to a cache.
In other words, a cache is a
tree-like data structure which is isomorphic to an execution trace
containing both immediate values and the execution traces of the
function calls issued during the evaluation.

The syntax for change terms of~$\target$ witnesses the CTS
discipline followed by the derivatives: to determine~$\tdy$, the
derivative of~$\tf$ evaluated at point $\tx$ with change $\tdx$
expects the cache produced by evaluating ~$\ty$ in the base
term. The derivative returns the updated cache which contains the
intermediate results that would be gathered by the evaluation
of~$\tf\,(\tx \oplus \tdx)$. The result term of every change term
returns a cache update $\tupdcache$ in addition to the computed
change.

The syntax for cache updates is almost the same as the one for caches
except each value identifier~$\tx$ of the input cache is adjusted with
its corresponding change~$\tdx$.

\paragraph{Semantics}
An evaluation environment $\tectx$ of $\target$ contains not only
values but also cache values. The syntax for values $\tclosedvalue$
includes closures, tuples, primitives and constants. The syntax for
cache values $\tcachedclosedvalues$ mimics the one for cache terms.
The evaluation of change terms expects the evaluation
environments~$\tdectx$ to also include bindings for change values.

There are five kinds of change values: closure changes, tuple changes,
literal changes, nil changes and replacements.  Closure changes embed
an environment $\tdectx$ and a code pointer for a function,
waiting for both a base value $\tx$ and a cache $\tcache$. By abuse of notation,
we reuse the same syntax~$\tcache$ to both deconstruct and
construct caches. Other changes are similar to the ones found
in~$\source$.

Base terms of the language are evaluated using a big-step
semantics defined in Figure~\ref{fig:target-definition-base-semantics}.
Judgment~``$\sneval{\tectx}{\tterm}{(\tclosedvalue,
  \tcachedclosedvalues)}$'' is read~``Under evaluation environment
$\tectx$, base term $\tterm$ evaluates to value
$\tclosedvalue$ and cache $\tcachedclosedvalues$''. Auxiliary
judgment ``$\sneval{\tectx}{\tcache}{\tcachedclosedvalues}$''
evaluates cache terms into cache values.
%
Rule~\rulename{TResult} not only looks into the environment
for the return value~$\tclosedvalue$ but it also evaluates
the returned cache~$\tcache$.
%
Rule~\rulename{TTuple} is similar to the rule of the source
language since no cache is produced by the allocation of a tuple.
%
Rule~\rulename{TClosureCall} works exactly
as~\rulename{SClosureCall} except that the cache value returned by the
closure is bound to cache identifier $\tcacheid\ty\tf\tx$.
In the same way, \rulename{TPrimitiveCall} resembles \rulename{SPrimitiveCall}
but also binds $\tcacheid\ty\tf\tx$.
%
Rule~\rulename{TEmptyCache} evaluates an empty cache term into an
empty cache value. Rule~\rulename{TCacheVar} computes the value of
cache term~$\tcachecons\tcache\tx$ by appending the value of
variable~$\tx$ to cache value~$\tcachedclosedvalues$ computed for
cache term~$\tcache$. Similarly,
rule~\rulename{TCacheSubCache} appends the cache value of a cache
named~$\tcacheid\ty\tf\tx$ to the cache value~$\tcachedclosedvalues$
computed for~$\tcache$.

Change terms of the target language are also evaluated using a
big-step semantics, defined in
\cref{fig:target-definition-change-semantics}.
Judgment~``$\sneval{\tdectx}{\tdterm}{(\tcloseddvalue,
  \tcachedclosedvalues)}$'' is read~``Under evaluation
environment~$\tectx$, change term $\tdterm$ evaluates to
change value $\tcloseddvalue$ and updated cache
$\tcachedclosedvalues$''. The first auxiliary
judgment~``$\sneval{\tdectx}{\tupdcache}{\tcachedclosedvalues}$''
defines evaluation of cache update terms. We omit the
rules for this judgment since it is similar to the one for cache terms,
except that cached values are computed by~$\tx \oplus \tdx$, not simply~$\tx$.
%
The final auxiliary
judgment~``$\matchcache\tcachedclosedvalues\tcache\tdectx$'' describes a limited
form of pattern matching used by CTS derivatives: namely,
how a cache pattern $\tcache$ matches a cache
value~$\tcachedclosedvalues$ to produce a change
environment~$\tdectx$.

Rule~\rulename{TDResult} returns the final change value of a
computation as well as a updated cache resulting from the evaluation
of the cache update term~$\tupdcache$.
%
The rule~\rulename{TDTuple} is similar to its counterpart in the
source language, except that no tuple is built for~$\ty$ as it has
already been pushed in the environment by the cache.

As for~$\source$, there are four rules to deal with {\bf
  let}-bindings depending on the shape of~the change bound to~$\tdf$
in the environment.
%
If $\tdf$ is bound to a replacement, the rule~\rulename{TDReplaceCall} applies.
In that case, we reevaluate the function call in the updated
environment~$\newenv{\tdectx}$~(defined similarly as in the source
language). This evaluation leads to a new value~$\tclosedvalue'$
which replaces the original one as well as an updated cache
for~$\tcacheid\ty\tf\tx$.

If $\tdf$ is bound to a nil change and $\tf$ is bound to a closure, the
rule~\rulename{TDClosureNil} applies. This rule mimicks again its
counterpart in the source language passing with the difference that
only the resulting change and the updated cache are bound in the
environment.

If $\tdf$ is bound to a nil change and $\tf$ is bound to a primitive, the
rule~\rulename{TDPrimitiveNil} applies. The derivative of the
primitive is invoked with the value of $\tx$, its change value and the
cache of the original call to this primitive. The semantics of this
derivative is given by a builtin function~$\tddelta\sprim\param$, as
in the source language.

If $\tdf$ is bound to a closure change and $\tf$ is bound to a
closure, the rule~\rulename{TDClosureNil} applies. The body of the
closure change is evaluated under the closure change environment
extended with the value of the formal argument~$\tdx$ and with the
environment resulting from the binding of the original cache value
to the variables occuring in the cache~$\tcache$. This evaluation
leads to a change and an updated cache bound in the environment
to continue with the evaluation of the rest of the term.

% The rule~\rulename{TUpdateEmpty} is the base case for cache
% update.
% %
% The rule~\rulename{TUpdateCachedValue} updates the cached
% value~$\tclosedvalue$ of~$\tx$ with the value of $\tdx$, a
% change~$\tcloseddvalue$, by applying the change application
% primitive~$\oplus$.
% %
% The rule~\rulename{TUpdateSubCache} extracts the value
% of~$\tcacheid\ty\tf\tx$ from the environment~$\tdectx$ to append it to
% the updated cache~$\tcachedclosedvalues$ coming from the evaluation of
% the cache update~$\tupdcache$.

% The rule~\rulename{TMatchEmptyCache} matches an empty cache pattern
% against an empty cache value, which binds nothing.
% %
% The rule~\rulename{TMatchCachedValue} matches a cache which ends with
% value identifier~$\tx$ against a cache value which ends with a
% value~$\tclosedvalue$. The resulting environment binds~$\tx$
% to~$\tclosedvalue$. The rule~\rulename{TMatchSubCache} is similar
% except that it binds a cache identifier instead of value identifier.
% %


