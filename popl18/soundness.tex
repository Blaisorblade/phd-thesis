\subsection{Soundness of CTS conversion}
\label{sec:transformation-soundness}

In this section, we outline definitions and main lemmas needed to prove CTS
conversion sound.
The proof is based on a
mostly straightforward simulation in lock-step, but two subtle points emerge.
First, we must relate $\source$ environments that do not contain caches, with
$\target$ environments that do.
Second, while evaluating CTS derivatives, the evaluation environment mixes
caches from the base computation and updated caches computed by the derivatives.

\paragraph{Evaluation commutes with CTS conversion}

\input{\poplPath{differentiation-and-static-caching-continued}}

Figure~\ref{fig:differentiation-and-static-caching-continued}
extends CTS translation to values, change values, environments and
change environments. CTS translation of base terms commutes with our semantics, as shown by
next lemma:

\begin{lemma}[Evaluation commutes with CTS translation]
  For all $\sectx, \sterm$ and $\sclosedvalue$, such that
  $\seval\sectx\sterm{}\sclosedvalue$, and
  for all $\tcache$, there exists $\tcachedclosedvalues$,
  $\seval
  {\compile{\sectx}}{\compileterm\tcache\sterm}
  {}
  {(\compile{\sclosedvalue}, \tcachedclosedvalues)}$.
\end{lemma}

Stating a corresponding lemma for CTS translation of derived terms is trickier.
If the derivative of $\sterm$ evaluates correctly in some environment
(that is $\seval{\iectx}{\iderive\sterm}{}{\icloseddvalue}$), CTS derivative
$\derive\tupdcache\sterm$ cannot be evaluated in environment $\compile\iectx$.
A CTS derivative can only evaluate against environments containing cache values
from the base computation, but no cache values appear in $\compile\iectx$!

%\pg{This consumes and produces change environments!}
As a fix, we enrich~$\compile\iectx$ with the values of a cache~$\tcache$, using
the pair of judgments $\envwithcache\tectx\tcache{\tectx'}$ (for $i = 1, 2$) defined in \cref{fig:envwithcache}.
Judgment~$\oldenvwithcache\tectx\tcache{\tectx'}$ is read ``The target
change environment~$\tectx'$ extends~$\tectx$ with the original values of
cache~$\tcache$.''
Analogously, judgment~$\newenvwithcache\tectx\tcache{\tectx'}$ is read ``The
target change environment~$\tectx'$ extends~$\tectx$ with the updated values of
cache~$\tcache$.''
Since $\tcache$ is essentially a list of variables containing no values, cache
values in $\tectx'$ must be computed from $\env\tectx$.

\begin{lemma}[Evaluation commutes with CTS differentiation]
  Let $\tcache$ be such that
  $(\tcache, \_) = \compileterm{\bullet}{\sterm}$.
  For all $\iectx, \sterm$ and $\icloseddvalue$,
  if
  $\seval{\iectx}{\iderive\sterm}{}{\icloseddvalue}$,
  and $\oldenvwithcache{\compile\iectx}{\tcache}{\tdectx}$,
  then
  $\seval{\tdectx}{\derive{\tupdcache}{\sterm}}{}{(\compile\icloseddvalue, \tcachedclosedvalues)}$.
\end{lemma}

The proof of this lemma is not immediate, since during the
evaluation of~$\derive{\tupdcache}{\sterm}$ the new caches replace the
old caches. In the Coq development, we enforce a physical separation
between the part of the environment containing the old caches and the
one containing the new caches, and we maintain the invariant that the
second part of the environment corresponds to the remaining part of the
term.

\begin{figure}
  \footnotesize

  \begin{mathpar}
    \infer{}{
      \envwithcache{\tectx}{\temptycache}{\tectx}
    }

    \infer{
      \envwithcache{\tectx}{\tcache}{\tectx'}
    }{
      \envwithcache{\tectx}{\tcachecons\tcache\tx}{\tectx'}
    }

    \infer{
      \envwithcache{\tectx}{\tcache}{\tectx'}
      \\
      \sneval{\env\tectx}{\tlet{\ty, \tcacheid\ty\tf\tx = \tapp\tf\tx}{(\ty, \tcacheid\ty\tf\tx)}}{(\_, \tcachedclosedvalues)}
    }{
      \envwithcache{\tectx}
      {\tcachecons\tcache{\tcacheid\ty\tf\tx}}
      {\tectxcons{\tectx'}{\tcacheid\ty\tf\tx = \tcachedclosedvalues}}
    }
\end{mathpar}

\caption{Extension of an environment with cache values
  $\envwithcache{\tectx}{\tcache}{\tectx'}$ (for $i = 1, 2$).}
\label{fig:envwithcache}
\end{figure}

\paragraph{Soundness of CTS conversion}
Finally, we can state soundness of CTS differentiation relative to differentiation.
The theorem says that (a) the CTS derivative $\derive\tcache\sterm$ computes the
CTS translation $\compile{\icloseddvalue}$ of the
change computed by the standard derivative $\iderive\sterm$; (b) the updated
cache $\newtcachedclosedvalues$ produced by
the CTS derivative coincides with the cache produced by the CTS-translated base
term $\tterm$ in the updated environment $\newenv{\new\tectx}$.
Since we require a correct cache via condition (b), we can use this cache
to invoke the CTS derivative on further changes, as described
in~\cref{sec:motivating-example}.

\begin{theorem}[Soundness of CTS differentiation wrt differentiation]
  \label{thm:soundness-compiled-changesfinal}
  Let $\tcache$ and $\tterm$ be such that 
  $(\tcache, \tterm) = \compileterm{\bullet}{\sterm}$.
  For all $\iectx, \sterm$ and $\icloseddvalue$,
  $\seval{\iectx}{\iderive\sterm}{}{\icloseddvalue}$,
  for $\old\tectx$ and $\new\tectx$ such that
  $\oldenvwithcache{\compile\iectx}{\tcache}{\old\tectx}$
  and $\newenvwithcache{\compile\iectx}{\tcache}{\new\tectx}$,
  $\seval{\oldenv{\old\tectx}}{\tterm}{}{(\old\tclosedvalue, \oldtcachedclosedvalues)}$ and
  $\seval{\newenv{\new\tectx}}{\tterm}{}{(\new\tclosedvalue, \newtcachedclosedvalues)}$,
  then
  $
  \seval{\old\tectx}{\derive\tcache\sterm}{}{(\compile{\icloseddvalue}, \newtcachedclosedvalues)}
  $.
\end{theorem}
