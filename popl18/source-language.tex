\subsection{The source language \source}
\label{sec:sourcelanguage}

\begin{figure}[htb]
  \footnotesize
%\begin{multicols}{2}
  \[
    \begin{array}{rclr}
    \multicolumn{4}{r}{\syntaxclass{Base terms}} \\
    \sterm
    & ::= & \slet{\ty = \sapp{\tf}{\tx}}{\sterm}
    & \rem{Call} \\
    & \mid & \slet{\ty = \stuple{\many\tx}}\sterm
    & \rem{Tuple} \\
    & \mid & \tx
    & \rem{Result} \\
      \nextline
    \multicolumn{4}{r}{\syntaxclass{Change terms}} \\
    \idterm
    & ::= 
    & \ilet{\ty = \iapp{\tf}{\tx}, \tdy = \idapp{\tdf}{\tx\,\tdx}}{\idterm}
    & \rem{Call} \\
    & \mid & \slet{\ty = \stuple{\many\tx}, \tdy = \stuple{\many\tdx}}\idterm
    & \rem{Tuple} \\
    & \mid & \tdx
    & \rem{Result} \\
    \nextline
    \multicolumn{4}{r}{\syntaxclass{Closed values}} \\
    \sclosedvalue
    &::=& \sclosure\sectx{\slam{\tx}{\sterm}}
    & \rem{Closure} \\
    & \mid & \stuple{\many\sclosedvalue}
    & \rem{Tuple} \\
    & \mid & \sconst
    & \rem{Literal} \\
    & \mid & \sprim
    & \rem{Primitive} \\
    \nextline
    \multicolumn{4}{r}{\syntaxclass{Value environments}} \\
    \sectx
    & ::= & \sectxcons\sectx{\tx = \sclosedvalue}
    & \rem{Value binding} \\
    & \mid & \sectxempty
    & \rem{Empty} \\
    & \nextline
    \multicolumn{4}{r}{\syntaxclass{Change values}} \\
    \icloseddvalue
    & ::= & \iclosure\iectx{\ilam{\tx\,\tdx}{\idterm}}
    & \rem{Closure} \\
    & \mid & \stuple{\many\icloseddvalue}
    & \rem{Tuple} \\
    & \mid & \idconst
    & \rem{Literal} \\
    & \mid & \replace\sclosedvalue
    & \rem{Replace} \\
    & \mid & \inil
    & \rem{Nil} \\
    \nextline
    \multicolumn{4}{r}{\syntaxclass{Change environments}} \\
    \iectx
    & ::= 
    & \iectxcons{\iectx}{\tx = \sclosedvalue, \tdx = \icloseddvalue}
    & \rem{Binding} \\
    & \mid & \iectxempty
    & \rem{Empty} \\
    \nextline
    n, k & \in & \mathbb{N} & \syntaxclass{Step indexes} \\
    \end{array}
  \]
%\end{multicols}
\caption{Source language $\source$ (syntax).}
  \label{fig:source-definition}
\end{figure}


\paragraph{Syntax}
The syntax of~$\source$ is given in
Figure~\ref{fig:source-definition}. Our source language allows
representing both \emph{base} terms~$\sterm$ and \emph{change}
terms~$\idterm$.

Our syntax for base terms~$\sterm$ represents $\lambda$-lifted
programs in A'-normal form (\cref{sec:motivating-example}).
So a term can be either a bound variable~$\tx$, or a {\bf let}
binding~$\ty$ in subterm~$\sterm$ to either a new
tuple~$\stuple{\many\tx}$ ($\slet{\ty = \stuple{\many\tx}}\sterm$), or
the result of calling function~$\tf$ on argument~$\tx$
($\slet{\ty = \sapp{\tf}{\tx}}{\sterm}$).  Both~$\tf$ and~$\tx$ are
variables to be looked up in the environment.  Terms cannot contain
$\lambda$-abstractions as they have been $\lambda$-lifted to top-level
definitions, which we encode as closures in the initial environments.
%
Unlike standard ANF we add no special syntax for function
calls in tail position (see \cref{sec:tail-calls} for a discussion
about this limitation).
%
We often need to inspect the result of
a function call~$\tf\tx$, which is not a valid term in our syntax.
To alleviate this, we write~``$\scall\tf\tx$''
for ``$\slet{\ty = \tf\;\tx}\ty$'' where $\ty$ is
chosen fresh.

Our syntax for change terms~$\idterm$ mimicks the syntax for base
terms except that (i) each function call is immediately followed by
the evaluation of its derivative and (ii) the final value returned by
a change term is a change variable~$\tdx$. As we will see, this
\emph{adhoc} syntax of change terms is tailored to capture the
programs produced by differentiation.  We only allows
$\alpha$-renamings that maintain the invariant that the definition
of~$\tx$ is immediately followed by the definition of~$\tdx$: if~$\tx$
is renamed to~$\ty$ then $\tdx$ must be renamed to~$\tdy$.

\paragraph{Semantics}

The closed values for the base terms are composed of closures, tuples
of values, constants and primitives. A closure is a pair of an
evaluation environment~$\sectx$ and a $\lambda$-abstraction closed
with respect to~$\sectx$. The set of available constants is left
abstract. It may contain usual first-order constants like integers. We
also leave abstract the primitives $\sprim$ like
\textbf{\texttt{if-then-else}} or projections of tuple components. As
usual, environments~$\sectx$ map variables to closed values. With no
loss of generality, we assume that all bound variables are distinct.

\begin{figure}
  \footnotesize
  \begin{mathpar}
    \infer[\rulename{SVar}]{
    }{
      \seval\sectx\tx{1}{\sectxlookup\sectx\tx}
    }

    \infer[\rulename{STuple}]{
      \seval{\sectxcons{\sectx}{\ty = \stuple{\sectxlookup\sectx{\many\tx}}}}{\sterm}{n}{\sclosedvalue}
    }{
      \seval\sectx{\slet{\ty = \stuple{\many\tx}}{\sterm}}{n + 1}{\sclosedvalue}
    }

    \infer[\rulename{SPrimitiveCall}]{
      \sectxlookup\sectx\tf = \sprim
      \\
      \seval{\sectxcons{\sectx}{\ty = \tdelta\sprim{\sectxlookup\sectx\tx}}}{\sterm}{n}{\sclosedvalue}
    }{
      \seval\sectx{\slet{\ty = \sapp\tf\tx}{\sterm}}{n + 1}{\sclosedvalue}
    }

    \infer[\rulename{SClosureCall}]{
      \sectxlookup\sectx\tf = \sclosure{\sectx_\tf}{\slam{\tx}{\sterm_\tf}}
      \\
      \seval{\sectxcons{\sectx_\tf}{\tx = \sectxlookup\sectx\tx}}{\sterm_\tf}{n}{\sclosedvalue_\ty}
      \\
      \seval{\sectxcons{\sectx}{\ty = \sclosedvalue_\ty}}{\sterm}{m}{\sclosedvalue}
    }{
      \seval\sectx{\slet{\ty = \sapp\tf\tx}{\sterm}}{n + m + 1}{\sclosedvalue}
    }
  \end{mathpar}
  \caption{Step-indexed big-step semantics for base terms of source language $\source$.}
  \label{fig:source-definition-base-terms-semantics}
\end{figure}

In Figure~\ref{fig:source-definition-base-terms-semantics}, the base
terms are evaluated using a step-indexed big-step semantics whose
judgment~$\seval\sectx\sterm{n}\sclosedvalue$ is read ``Under the
environment~$\sectx$, the base term~$\sterm$ evaluates into the closed
value~$\sclosedvalue$ in~$n$~steps.''
%
\rulename{SVar} looks up into~$\sectx$ to return the closed value of a
variable~$\tx$.
%
\rulename{STuple} looks up for~$\many\tx$ into~$\sectx$ to evaluate
the tuple~$\stuple{\many\tx}$, binds it to~$\ty$ and continues with
the next term.
%
\rulename{SPrimitiveCall} assumes that~$\tf$ is mapped to a
primitive~$\sprim$ in~$\sectx$ and that $\sprim$ has an associated
interpretation realized as a function~$\tdelta\sprim\param$ from
closed values to closed values. The evaluation of this interpretation
function applied to the value of~$\tx$ is bound to~$\ty$ in~$\sectx$
in order to evaluate the remaining term~$\sterm$ in~$n$ steps. In that
case, the evaluation of the $\textbf{let}$-binding takes~$n + 1$~steps.
%
\rulename{SClosureCall} assumes that~$\tf$ is mapped
to a closure in~$\sectx$ and evaluates the body of this closure in $n$
steps in the closure environment~$\sectx_\tf$ extended with the value
of its argument~$\tx$ found in~$\sectx$. The resulting value is bound
to~$\ty$ in $\sectx$ to evaluate the remaining term $\sterm$ in~$m$
steps. Hence, the whole term evaluates in~$n + m + 1$~steps.

A closed change value is either a closure change, a tuple change,
a literal change, a replacement change or a nil change.
%
A closure change is a pair made of an evaluation environment~$\iectx$
and a~$\lambda$-abstraction expecting a value and a change value as
arguments to evaluate a change term into an output change value. An
evaluation environment~$\iectx$ follows the same structure as
$\textbf{let}$-bindings of the change terms: it binds variables to
closed values and each variable~$\tx$ is immediately followed by a
binding for its associated change variable~$\tdx$. As with
$\textbf{let}$-bindings of change terms, $\alpha$-renamings in an
environment~$\iectx$ must rename~$\tdx$ into~$\tdy$ if~$\tx$ is
renamed into~$\ty$. With no loss of generality, we assume that all
bound term variables are distinct in these environments.  We define
the \textit{original environment}~$\oldenv\iectx$ of a change
environment~$\iectx$ by induction over~$\iectx$:
\[
\begin{array}{rcl}
\oldenv\sectxempty & = & \sectxempty \\
\oldenv{\iectxcons\iectx{\tx = \sclosedvalue; \tdx = \icloseddvalue}}
& = & \iectxcons{\oldenv{\iectx}}{\tx = \sclosedvalue}
\end{array}
\]

The \textit{new environment}~$\newenv\iectx$ of a change
environment~$\iectx$ is defined by induction over~$\iectx$:
\[
\begin{array}{rcl}
\newenv\sectxempty & = & \sectxempty \\
\newenv{\iectxcons\iectx{\tx = \sclosedvalue; \tdx = \icloseddvalue}}
& = & \iectxcons{\newenv{\iectx}}{\tx = \sclosedvalue \oplus \icloseddvalue}
\end{array}
\]

This last rule makes use of an operation $\oplus$ to update a value
with a change, which may fail at runtime. Indeed, the application of a
change is a partial function written
``$\sclosedvalue \oplus \icloseddvalue$'' which is defined as follows:
\[
  \begin{array}{rclcl}
    \sconst & \oplus & \idconst & = & \delta_{\oplus} (\sconst, \idconst)
    \\
    \sclosedvalue_1 & \oplus & \replace{\sclosedvalue_2} & = & \sclosedvalue_2
    \\
    \sclosedvalue & \oplus & \inil & = & \sclosedvalue
    \\
    \sclosure{\sectx}{\slam\tx\sterm} & \oplus &
    \iclosure\iectx{\ilam{\tx\,\tdx}{\idterm}} & = &
    \sclosure{(\sectx \oplus \iectx)}{\slam\tx\sterm}
    \\
    \stuple{\sclosedvalue_1, \ldots, \sclosedvalue_n}
    & \oplus
    & \stuple{\icloseddvalue_1, \ldots, \icloseddvalue_n}
    & =
    & \stuple{\sclosedvalue_1 \oplus \icloseddvalue_1, \ldots, \sclosedvalue_n \oplus \icloseddvalue_n}
  \end{array}
\]
\noindent where
\[
    (\tectxcons\sectx{\tx = \sclosedvalue}) \oplus
    (\tectxcons\iectx{\tx = \sclosedvalue; \tdx = \icloseddvalue})
    =
    (\tectxcons{(\sectx \oplus \iectx)}{\tx = (\sclosedvalue \oplus \icloseddvalue)})
\]

Nil and replacement changes can be used to update both
constants, tuples, primitives and closures, while tuple change can
only update tuples, literal change can only update literals and
closure changes can only update closures. Indeed, $\oplus$ on literals
is defined \textit{via} some interpretation function~$\delta_{\oplus}$. A
replacement change overrides the current value~$\sclosedvalue$ with a
new one~$\sclosedvalue'$.
Change application for a closure
ignores~$\idterm$ instead of combining it with $\sterm$ somehow. This may seem
surprising, but we only need $\oplus$ to behave well for valid changes (as shown
by \cref{lemma:crel-oplus}): for valid closure changes,
$\idterm$ must behave similarly to $\iderive\sterm$ anyway,\pg{Not formal, not
  visible in the definition, but true for a suitable meaning of ``similarly'';
  but I'm clearly not claiming a theorem, just giving a hint for clarity.}
so only environment updates matter. This definition also avoids having to modify
terms at runtime, which would have been difficult to implement safely.
\pg{$\oplus$ is not used by derivatives, so commenting this part out
A change application for a closure
ignores~$\idterm$, which may seems strange at first but is needed to
but is needed to
get guarantees about the evaluation of derivatives.}
% With this restriction, we avoid the case of a new
% closure code that would refer to a variable undefined in the closure
% environment.
% \pg{I can imagine fixes to that problem, so I'd propose to stick to the
% stronger arguments.}

\begin{figure}
  \footnotesize
  \begin{mathpar}
\infer[\rulename{SDVar}]{
    }{
      \seval\iectx\tdx{}{\sectxlookup\iectx\tdx}
    }

    \infer[\rulename{SDTuple}]{
      \iectxlookup\iectx{\many\tx, \many\tdx} = \many\sclosedvalue_\tx, \many\icloseddvalue_\tx \\\\
      \seval{\sectxcons\iectx
        {
          \ty = \stuple{\many\sclosedvalue_\tx};
          \tdy = \stuple{\many\icloseddvalue_\tx}
        }
      }{\idterm}{}{\icloseddvalue}
    }{
      \seval\iectx{\slet{\ty = \stuple{\many\tx}, \tdy = \stuple{\many\tdx}}{\idterm}}{}{\icloseddvalue}
    }

    \infer[\rulename{SDReplaceCall}]{
      \sectxlookup\iectx\tdf = \replace{\sclosedvalue_\tf} \\\\ 
      \seval{\oldenv\iectx}{\scall{\tf}{\tx}}{n}{\sclosedvalue_\ty} \\\\
      \seval{\newenv\iectx}{\scall{\tf}{\tx}}{m}{{\sclosedvalue_\ty}'} \\\\
      \seval{\iectxcons\iectx{\ty = \sclosedvalue_\ty ; \tdy = \replace{{\sclosedvalue_\ty}'}}}
      {\idterm}
      {}
      {\icloseddvalue}
    }{
      \seval
      \iectx
      {\slet{\ty = \sapp\tf\tx, \tdy = \idapp\tdf{\tx\,\tdx}}{\idterm}}
      {}
      {\icloseddvalue}
    }

    \infer[\rulename{SDClosureNil}]{
      \sectxlookup\iectx{\tf, \tdf} = \sclosure{\sectx_f}{\slam\tx{\sterm_f}}, \inil \\\\
      \seval{\sectxcons{\sectx_f}{\tx = \sectxlookup\iectx\tx}}{\sterm_f}{n}{\sclosedvalue_\ty} \\\\
      \seval{\sectxcons{\sectx_f}{\tx = \sectxlookup{\newenv\iectx}\tx}}{\sterm_f}{m}{{\sclosedvalue_\ty}'} \\\\
      \seval{\iectxcons\iectx{\ty = \sclosedvalue_\ty ; \tdy = \replace{{\sclosedvalue_\ty}'}}}
      {\idterm}
      {}
      {\icloseddvalue}
    }{
      \seval
      \iectx
      {\slet{\ty = \sapp\tf\tx, \tdy = \idapp\tdf{\tx\,\tdx}}{\idterm}}
      {}
      {\icloseddvalue}
    }

    \infer[\rulename{SDPrimitiveNil}]{
      \sectxlookup\iectx{\tf,\tdf} = \sprim, \inil \\\\
      \sectxlookup\iectx{\tx, \tdx} = \sclosedvalue_\tx, \icloseddvalue_\tx \\\\
      \seval
      {\iectxcons
        \iectx
        {\ty = \tdelta\sprim{\sclosedvalue_\tx} ;
          \tdy = \tddelta\sprim{\sclosedvalue_\tx,\icloseddvalue_\tx}}}
      {\idterm}
      {}
      {\icloseddvalue}
    }{
      \seval
      \iectx
      {\slet{\ty = \sapp\tf\tx, \tdy = \idapp\tdf{\tx\,\tdx}}{\idterm}}
      {}
      {\icloseddvalue}
    }

    \infer[\rulename{SDClosureChange}]{
      \sectxlookup\iectx{\tf,\tdf} = \sclosure{\sectx_\tf}{\slam\tx{\sterm_\tf}}, \iclosure{\iectx_\tf}{\ilam{\tx\,\tdx}{\idterm_\tf}} \\\\
      \sectxlookup\iectx{\tx,\tdx} = \sclosedvalue_\tx, \icloseddvalue_\tx \\\\
      \seval{\sectxcons{\sectx_f}{\tx = \sclosedvalue_\tx}}{\sterm_f}{n}{\sclosedvalue_\ty} \\\\
      \seval{\iectxcons{\iectx_\tf}{
          \tx = \sclosedvalue_\tx; 
          \tdx = \icloseddvalue_\tx}}
      {\idterm_\tf}{}{\icloseddvalue_\ty} \\\\
      \seval{\iectxcons\iectx{\ty = \sclosedvalue_\ty ; \tdy = \icloseddvalue_\ty}}{\idterm}
      {}{\icloseddvalue}
    }{
      \seval
      \iectx
      {\slet{\ty = \sapp\tf\tx, \tdy = \idapp\tdf{\tx\,\tdx}}{\idterm}}
      {}
      {\icloseddvalue}
    }
  \end{mathpar}
  \caption{Step-indexed big-step semantics for the change terms of the source language $\source$.}
  \label{fig:source-definition-change-terms-semantics}
\end{figure}


In Figure~\ref{fig:source-definition-change-terms-semantics}, change
terms are evaluated using a big-step semantics whose
judgment~$\seval\iectx\idterm{}\icloseddvalue$ is read ``Under the
environment $\iectx$, the change term $\idterm$ evaluates into the
closed change value $\icloseddvalue$.''
%
$\rulename{SDVar}$ looks up into $\iectx$ to return a value for $\tdx$.
%
$\rulename{SDTuple}$ builds a tuple out of the values of~$\many\tx$
and a change tuple out of the change values of~$\many\tdx$ as found
in the environment~$\iectx$.
%
There are four rules to evaluate $\textbf{let}$-binding depending on
the nature of~$\iectxlookup\iectx\tdf$. These four rules
systematically recomputes the value~$\sclosedvalue_\ty$ of~$\ty$ in
the original environment. They differ in the way they compute the
change~$\tdy$ to~$\ty$.

If $\iectxlookup\iectx\tdf$ is a replacement,
$\rulename{SDReplaceCall}$ applies. Replacing the value of~$\tf$
in the environment forces the recomputation of~$\tf\,\tx$ from
scratch in the new environment. The resulting value~$\sclosedvalue'_\ty$
is the new value which must replace~$\sclosedvalue_\ty$: therefore
$\tdy$ equals~$\replace{\sclosedvalue'_\ty}$ in that case.

If $\iectxlookup\iectx\tdf$ is a nil change. We have two rules
depending on the nature of $\iectxlookup\iectx\tf$. If
$\iectxlookup\iectx\tf$ is a closure, $\rulename{SDClosureNil}$
applies and in that case the nil change of~$\iectxlookup\iectx\tf$ is
the exact same closure. Hence, to compute $\tdy$, we must reevaluate
this closure applied to the updated
argument~$\iectxlookup{\newenv\iectx}\tx$ to a
value~$\sclosedvalue'_\ty$ and bind~$\tdy$ to
$\replace{\sclosedvalue'_\ty}$. In other words, this rule is
equivalent to $\rulename{SDReplaceCall}$ in the case where a closure
is replaced by itself.
%
If $\iectxlookup\iectx\tf$ is a primitive, $\rulename{SDPrimitiveNil}$
applies. The nil change of a primitive~$\sprim$ is its derivative
which interpretation is realized by a
function~$\Delta_\sprim(\param)$.  The evaluation of this function on
the input value and the input change leads to the
change~$\icloseddvalue_\ty$ bound to $\tdy$.

If $\iectxlookup\iectx\tf$ is a closure change
$\iclosure{\iectx_\tf}{\ilam{\tx\,\tdx}{\idterm_\tf}}$,
$\rulename{SDClosureChange}$ applies. The
change~$\icloseddvalue_\ty$ results from the evaluation
of~$\idterm_\tf$ in the closure change environment~$\iectx_\tf$
augmented with an input value for~$\tx$ and a change value
for~$\tdx$. Again, let us recall that we will maintain the
invariant that the term~$\idterm_\tf$ behaves as the derivative
of~$\tf$ so this rule can be seen as the invokation of~$\tf$'s
derivative.

\paragraph{Expressiveness}

A closure in the initial environment can be used to represent a
top-level definition. Since environment entries can point to
primitives, we need no syntax to directly represent calls of
primitives in the syntax of base terms. To encode in our syntax a
program with top-level definitions and a term to be evaluated
representing the entry point, one can produce a term~$\sterm$
representing the entry point together with an environment~$\sectx$
containing as values any top-level definitions, primitives and
constants used in the program.

Our formalization does not model directly $n$-ary functions, but they can be
encoded through unary functions and tuples. This encoding does not support
currying efficiently, but we discuss possible solutions in
\cref{sec:nary-abstraction}.

Control operators, like recursion combinators or branching, can be
introduced as primitive operations as well. If the branching condition changes,
expressing the output change in general requires replacement changes. Similarly
to branching we can add tagged unions.
