\subsection{A new soundness proof for static differentiation}
\label{sec:sound-derive}

As in \citet{CaiEtAl2014ILC}'s development, static differentiation is
only sound on input changes that are \emph{valid}. However, our
definition of validity needs to be significantly different.
%
\citeauthor{CaiEtAl2014ILC} prove soundness for a strongly normalizing
simply-typed $\lambda$-calculus using denotational semantics. We generalize this
result to an untyped and Turing-complete language using purely syntactic
arguments. In both scenarios, a function change is only valid if it turns valid
input changes into valid output changes, so validity is a logical relation.
Since standard logical relations only apply to typed languages, we turn to
\emph{step-indexed} logical relations.

\subsubsection{Validity as a step-indexed logical relation}
To state and prove soundness of differentiation, we define validity by
introducing a ternary step-indexed relation over base values, changes and
updated values, following previous work on step-indexed
logical relations~\citep{Ahmed2006stepindexed,Acar08}. Experts might notice
small differences in our step-indexing, as mentioned in
%\pg{Maybe move this into sec:cts-rw.}
\cref{sec:sourcelanguage}, but they do not affect the substance of the proof.
% \pg{We are not using biorthogonality: that notion embeds contextual equivalence
%   inside the definition of equivalence.}
%biorthogonality~\citep{pitts2010step}.
%
We write
\[\vvcrel {\sclosedvalue_2} k {\sclosedvalue_1} \icloseddvalue\]
\noindent and say that ``$\icloseddvalue$ is a valid change from
$\sclosedvalue_1$ to $\sclosedvalue_2$, up to $k$ steps'' to mean that
$\icloseddvalue$ is a change from $\sclosedvalue_1$ to
$\sclosedvalue_2$ and that $\icloseddvalue$ is a \emph{valid}
description of the differences between~$\sclosedvalue_1$ 
and~$\sclosedvalue_2$, with validity tested with up to~$k$~steps.
%
To justify this intuition of validity, we later state and prove two
lemmas: a valid change from~$\sclosedvalue_1$ to~$\sclosedvalue_2$
goes indeed from~$\sclosedvalue_1$ to~$\sclosedvalue_2$
(Lemma~\ref{lemma:crel-oplus}), and if a change is valid up to~$k$~steps it
is also valid up to fewer steps (Lemma~\ref{lemma:vvcrel-antimono}).

\begin{lemma}[$\oplus$ agrees with validity]
  \label{lemma:crel-oplus}
  If
  $\vvcrel {\sclosedvalue_2} k {\sclosedvalue_1} \icloseddvalue$ then
  $\sclosedvalue_1 \oplus \icloseddvalue = \sclosedvalue_2$.
\end{lemma}
\begin{lemma}[Downward-closure]
  \label{lemma:vvcrel-antimono}
  If $N \ge n$, then $\vvcrel{\sclosedvalue_2} {N} {\sclosedvalue_1} \icloseddvalue$ implies
  $\vvcrel{\sclosedvalue_2} {n} {\sclosedvalue_1} \icloseddvalue$.
\end{lemma}

\input{\poplPath{differentiation-step-indexed-relation}}

As usual with step-indexed logical relations, validity is defined by
well-founded induction over naturals ordered by $<$. To show this, it helps to
observe that evaluation always takes at least one step.

Validity is formally defined by cases in
Figure~\ref{fig:step-index-relation-between-values-and-changes}; we
describe in turn each case. First, a constant change $\idconst$ is a
valid change from $\sconst$ to
$\sconst \oplus \idconst = \delta_{\oplus} (\sconst, \idconst)$.
Since the function $\delta_\oplus$ is partial, the relation only holds
for the constant changes $\idconst$ which are valid changes for
$\sconst$.
%
Second, a replacement change~${\replace{\sclosedvalue_2}}$ is always a valid
change from any value~$\sclosedvalue_1$ to $\sclosedvalue_2$.
%
Third, a nil change is a valid change between any value and itself.
%
Fourth, a tuple change is valid up to step~$n$, if each of its components
is valid up to any step strictly less than~$k$.
%
Fifth, we define validity for closure changes. Roughly
speaking, this statement means that a closure change is valid if (i)
its environment change~$\iectx$ is valid for the original closure
environment~$\sectx_1$ and for the new closure environment~$\sectx_2$;
(ii) when applied to related values, the closure \textit{bodies}~$\sterm_1$
and~$\sterm_2$ are related by~$\idterm$. The validity relation between
terms is defined as follows:
\[
\begin{array}{l}
  \vvcrelterm{\sectx_1}{\iectx}{\sectx_2}{n}{\sterm_1}{\idterm}{\sterm_2} \\
  \textrm{if and only if }
  \forall k < n, \sclosedvalue_1, \sclosedvalue_2, \\
  \quad\seval{\sectx_1}{\sterm_1}{k}{\sclosedvalue_1} \textrm{ and }
  \seval{\sectx_2}{\sterm_2}{}{\sclosedvalue_2} \\
  \quad\textrm{implies that } \exists \icloseddvalue,
  \seval{\iectx}{\idterm}{}{\icloseddvalue} \wedge
  \vvcrel{\sclosedvalue_2}{n - k}{\sclosedvalue_1}{\icloseddvalue}
\end{array}
\]

\noindent
We extend this relation from values to environments by defining a
judgment $\vvcrel {\sectx_2} k {\sectx_1} \iectx$ defined as follows:
\begin{mathpar}
  \infer{}{
    \vvcrel \bullet k \bullet \bullet
  }

  \infer{
    \vvcrel {\sectx_2} k {\sectx_1} \iectx
    \\
    \vvcrel {\sclosedvalue_2} k {\sclosedvalue_1} \icloseddvalue
  }{
    \vvcrel
    {\tectxcons{\sectx_2}{\tx = \sclosedvalue_2}}
    k
    {(\tectxcons{\sectx_1}{\tx = \sclosedvalue_1})}
    {(\tectxcons\iectx{\tx = \sclosedvalue_1; \tdx = \icloseddvalue})}
  }
\end{mathpar}

The above lemmas about validity for values extend to environments.
\begin{lemma}[$\oplus$ agrees with validity, for environments]
  \label{lemma:crel-oplus-env}
  If
  $\vvcrel {\sectx_2} k {\sectx_1} \iectx$ then
  $\sectx_1 \oplus \iectx = \sectx_2$.
\end{lemma}
\begin{lemma}[Downward-closure, for environments]
  \label{lemma:vvcrel-antimono-env}
  If $N \ge n$, then $\vvcrel{\sectx_2} {N} {\sectx_1} \iectx$ implies
  $\vvcrel{\sectx_2} {n} {\sectx_1} \iectx$.
\end{lemma}

Finally, for both values, terms and environments, omitting the step
count~$k$ from validity means validity holds for all~$k$s. That is,
for instance,
$\vvcrel {\sclosedvalue_2} {} {\sclosedvalue_1} \icloseddvalue$ means
$\vvcrel {\sclosedvalue_2} {k} {\sclosedvalue_1} \icloseddvalue$ for
all $k$.

\subsubsection{Soundness of differentiation}

\yrg{This section will be expanded a bit to give a bit more insight about the proof.}

We can state a soundness theorem for differentiation without
mentioning step-indexes. Instead of proving it directly, we must first
prove a more technical statement
(Lemma~\ref{lemma:fundamental-property}) that mentions step-indexes
explicitly.
%
\begin{theorem}[Soundness of differentiation in $\source$]
  \label{thm:sound-derive}
  If $\iectx$ is a valid change environment from base environment $\sectx_1$ to
  updated environment $\sectx_2$, that is $\vvcrel {\sectx_2} {} {\sectx_1}\iectx$,
  %
  and if $\sterm$ converges both in the base and updated environment, that is
  $\sneval{\sectx_1}\sterm{\sclosedvalue_1}$ and
  $\sneval{\sectx_2}\sterm{\sclosedvalue_2}$,
  %
  then $\iderive\sterm$ evaluates under the change environment~$\iectx$ to a valid
  change $\icloseddvalue$ between base result ${\sclosedvalue_1}$ and updated result
  ${\sclosedvalue_2}$, that is
  $\sneval\iectx{\iderive\sterm}{\icloseddvalue}$,
  $\vvcrel{\sclosedvalue_2} {} {\sclosedvalue_1}
  \icloseddvalue$ and
  $\sclosedvalue_1 \oplus \icloseddvalue = \sclosedvalue_2$.
\end{theorem}

\begin{lemma}[Fundamental Property]
  \label{lemma:fundamental-property}
For each $n$,
if $\vvcrel{\sectx_2}{n}{\sectx_1}{\iectx}$ then
$\vvcrelterm{\sectx_1}{\iectx}{\sectx_2}{n}{\sterm}{\iderive{\sterm}}{\sterm}$.
\end{lemma}
