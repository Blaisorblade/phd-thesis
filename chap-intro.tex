\chapter{Introduction}

Many program perform queries on collections of data, and for non-trivial amounts
of data it is useful to execute the queries efficiently. When the data is
updated often enough, it can also be useful to update the results of some
queries \emph{incrementally} whenever the input changes, so that up-to-date
results are quickly available, even for queries that are expensive to execute.

For instance, a program manipulating anagraphic data about citizens of a country
might need to compute statistics on them, such as their average age, and update
those statistics when the set of citizens changes.

Traditional relational database management systems (RDBMS) support both queries
optimization and (in quite a few cases) incremental update of query results
(called there \emph{incremental view maintenance}).

However, often queries are executed on collections of data that are not stored
in a database, but in collections manipulated by some program. Moving in-memory
data to RDBMSs typically does not improve
performance~\citep{Stonebraker07,Rompf2015functional}, and reusing database
optimizers is not trivial.

Moreover, many programming languages are far more expressive than RDBMSs.
Typical RDBMS can only manipulate SQL relations, that is multisets (or bags) of
tuples (or sometimes sets of tuples, after duplicate elimination). Typical
programming languages (PL) support also arbitrarily nested lists and maps of
data, and allow programmers to define new data types with few restrictions; a
typical PL will also allow a far richer set of operations than SQL.

However, typical PL do not apply typical database optimizations to collection
queries, and if queries are incrementalized, this is often done by hand, even
though code implementing incremental query is error-prone and
hard-to-maintain~\citep{Salvaneschi13reactive}.

What's worse, some of these manual optimizations are best done over the whole
program. For instance, adding an index on some collection can speed up looking
for some information, but each index must be maintained incrementally when the
underlying data changes. Depending on the actual queries and updates performed
on the collection, and on how often they happen, it might turn out that updating
an index takes more time than the index saves; hence the choice of which indexes
to enable depends on the whole program. However, adding/removing an index
requires updating all PL queries to use it, while RDBMS queries can use an index
transparently.

Therefore, in this thesis we propose techniques for optimizing collection
queries and executing them incrementally.\pg{Refine}

We consider the problem for functional programming languages such as Haskell or
Scala, and we consider collection queries written using the APIs of their
collection libraries, which we treat as an embedded domain-specific language (EDSL). Such
APIs (or DSLs) contain powerful operators on collections such as $\Varid{map}$,\pg{exemplify or take for granted?}
which are higher-order, that
is they take as arguments arbitrary functions in the host language that we must
also handle.

Therefore, our optimizations and incrementalizations must handle programs in
higher-order EDSLs that can contain arbitrary code in the host language. Hence,
many of our optimizations will focus\pg{what's the ``focus'' of an optimization??}
on properties of our collection EDSL, but
will need to handle host language code. \pg{What's the focus/flow here?}We restrict the problem to purely
functional programs (without mutation or other side effects, including
non-termination), because such programs can be more ``declarative'' and because
avoiding side effects can enable more powerful optimization and simplify the
work of the optimizer at the same time.

In particular, this thesis is divided into two parts:
\begin{itemize}
\item we first describe, in \cref{sec:ch-aosd13}, work on optimizing collection queries by static program transformation.
\item in the remaining chapters, we describe work on incrementalizing programs
  by static program transformation. This thesis presents the first approach that
  handles higher-order programs by using program transformation; hence, while
  our main examples use collection queries, we phrase the work in terms of
  $\lambda$-calculi with unspecified primitives.
\end{itemize}
% Modern libraries offer high-level APIs for manipulating (in-memory) collections;
% equivalently, such an API is an embedded domain-specific language (EDSL).

% Compared to SQL, EDSL queries will typically be able to use user-defined
% functions in queries. Moreover, the author of EDSL queries can enjoy abstraction
% mechanisms from the underlying language.

% \pg{Examples!}

% Collection operations are typically executed directly; however, they can instead
% construct a query representation that can then be optimized.

% \section{Our design for incrementalizing queries}
% We designed our incrementalization system by abstracting from needs and ideas for
% collection APIs.
% \begin{itemize}
% \item collection APIs are typically higher-order. This allows further
%   flexibility compared to first-order query languages: in particular we can
%   design new operators in terms of existing ones.
% \item many collection types support equations that can be used for optimization.
%   For instance
% \end{itemize}

% \subsection{Domain-specific}
% Special optimizations are possible because many collection datatypes, equality
% is not purely structural. Two lists are equal if they are structurally equal.
% But two sets are equal if they have the same elements, which does not imply they
% are structurally equal.\footnote{\pg{Cool but not necessarily appropriate here.}
%   Technically, the datatypes are not freely generated and their signatures
%   contain further equations.}
% %
% A general-purpose optimizer cannot exploit this to return a structurally
% different set with ``the same meaning'', but a domain-specific optimizer can be
% instructed to do so.

% Taking this into account we were led to design our incrementalization system to
% allow domain-specific support from the start.

%%% Local Variables:
%%% mode: latex
%%% TeX-master: "thesis-main"
%%% End:
